\documentclass[twoside]{scrartcl}

\usepackage[blacklinks]{drangreport}
% \usepackage[caecilia]{drangreport}

\usepackage{stmaryrd}

\usepackage{awlc-algebra}
\setcounter{section}{-1}

%%%%%%%%%%%%%%%%%%%%%%%%%%%%%%%%%%%%
%Messy stuff for plotting 
\newcommand{\vasymptote}[2][]{
    \draw [densely dashed,#1] ({rel axis cs:0,0} -| {axis cs:#2,0}) -- ({rel axis cs:0,1} -| {axis cs:#2,0});
}

\newcommand{\hasymptote}[2][]{
    \draw [densely dashed,#1] ({rel axis cs:0,0} -| {axis cs:0,#2}) -- ({rel axis cs:1,0} -| {axis cs:0,#2});
}


\pgfplotsset{
    double y domain/.code 2 args={
        \pgfmathsetmacro\doubleymin{#1*2}
        \pgfmathsetmacro\doubleymax{#2*2}
    },
    ejes/.style args={#1:#2 #3:#4}{
        double y domain={#3}{#4},
        domain=#1:#2,
        ymin=#3,ymax=#4, restrict y to domain=\doubleymin:\doubleymax,
        samples=60,
        enlargelimits=false,
        axis lines=middle,
        xticklabels=\empty, yticklabels=\empty,
        every axis plot post/.style={
            black,
            mark=none,
            smooth
        },
        scale only axis,
        width=6cm,
        height=6cm
    }
}
%%%%%%%%%%%%%%%%%%%%%%%%%%%%%%%%


\title{Methods I}

\let\Im\relax
\DeclareMathOperator*{\Im}{Im}

\newcommand{\herm}[1]{\overline{#1}^\Trans}

\DeclareMathOperator*{\ch}{ch}
\renewcommand{\chi}{\textstyle\ch}

\begin{document}

\NotesTitle{}{Autumn 2014}{Mathematical Methods I}{Prof.~J.}{Mestel}
{
{\Large\bfseries\sffamily{Syllabus}}

\textit{The course supplies a firm grounding to A-level topics such as differentiation, integration, complex numbers and series expansions.
}



\begin{itemize}
\item Functions: Polynomial, rational, exponential, logarithmic, trigonometric and hyperbolic functions. Odd, even and inverse functions.

\item Limits: basic properties and evaluation. Continuity \& discontinuous functions.

\item Differentiation: First principles, differentiability; logarithmic and implicit differentiation; higher derivatives; Leibniz's formula; stationary points and points of inflexion; curve sketching; parametric representation, polar co-ordinates.

\item Power Series Expansions: The Mean Value Theorem. Taylor's Theorem with remainder. Infinite power series, radius of convergence. Ratio test; Taylor and Maclaurin Series. De l'Hopital's rule.
\itemIntegration: definition as Riemann limit; indefinite & definite integrals; the fundamental theorem of calculus; integration by substitution and by parts; partial fractions; Existence of improper and infinite integrals. Integrals over areas and volumes.
\item Complex Numbers: definition; the complex plane; standard and polar representation; de Moivre's Theorem; exp(z) and log(z)

\item First order differential equations. Separable, homogeneous and linear equations. Special cases. Linear higher order equations with constant coefficients,
\end{itemize}

Course content at \url{http://www.ma.ic.ac.uk/~ajm8/M1M1}\\

\subsection*{Appropriate books}

{\shortskip
M.~Liebeck \emph{A Concise Introduction to Pure Mathematics}.

K.~Houston \emph{How to Think Like a Mathematician}. 

E.~Hurst and M.~Gould \emph{Bridging the Gap to University Mathematics}. 

None are particularly recommended. The course is completely self-contained. But Stephenson's ``Maths for Science Students'', and Kreyszig's of a similar name are useful. 

}}

\TableofContents

%!TEX root = linear-algebra.tex
\stepcounter{lecture}
\setcounter{lecture}{1}
 \section{Vectors in $\R^2$}

 $\R =$ the set of real numbers. \marginpar{\footnotesize \textit{Lecture 1 14/10} } This has the properties of being:\begin{itemize}
\item Commutative: $ab = ba$
\item Associative: $a+(b+c) = (a+b) + c	$
\item Distributive: $c(a+b) = ca + cb$
\item If $a<b,~b>0$ then $ca < cb$
\end{itemize}

$\R^2 = $ the set of ordered pairs $(a,b)$ where $a,b \in \R$ i.e. $(a,b) \neq (b,a)$ in general. Elements of $\R^2$ are \emph{points}, or \emph{vectors}. $\mathbf{0} = (0,0) = $ origin of $\R^2$. A \emph{scalar} is an element of $\R$.\\

Properties:\begin{enumerate}
\item Addition (sum): $(a,b) + (a',b') = (a+a',b+b')$%picture
\item Scalar Multiplication: If $\lambda = \R~ (a,b) \in \R^2$, then $\lambda \cdot (a,b) = (\lambda a,\lambda b)$. $v \in \R^2 \implies \lambda v \in \R^2$.	
\end{enumerate}
\textit{Exercise:} Check $\lambda(v_1+v_2) = \lambda v_1 + \lambda v_2$ where $\lambda \in \R,~ v_1,v_2 \in \R^2$\\

\noindent \textbf{Lines in $\R^2$.} $L$ is a \emph{line in $\R^2$}\index{Line in $RR^2$} if $\exists u, v \in \R^2~ v \neq 0$ with $L = \{u + \lambda v ~|~ \lambda \in \R\}$. If $u = (a_1,b_1),~v = (a_2,b_2)$ then $L = \{(a_1 + \lambda b_1, a_2 + \lambda b_2) ~|~ \lambda \in \R\}$. \\ %picture really needed for example

\textit{Examples:} $x$ line $\{(0,0) + \lambda(1,0) ~|~ \lambda \in \R\}$, $y$ line $\{(0,0) + \lambda (0,1) ~|~ \lambda \in \R\}$. $L = \{x+y = 1\} = \{(1,0) + \lambda(-1,1) ~|~ \lambda \in \R\}.$ Note that $v$ is the vector for which $L$ is parallel to. We check from the components $x = 1 + (-1) \lambda = 1 - \lambda$. $y = 0 + 1 \cdot \lambda = \lambda$. So $x + y = 1$.\\

\noindent Assume $L, M$ are two lines in $\R^2$, with $L = \{ u = \lambda v ~|~ \lambda \in \R\}$ and $v \neq 0$, $M = \{a + \mu b ~|~ \mu \in \R\}$ and $b \neq 0$.

\pagebreak

\noindent \textbf{Proposition.} The two lines are the same ($L = M$) if and only if the following holds:\begin{enumerate}
\item $v = \alpha b$ for some $\alpha \in \R$
\item $L \cap M \neq \emptyset$ ($L$ and $M$ have a point in common) 	
\end{enumerate}
\begin{proof}
($\implies$) Assume that $L= M$. We know that $u \in L \implies u \in M \implies u = a + \mu b$ for some $\mu \in \R$. Also $u + v = L ~(\lambda = v) \implies u + v \in M \implies u + v = a + \mu_1b$ for some $\mu_1 \in \R \implies v = (a+\mu _1b) - (a-\mu b) = (\mu _1 - \mu )b = \alpha b$. Since $L=m$, surely $L \cap M = \emptyset$.\\

($\impliedby$) Assume $v = \alpha b \implies L = \{u + \lambda \alpha b ~|~ \lambda \in \R\}$. We also know that $L \cap M \neq \emptyset \implies \exists c \in L \cap M \implies c \in L \implies c = u + \lambda_0\alpha b$, for some $\lambda_0 \in \R.$ Then also $c \in M \implies c = a + \mu b$, for some $\mu \in \R \implies u + \lambda_0 \alpha b = a + \mu b \implies u = a + \mu b - \lambda_0\alpha b = a + (\mu - \lambda_0\alpha)b ~(*)$. \\

By $(*)$, a point inside $L = u + \lambda \alpha b = a + (\mu - \lambda_0 \alpha) b + \lambda \alpha b = a + (\mu - \lambda_0 \alpha  + \lambda \alpha)b \in M$. Hence any point in $L$ is inside $M$, so $L \subseteq M$. Similarly by symmetry $M \subseteq L \implies L = M$.
\end{proof}

\sektion{Matrices}
\subsektion{Inverses}
\noindent \textbf{Theorem 5.1.} Let $A$ be a square matrix. If there exists a square matrix $B$ such that $AB = I$ then this $B$ is unique and satisfies $BA = I$.
\begin{proof}
(Also gives a method for finding this $B$!)\\	Let $X$ be the square matrix with unknown entires. We want to solve the equation $AX = I$. The entires of $X$ are $x_{ij}$. We have $n^2$ unknowns and $n^2$ equations. We record this as follows: $(A ~|~ I)$, a $n \times 2n$ matrix.\\

This is $n$ systems of linear equations in $n$ variables, e.g. for each column of $I$ we have the following system:

\[A\left(\begin{smallmatrix}
x_{1j}\\x_{2j}\\ \vdots \\ x_{nj}	
\end{smallmatrix}
\right)= \text{ the } j\text{th column of } I = \left(\begin{smallmatrix}
0\\ \vdots \\ 1 \\ 0 \\ \vdots \\ 0
\end{smallmatrix}\right)  \addtag
\] 

All these $n$ systems have the same co-efficient matrix $A$, so we solve them using the same process. Apply reduction to echelon form. Perform elementary row operations on the matrix $(A ~|~ I)$

\[(A_{ech} ~|~ I) =\left(
    \begin{array}{ccccc |}
    1    &       &    &     & \\ \cline{1-1}
    \bord & 1       &    &  \scalebox{1.5}{*}   & \\ \cline{2-2}
          & \bord    & 1     &    & \\ \cline{3-3}
       ~\scalebox{1.5}{$0$}   &  & \bord & 1     &  \\ \cline{4-5}
          &          &       &  &  \\ %\cline{5-5}
  \end{array} ~~D \right) \]

\noindent \textbf{Claim:} The matrix on the LHS cannot have any rows made entirely of zeros.
\begin{proof}[Proof of Claim]
Remember that $D$ is obtained by row operations from $I$. We know that two matrices that are obtained from each other by row operations define equivalent linear systems. This means that the linear system $I(y_1,\dots,y_n) = 0$ has the same solutions as $D(y_1,\dots,y_n) = 0$. But $(0,\dots,0)$ is the only solution to this. Now if $D$ has an all zero row, the system $D(y_1,\dots,y_n) = 0$ has free variables, hence infinitely many solutions. This contradiction proves that $D$ does not have an all zero row. 
\end{proof}

Therefore there is a non-zero entry in the bottom row of $D$. Say this entry is in the $j$th column. Then the system (1.1) has no solutions (follows from the echelon form method). Therefore, if the matrix on the left has a bottom row made of zeros, then $AX = I$ has no solutions. So $A$ has on right inverse. It remains to consider the case when the matrix on the left has no all-zero rows:

\[(A_{ech} ~|~ I) =\left(
    \begin{array}{ccccc |}
    1    &       &    &     & \\ 
  & 1       &    &  \scalebox{1.5}{*}   & \\ 
          &    & \ddots     &    & \\ 
       ~\scalebox{1.5}{$0$}   &  &  & 1     &  \\ 
  \end{array} ~~D \right) \]
  
 Perform more elementary row operations to clear the entries above the main diagonal (This is possible because all diagonal entries equal $1$). After this step, we obtain: 
\[(I~|~ E) =\left(
    \begin{array}{ccccc |}
    1    &       &    &     & \\ 
  & 1       &    &  \scalebox{1.5}{$0$}   & \\ 
          &    & \ddots     &    & \\ 
       ~\scalebox{1.5}{$0$}   &  &  & 1     &  \\ 
  \end{array} ~~E \right) \]
  
  Since row operations don't change the solution of our linear system, we have $IX = E$. Hence $E$ is a unique solution of the system $AX = I$, i.e. $AE = I$. We've proven that if the right inverse exists it can be obtained by the procedure, and it is unique.\\
  
   \noindent Finally we now prove that $EA = I$:\\
 
   Consider the equation $EY = I$, where $Y = (y_{ij})$ is a square matrix with unknown entires $y_{ij}$.  Reverse row operations from the first part of the proof to so $(E ~|~ I) \mapsto (I ~|~ A)$.  $EY = I$ is equivalent to $IY = A$, that is $Y = A$. Therefore $EA = I$. 
\end{proof}~\\

Finding inverses of $2 \times 2$ matrices is easy: 

\[A = \begin{pmatrix}
 a & b \\ c & d	
 \end{pmatrix} \text{, consider }
B= \begin{pmatrix}
 d & -b \\ -c & a	
 \end{pmatrix}
\]
Then \[
AB = \begin{pmatrix}
 a & b \\ c & d	
 \end{pmatrix} \begin{pmatrix}
 d & -b \\ -c & a	
 \end{pmatrix} = \begin{pmatrix}
 ad -bc & 0\\ 0 & ad-bc	
 \end{pmatrix} = (ad -bc)I
\]

\begin{enumerate}
\item[Case 1.] $(ad -bc)$ is non-zero. Then\[
\frac{1}{ad-bc}\begin{pmatrix}
d & -b \\ -c & a	
\end{pmatrix}
\text{ is the inverse of } A
\]	
\item[Case 2.] $ad -bc = 0$. In this case $AB = 0$. Then the inverse does not exist (If $CA = I$, then $C(AB) = (CA)B = IB = B.$ If $A$ is non-zero then $B$ is non-zero, so we get a contradiction for $0 = B$.) Hence $A$ is not invertible. 
\end{enumerate}


\subsektion{Determinants}
\noindent \textbf{Corollary 6.4.} If $A'$ is obtained from $A$ by row operations, then det$(A') \neq 0$ if and only if det$(A) \neq 0$.
\begin{proof}
A direct consequence of Proposition 6.3.	
\end{proof}\vspace*{10pt}

\noindent \textbf{Theorem 6.5.} Let $A$ be a $3 \times 3$ matrix. Then $A^{-1}$ exists if and only if det$(A) \neq 0$.
\begin{proof}
Recall that $A$ can be reduced to echelon form by row operations. Let $A'$ be the matrix in echelon form to which $A$ reduces. Then det$(A')\neq 0 \iff$ det$(A) \neq 0$. Hence we are in Case 2 (If $A'$ has an all zero-row then we expand in this row and det$(A') = 0$.) In Case 2, $A'$ can be reduced to $I$ by further row operations. By Theorem 5.1, $A$ is invertible.\\

We  \marginpar{\footnotesize \textit{Lecture 13 11/11} } need to prove that if $A^{-1}$ exists, then det$(A) \neq 0$. Indeed, if $A^{-1}$ exists, then the echelon form of $A$ has no all-zero rows. Then $A$ can be reduced to $I$ by row operations. Row operations can only multiply det by a non-zero number, and they can be reversed. Therefore, det $(A) \neq 0$.	
\end{proof}\vspace*{10pt}

\noindent \textit{Remark.} For any square matrices $A$ and $B$ of the same size det$(AB) =$ det$(A)$det$(B)$. If $A^{-1}$ exists, then $AA^{-1} = I$, so det$(A)$det$(A^{-1}) = 1$. Hence det$(A) \neq 0$ if $A^{-1}$ exists.\\

\noindent \textit{Final Comment.} If $A$ is a square matrix, then $Ax = 0$ has non-zero solutions if and only if det$(A) = 0$. (Indeed if $Ax = 0$ has a non-zero solution, then it has at least two distinct solutions, so it has infinitely many solutions. Then $A^{-1}$ does't exist, and det$(A) = 0$). 

\pagebreak

\subsektion{Eigenvalues and Eigenvectors}

\noindent \textbf{Definition.} Let $A$ be a $n \times n$ matrix. Then a non-zero vector, $v$, is called an \emph{eigenvector}\index{Eigenvector} of $A$ if $Av = \lambda v$ for some $\lambda \in \R$. In this case $\lambda$ is called an \emph{eigenvalue}\index{Eigenvalue} of $A$ corresponding to the eigenvector $v$.\\

\noindent \textit{Remarks.} A scalar multiple of an eigenvector is also an eigenvector with the same eigenvalue.\\

\noindent \textbf{Example.} \\

\noindent \textbf{Definition.} The determinant of $tI - A$ is called the \emph{characteristic polynomial}\index{Characteristic Polynomial} of $A$ (For us $n = 3,$ or $n=2$). For example\\

\noindent \textbf{Proposition 7.1.} Let $A$ be a $2 \times 2$ or $3 \times 3$ matrix. Then the eigenvalues of $A$ are the roots of the characteristic polynomial of $A$, i.e. every eigenvalue $\lambda$ satisfies det$(\lambda I - A) = 0$. The eigenvectors of $A$ with eigenvalue $\lambda$ are non-zero solutions of the system of linear equations $(\lambda I - A)v = 0$.

\begin{proof}
The real numbers $\lambda$ for which det$(\lambda I - A) = 0$ are by definition the roots of the characteristic polynomial of $A$. Hence $v$ is a non-zero solution of $(\lambda I - A)v = 0$.	
\end{proof}


		% Vector spaces
\input{linalg-chap-02}		% Endomorhisms


\end{document}
%!TEX root = linear-algebra.tex
\stepcounter{lecture}
\setcounter{lecture}{4}
\sektion{Duals}
This chapter really belongs after chapter 1 -- it's just definitions and intepretations of row reduction.

\lecturemarker{16}{9 Nov}
\begin{definition}
	Let $V$ be a vector space over a field $\F$. Then %
	\begin{equation*}
		V^* = \Lin(V,\F) = \left\{\text{linear functions } V\to\F\right\}
	\end{equation*}
	is the \emph{dual space} of $V$.
\end{definition}

\begin{examples}
\mbox{}
\begin{enumerate}
	\shortskip
	\item Let $V=\R^3$. Then $(x,y,z) \mapsto x-y$ is in $ V^*$.
	\item If $V=C([0,1])=\left\langle \text{continuous functions } [0,1]\to\R \right\rangle$, then $f\mapsto\int_0^1 f(t)\dif{t}$ is in $C([0,1])^*$. %
\end{enumerate}
\end{examples}

\begin{definition}
	Let $V$ be a finite dimensional vector space over $\F$, and $v_1,\ldots,v_n$ be a basis of $V$. Then define $v_i^*\in V^*$ by %
	\begin{equation*}
		v_i^*(v_j) = \delta_{ij} =
		\begin{cases}
			0 & \text{if } i\neq j, \\ %
			1 & \text{if } i=j,
		\end{cases}
	\end{equation*}
	and extend linearly. That is, $v_i^*\left( \sum_j \lambda_j\,v_j \right) = \lambda_i$.
\end{definition}

\begin{lemma}
	The set $v_1^*,\ldots,v_n^*$ is a basis for $V^*$, called the \emph{basis dual to} or \emph{dual basis for} $v_1,\ldots,v_n$. In particular, $\dim V^*=\dim V$. %
\end{lemma}

\begin{proof}
	Linear independence: if $\sum \lambda_i\,v_i^*=0$, then $0=\left( \sum \lambda_i \, v_i^* \right) (v_j) = \lambda_j$, so $\lambda_j=0$ for all $j$. Span: if $\varphi\in V^*$, then we claim %
	\begin{equation*}
		\varphi = \sum_{j=1}^n \varphi(v_j) \cdot v_j^*.
	\end{equation*}
	As $\varphi$ is linear, it is enough to check that the right hand side % of the above
	applied to $v_k$ is $\varphi(v_k)$. But %
	\begin{equation*}
		\textstyle \sum_j \varphi(v_j) \, v_j^* (v_k) = \sum_j \varphi(v_j) \, \delta_{jk} = \varphi(v_k). \qedhere %
	\end{equation*}
\end{proof}

\begin{remarks}
\mbox{}
\begin{enumerate}
	\item We know in general that $\dim \Lin(V,W)=\dim V \dim W$.
	\item If $V$ is finite dimensional, then this shows that $V\isom V^*$, as any two vector spaces of dimension $n$ are isomorphic. But they are not \emph{canonically} isomorphic (there is no natural choice of isomorphism). %
	
	If the vector space $V$ has more structure (for example, a group $G$ acts upon it), then $V$ and $V^*$ are not usually isomorphic in a way that respects this structure. %
	
	\item If $V=F[x]$, then $V^* \isomto \F^\N$ by the isomorphism $\theta\in V^* \mapsto (\theta(1),\theta(x),\theta(x^2),\ldots)$, and conversely, if $\lambda_i\iF$, $i=0,1,2,\ldots$ is any sequence of elements of $\F$,  we get an element of $V^*$ by sending $\sum a_i \, x^i \mapsto \sum a_i\,\lambda_i$ (notice this is a finite sum). %
	
	Thus $V$ and $V^*$ are not isomorphic, since $\dim V$ is countable, but $\dim \F^\N$ is uncountable.
\end{enumerate}
\end{remarks}

\begin{definition}
	Let $V$ and $W$ be vector space over $\F$, and $\alpha$ a linear map $V\to W$, $\alpha\in \Lin(V,W)$. Then we define $\alpha^*:W^*\to V^* \in \Lin(W^*,V^*)$, by setting $\alpha^*(\theta) = \theta\alpha:V\to\F$. %
	
	(Note: $\alpha$ linear, $\theta$ linear implies $\theta\alpha$ linear, and so $\alpha^*(\theta)\in V^*$ as claimed, if $\theta\in W^*$.) %
\end{definition}

\begin{lemma}
	Let $V,W$ be finite dimensional vector spaces, with %
	\begin{itemize}
		\shortskip
		\item [] $v_1,\ldots,v_n$ a basis of $V$, and $w_1,\ldots,w_m$ a basis for $W$;
		\item [] $v_1^*,\ldots,v_n^*$ the dual basis of $V^*$, and $w_1^*,\ldots,w_m^*$ the dual basis for $W^*$; %
	\end{itemize}
	If $\alpha$ is a linear map $V\to W$, and $A$ is the matrix of $\alpha$ with respect to $v_i,w_j$, then $A^\Trans$ is the matrix of $\alpha^*:W^*\to V^*$ with respect to $w_j^*,v_i^*$.
\end{lemma}

\begin{proof}
	Write $\alpha^*(w_i^*) = \sum_{j=1}^n c_{ji} \, v_j^*$, so $c_{ij}$ is a matrix of $\alpha^*$. Apply this to $v_k$: %
	\begin{align*}
		\LHS
		&= \left( \alpha^* (w_i^*) \right)(v_k)
		 = w_i^*(\alpha(v_k))
		 = w_i^*\left( \textstyle\sum_\ell a_{\ell k} \,w_\ell \right) = a_{ik} \\
		\RHS
		&= c_{ki},
	\end{align*}
	that is, $c_{ji}=a_{ij}$ for all $i,j$.
\end{proof}
 This was the promised interpretation of $A^\Trans$. %


\begin{corollary}
\mbox{}
\begin{enumerate}
	\shortskip
	\item $(\alpha\beta)^* = \beta^* \alpha^*$;
	\item $(\alpha+\beta)^* = \alpha^* + \beta^*$;
	\item $\det \alpha^* = \det\alpha$
\end{enumerate}
\end{corollary}

\begin{proof}
	(i) and (ii) are immediate from the definition, or use the result $(AB)^\Trans = B^\Trans A^\Trans$. (iii) we proved in the section on determinants where we showed that $\det A^\Trans=\det A$. %
\end{proof}

Now observe that $\left.(A^\Trans)\right.^\Trans = A$. What does this mean?

\begin{proposition}
\mbox{}
\begin{enumerate}
	\item Consider the map $V\to V^{**} = (V^*)^*$ taking $v\mapsto \hat{\hat{v}}$, where $\hat{\hat{v}}(\theta) = \theta(v)$ if $\theta\in V^*$. Then $\hat{\hat{v}} \in V^{**}$, and the map $V\mapsto V^{**}$ is linear and injective. %
	\item Hence if $V$ is a finite dimensional vector space over $\F$, then this map is an isomorphism, so $V\isomto V^{**}$ canonically. %
\end{enumerate}
\end{proposition}

\begin{proof}
\mbox{}
\begin{enumerate}
	\item We first show $\hat{\hat{v}}\in V^{**}$, that is $\hat{\hat{v}}:V^*\to\F$, is linear: %
	\begin{align*}
		\hat{\hat{v}} (a_1\theta_1+a_2\theta_2)
		&= (a_1\theta_1+a_2\theta_2)(v) = a_1\,\theta_1(v) + a_2 \,\theta_2(v) \\
		&= a_1\,\hat{\hat{v}}(\theta_1) + a_2\,\hat{\hat{v}}(\theta_2).
	\end{align*}
	Next, the map $V\to V^{**}$ is linear. This is because
	\begin{align*}
		\left( \lambda_1 v_1 + \lambda_2 v_2 \right)^{\hat\hat}(\theta)
		&= \theta\left( \lambda_1 v_1 + \lambda_2 v_2 \right) \\
		&= \lambda_1\,\theta(v_1) + \lambda_2\,\theta(v_2) \\
		&= \left( \lambda_1 \hat{\hat{v_1}} + \lambda_2 \hat{\hat{v_2}} \right)(\theta).
	\end{align*}
	Finally, if $v\neq 0$, then there exists a linear function $\theta:V\to\F$ such that $\theta(v)\neq 0$. %
	
	(Proof: extend $v$ to a basis, and then define $\theta$ on this basis. We've only proved that this is okay when $V$ is finite dimensional, but it's always okay.) %
	
	Thus $\hat{\hat{v}}(\theta)\neq 0$, so $\hat{\hat{v}}\neq 0$, and $V\to V^{**}$ is injective. %
	\item Immediate. \qedhere
\end{enumerate}
\end{proof}

\begin{definition}
\mbox{}
\begin{enumerate}
	\item If $U\leq V$, then define
	\begin{equation*}
		U^\circ = \left\{\theta\in V^* \mid \theta(U)=0\right\}
		= \left\{\theta\in V^* \mid \theta(u) = 0 \;\forall u\in U\right\}
		\leq V^*.
	\end{equation*}
	This is the \emph{annihilator} of $U$, a subspace of $V^*$, often denoted $U^\perp$. %
	\item If $W\leq V^*$, then define
	\begin{equation*}
		{}^\circ W = \left\{v\in V \mid \varphi(v)=0\;\forall\varphi\in W\right\}\leq V.
	\end{equation*}
	This is often denoted $^\perp W$.
\end{enumerate}
\end{definition}

\begin{example}
	If $V=\R^3$, $U=\left\langle (1,2,1) \right\rangle$, then %
	\begin{equation*}
		U^\circ = \left\{\sum_{i=1}^3 a_i\,e_i^* \in V^* \mid a_1+2a_2+a_3 = 0 \right\}
		= \left\langle \mat{-2 \\ 1 \\ 0}, \mat{0 \\ 1 \\ -2} \right\rangle.
	\end{equation*}
\end{example}

\begin{remark}
	If $V$ is finite dimensional, and $W\leq V^*$, then under the canonical isomorphism $V\to V^{**}$, we have $^\circ W \mapsto W^\circ$, where $^\circ W\leq V$ and $(W^\circ)\leq (V^*)^*$. Proof is an exercise. %
\end{remark}

\begin{lemma}
	Let $V$ be a finite dimensional vector space with $U\leq V$. Then %
	\begin{equation*}
		\dim U+\dim U^\circ=\dim V.
	\end{equation*}
\end{lemma}

\begin{proof}
	Consider the restriction map $\Res:V^* \to U^*$ taking $\varphi\mapsto \varphi |_U$. (Note that $\Res=\iota^*$, where $\iota: U\injto V$ is the inclusion.) %
	
	Then $\ker\Res=U^\circ$, by definition, and $\Res$ is surjective (why?).
	
	So the rank-nullity theorem implies the result, as $\dim V^*=\dim V$.
\end{proof}

\begin{proposition}
	Let $V,W$ be a finite dimensional vector space over $\F$, with $\alpha\in \Lin(V,W)$. Then %
	\begin{enumerate}
		\shortskip
		\item $\ker(\alpha^*:W^*\to V^*)=\left( \Im\alpha \right)^\circ$ ($\leq W^*$);
		\item $\rk(\alpha^*) = \rk(\alpha)$; that is, $\rk A^\Trans = \rk A$, as promised;
		\item $\Im \alpha^* = \left( \ker\alpha \right)^\circ$.
	\end{enumerate}
\end{proposition}

\begin{proof}
\mbox{}
\begin{enumerate}
	\shortskip
	\item Let $\theta\in W^*$. Then $\theta\in\ker\alpha^* \iff \theta\alpha=0 \iff \theta\alpha(v)=0 \;\forall v\in V \iff \theta\in\left( \Im\alpha \right)^\circ$. %
	\item By rank-nullity, we have
	\begin{align*}
		\rk\alpha^*
		&= \dim W-\dim\ker\alpha^* \\
		&= \dim W-\dim(\Im \alpha)^\circ, \text{by (i)}, \\
		&= \dim \Im\alpha, \text{ by the previous lemma,} \\
		&= \rk\alpha, \text{ by definition.}
	\end{align*}
	\item Let $\varphi\in\Im\alpha^*$, and then $\varphi=\theta\circ\alpha$ for some $\theta\in W^*$. Now, let $v\in\ker\alpha$. Then $\varphi(v)=\theta\alpha(v)=0$, so $\varphi\in\left( \ker\alpha \right)^\circ$; that is, $\Im\alpha^* \subseteq\left( \ker\alpha \right)^\circ$. %
	
	But by (ii),
	\begin{align*}
		\dim\Im\alpha^* = \rk(\alpha^*)
		&= \rk\alpha = \dim V - \dim\ker\alpha \\
		&= \dim\left( \ker\alpha \right)^\circ
	\end{align*}
	by the previous lemma; that is, they both have the same dimension, so they are equal. \qedhere %
\end{enumerate}
\end{proof}

\begin{lemma}
	Let $U_1,U_2\leq V$, and $V$ be finite dimensional. Then %
	\begin{enumerate}
		\shortskip
		\item $U_1^{\circ\circ} \isomto {}^\circ\!(U_1^\circ) \isomto U_1$ under the isomorphism $V\isomto V^{**}$. %
		\item $(U_1+U_2)^\circ = U_1^\circ \cap U_2^\circ$.
		\item $(U_1\cap U_2)^\circ = U_1^\circ + U_2^\circ$.
	\end{enumerate}
\end{lemma}

\begin{proof*}
	Exercise!
\end{proof*}

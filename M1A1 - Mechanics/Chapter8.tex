%!TEX root = linear-algebra.tex
\stepcounter{lecture}
\setcounter{lecture}{8}


\sektion{Rigid Body Motion}
\begin{definition}
\emph{Rigid Body Motion} occurs when 
\[\frac{d|\vec{r}_- \vec{r}_j|}{dt} = 0,~ \forall i,j\]
For such a system
\[\vec{v}_i = \vec{v}_{cm} + \underbrace{\vec{\omega}\times\vec{R}_i}_{\vec{u}_i}\]	
Where $\vec{\omega}$ is the \emph{angular velocity of the rigid body}.

We can also write 
\[\vec{v}_i = \vec{V} + \vec{\omega}\times\vec{r}_i\]
where 
$\vec{V} = \vec{v}_{cm} - \vec{\omega}\times\vec{r}_{cm}$
\end{definition}

To determine the motion of the system we'll need to find $\vec{v}_{cm}$ and $\vec{\omega}$. For $\vec{v}_{cm}$ we already have this!
\begin{equation}M\frac{d\vec{v}_{cm}}{dt} = \vec{F}^{EXT}\end{equation}

What about $\vec{\omega}$?
\[\vec{J}_{cm} = \sum_i\vec{R}_i\times m_i\vec{u}_i\]

For a rigid body $\vec{u}_i = \vec{\omega}\times\vec{R}_i$
\[\vec{J}_{cm} = \sum_i\vec{R}_i \times m_i(\vec{\omega}\times\vec{R})i) = \sum_i m_i(\vec{R}_i\times(\vec{\omega}\times\vec{R}_i))\]

From the identity for the triple vector product, we have
\[\vec{J}_{cm} = \sum_im_i[R_i^2\vec{\omega} - (\vec{\omega}\cdot\vec{R}_i)\vec{R}_i]\]

Consider only planar motion: we have $\vec{\omega} = \omega\hat{k}$, and $\vec{R}_i = X_i\hat{i} + Y_i\hat{j}$. Thus 
\[\vec{\omega}\cdot\vec{R}_i = 0,~\forall i\]

As a result:
\begin{equation}
\vec{J}_{cm} = \underbrace{(\sum_im_iR_i^2)}_{I_{cm}}\vec{\omega}	
\end{equation}

\begin{definition}
$I_{cm}$ is the \emph{moment of inertia} about the centre of mass.	
\end{definition}

For this Rigid Body Motion $\dfrac{d|R_i|}{dt} = 0$. This means that $I_{cm}$ is constant.

Consider
\[\frac{d\vec{J}_{cm}}{dt} = I_{cm}\frac{d\vec{\omega}}{dt} = \vec{\tau}_{cm}^{EXT} = \sum_i\vec{R}_i\times\vec{F}_i^{EXT}\]

For a rigid body undergoing planar motion:
\begin{equation}M\frac{d\vec{v}_{cm}}{dt} = \vec{F}^{EXT}\end{equation}
\begin{equation}I_{cm}\frac{d{\omega}}{dt} = {\tau}_{cm}^{EXT}\end{equation}
 (Scalar Equation since all in $\hat{k}$)
 
\subsektion{Kinetic Energy}~
\[T = \frac{1}{2}M\vec{v}_{cm}^2 + \frac{1}{2}\sum_im_i\vec{u}_i^2\]
$\vec{u}_i = \vec{\omega} \times \vec{R}_i$, $u_i^ 2 = (\vec{\omega}\times\vec{R}_i) \cdot (\vec{\omega}\times\vec{R}_i)$

For planar motion $|\vec{\omega}\times\vec{R}_i| = |\vec{\omega}||\vec{R}_i| \implies u_i^2 = \omega^2R_i^2$
\[\begin{aligned}T &= \frac{1}{2}M\vec{v}_{cm}^2 + \frac{1}{2}\left(\sum_im_i{R}_i^2\right)\omega^2\\
\implies 	T &= \frac{1}{2}M\vec{v}_{cm}^2 + \frac{1}{2}I_{cm}\omega^2
\end{aligned}
\]
\lecturemarker{30}{22 Oct}
\begin{definition}
The continuous case:
\[M = \sum_i m_i = \int_B \,dm\]
\[\vec{r}_{cm} = \frac{\sum_i m_i\vec{r}_i}{M} = \frac{\int_B\vec{r}\,dm}{M}\]
\[I_{cm} = \sum_i m_iR_i^2 =  \int_B R^2\,dm\]
Equations of motion remain the same.
\end{definition}~

\begin{example}[Uniform Rod]
	\vspace*{50pt}
\end{example}



\subsektion{* Parallel Axis Theorem *}
\lecturemarker{31}{22 Oct}

(Non-examinable in 2015)
\begin{theorem}[Parallel Axis Theorem]
For an axis, $P$, parallel to the centre of mass	\[I_P = I_{CM} + Mr^2_{CM}\]
\end{theorem}
\begin{proof}
\[\begin{aligned}I_P = \sum_{i} m_ir_i^2 &= \sum_i m_i(\vec{r}_{CM} + \vec{R}_i)^2\\
&= \sum_i m_i r_{CM}^2 + 2	
\end{aligned}
\]
\end{proof}

\begin{example}[Physical Pendulum]
	Blah
\end{example}






  \begin{center}
  \textsf{\textbf{- End of Mechanics -}}	
  \end{center}
  
  




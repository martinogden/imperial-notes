%!TEX root = linear-algebra.tex
\stepcounter{lecture}
\setcounter{lecture}{2}
\sektion{Kinetics and Newtons Laws}
\subsektion{Newton's Laws}
\lecturemarker{9}{22 Oct}
\begin{definition}
\begin{itemize}

\item \emph{Mass, $m$} - ``Quantity of Matter'', measured in kg (scalar)

\item \emph{Momentum, $\vec{p} = m\vec{v}$} - ``Quantity of Motion'' (vector)

\item \emph{Inertia} - ``Vis Insita'' (innate force of matter). The resistance of an object to change its state of motion.

\item \emph{Force} - An action that changes an objects state of motion
\end{itemize}
\end{definition}

\begin{theorem}[Newton's First Law]
Every body has inertia. 	
\end{theorem}


\begin{theorem}[Newton's Second Law]
The net force on an object is equal to the rate of change of momentum:

\[ \vec{F} = \frac{d\vec{p}}{dt} = \frac{d(m\vec{v})}{dt}\]
\end{theorem}

\begin{theorem}[Newton's Third Law]
If $\vec{F}_{AB}$ is the force on object $A$ due to object $B$. Then $\vec{F}_{BA} = -\vec{F}_{AB}$.
\end{theorem}


%!TEX root = linear-algebra.tex
\stepcounter{lecture}
\setcounter{lecture}{5}


\sektion{Energy}

\lecturemarker{19}{22 Oct}

Energy gives us another viewpoint on mechanical systems. 

1D: From Newton's 2nd Law
\[m\ddot{x} = F(x,\dot{x},t) \implies m\ddot{x}\dot{x} = F\dot{x}\]
Since $\ddot{x}\dot{x} = \frac{d}{dt}\left(\frac{1}{2}\dot{x}^2\right)$ \begin{equation}\boxed{ \frac{d}{dt}\left(\frac{1}{2}m\dot{x}^2\right) = F\dot{x}}\end{equation}


Call $T = \frac{1}{2}m\dot{x}^2$ and integrate (5.1) with respect to time
\[\begin{aligned}\int_{t_1}^{t_2}\frac{\mathrm{d}}{\mathrm{d}t}T\,\mathrm{d}t &= \int_{t_1}^{t_2}F\dot{x}\,\mathrm{d}t\\ 
\implies T(t_2) - T(t_1) &= \int_{x(t_1)}^{x(t_2)}F\,\mathrm{d}x	
\end{aligned}
\]

\begin{definition}We call $T = \frac{1}{2}m\dot{x}^2$ the \emph{kinetic energy}, $F\dot{x}$ the \emph{rate of work}.	


$\displaystyle{W_{12} = \int_{x(t_1)}^{x(t_2)}F\,\mathrm{d}x	}$ is the \emph{work done} on $m$ by $F$.

Define $\displaystyle{V(x) = -\int F\,dx + C}$ is \emph{potential energy}. $T + V = E$, the \emph{total energy}. 

A force that can be written in terms of a potential ($\vec{F} = -\vec{\nabla}V$) is \emph{conservative}.
\end{definition}


\begin{theorem}[Conservation of Energy]
Under conservative forces, the total energy of a system is constant.
\end{theorem}
\begin{proof}

Suppose that $F = F(x)$, $V(x) = -\int F\,dx + C$ or $F = -\frac{dV}{dx}$

\[\begin{aligned}  \int_{x(t_1)}^{x(t_2)}F\,\mathrm{d}x	 &= \int_{x(t_1)}^{x(t_2)} - \frac{\mathrm{d}V}{\mathrm{d}x}\,\mathrm{d}x\\
~\\
\implies T(t_2) + V(t_2) &= T(t_1) + V(t_1) = E
\end{aligned}
\]
More generally, from (5.1)
\[\frac{d}{dt}\left(\frac{1}{2}m\dot{x}^2\right) - F\dot(x) = 0\]

Since $F\dot(x) = \dfrac{dV}{dx}\dfrac{dx}{dt} = \dfrac{dV}{dt}$
\[\frac{d}{dt}\left(\frac{1}{2}m\dot{x}^2 - V\right) = 0
\]

\[\implies T + V = E, \text{ a constant}\qedhere\]
\end{proof}


Not all forces are conservative!\\

\begin{example}
$F_D = -C_D\dot{x}$	is not conservative. 

Suppose that 
\vspace*{80pt}

Newton's Second Law:
\[m\ddot{x} = F_{CON} + F_D\]
\[\implies m\ddot{x} + \frac{dV}{dx} = -C_D\dot{x}\]
Multiplying by $\dot{x}$ and rearranging the terms:
\[\frac{d}{dt}(\underbrace{T + V}_{E}) = -C_D\dot{x}^2 \leq 0\]
\[\implies \frac{dE}{dt} \leq 0 \implies \text{ Energy decreases with time}\]
\end{example}

Examples of Conservative Forces
\begin{examples}
\begin{itemize}
\item Gravity: $F = -mg \implies V = mgx + C$	
\item Spring Force: $F = -kx \implies V = \frac{1}{2}kx^2 + C$
\end{itemize}
We can choose $C$ for our convenience. 
\end{examples}

\lecturemarker{20}{22 Oct}
Recall that forces that are related to a potential are called \emph{conservative forces}.

 Another way to think about conservative forces is through the \emph{work done}:
\[W_{12} = \int_{x(t_1)}^{x(t_2)}F\,\mathrm{d}x	\]
If the forces is conservative $F = -\dfrac{dV}{dx} \implies W_{12} = -V(x_2) + V(x_1)$. 

Hence the work done just depends on the initial and final position. It is path independent! We also saw that as a result:
\[T(t_1) + V(t_1) = T(t_2) + V(t_2) = E, \text{ the total energy}\]

\pagebreak

\subsektion{Potential Wells} 


Suppose we know $\dot{x}$ and $x$ at $t = 0$. With this, we can find
\[E = \frac{1}{2}m\dot{x}^2(0) + V(x(0))\]
And we know this for all times.~\\

\begin{definition} The points $x_0,x_1$ and $x_2$ are where $V = E$. These points are called \emph{turning points}.
	
\end{definition}
\subsubsektion{Oscillations between Turning Points}
At the turning points, for example $V(x_1) = E$, we know that $T(x_1) = 0 \implies \dot{x_1} = 0$. 

We know that if the particle is between $x_0$ and $x_1$, it will oscillate between these points forever! We say that this particle is \emph{trapped}!

Period of oscillation between $x_0$ and $x_1$:
\[E = \frac{1}{2}m\dot{x}^2 + V(x)\]
Solve for $\dot{x}$
\begin{equation}\frac{dx}{dt} = \dot{x} = \pm\left[\frac{2}{m}(E-V(x))\right]^{1/2}\end{equation}

We need to choose the correct root based on $\dot{x}$ at a particular point in time. Suppose we know going from $x_0$ to $x_1$, $\dot{x} >0$. 

We need to integrate (5.5) to find the time it takes to go from $x_0$ to $x_1$
\[\begin{aligned}\int_{x_0}^{x_1}\frac{dx}{\left[\frac{2}{m}(E-V(x))\right]^{1/2}} &= \int_{t_0}^{t_1} \,dt\\
&= T_{osc}/2	
\end{aligned}
 \]
 Thus
 \begin{equation}
 T_{osc} = 2\int_{x_0}^{x_1}\frac{dx}{\left[\frac{2}{m}(E-V(x))\right]^{1/2}}
\end{equation}
	
\begin{example}[Spring]

Spring: $V = \frac{1}{2}kx^2$
\vspace*{80pt}

Initially $x(0) = L$, $\dot{x}(0) = 0$
\[E = \frac{1}{2}m\dot{x}(0) + V(L) = \frac{1}{2}kL^2\]
Then 
\setlength{\jot}{10pt}
\[\begin{aligned}T_{osc} &= 2\int_{-L}^L \frac{dx}{[\frac{2}{m}(\frac{1}{2}kL^2 - \frac{1}{2}kx^2)]^{1/2}}\\
&= 2\sqrt{\frac{m}{k}} \int_{-L}^{L} \frac{dx}{[L^2-x^2]^{1/2}}\\
&= 2\sqrt{\frac{m}{k}} \int_{-L}^{L} \frac{dx}{L[1-(x/L)^2]^{1/2}}\\
&u = x/L\\
&= 2\sqrt{\frac{m}{k}} \int_{-1}^{1} \frac{du}{[1-u^2]^{1/2}}\\
&= 2\sqrt{\frac{m}{k}}  \arcsin u\bigg|_{-1}^1 =  2\pi \sqrt{\frac{m}{k}} 
\end{aligned}
\]

So $T_{osc} =  2\pi \sqrt{\dfrac{m}{k}},~ \omega_0 = \sqrt{\dfrac{k}{m}}$
	
\end{example}

\subsubsektion{Escape}
\lecturemarker{21}{22 Oct}
Suppose the particle is at $x_A$. What speed does is need to not be trapped, i.e. $x\to \infty$ as $t \to \infty$?

Initial speed: $u$
\[E = \frac{1}{2}mu^2 + V(x_A)\]

We want $E > E^*$ to allow our particle to escape. $E* = V(X_1)$. We require then 
\[V(X_1) < \frac{1}{2}mu^2 + V(x_A)\]
\[\implies u > \sqrt{\frac{2}{m}(V(X_1)-V(x_A))}\] 

\pagebreak


\subsektion{Stability} 

\begin{definition}
\emph{Equilibrium Points} are where $\dfrac{dV}{dx} = 0 \implies F = 0 \implies m\ddot{x} = 0$

We say that an equilibrium point is 
\begin{itemize}
\item  \emph{stable} if $\dfrac{d^2V}{dx^2} > 0$ (Minimum) e.g. $X_0$ 
\item \emph{unstable} if $\dfrac{d^2V}{dx^2} < 0$ (Maximum) e.g. $X_1$
 \end{itemize}

\end{definition}

\subsubsektion{Oscillations near Equilibrium Point}
Suppose we are near and very close to a stable equilibrium point, $X_0$, so $|x - X_0| << 1$. 

Taylor expansion of $V(x)$ about $X_0$:
\begin{equation}V(x) = V(X_0) + V'(X_0)(X-X_0) + \frac{1}{2}V''(X_0)(X-X_0)^2 + \dots	
\end{equation}
Since $X_0$ is an equilibrium point, we know $V'(X_0) = 0$
\[V(x) = V(X_0)+ \frac{1}{2}V''(X_0)(X-X_0)^2 \]
Since $X_0$ is a stable equilibrium point $V''(X_0) >0$
\[F = \frac{-dV}{dx} = -(x-X_0)V''(X_0)\]
From Newton's 2nd Law
\[m\ddot{x} = -(x-X_0)V''(X_0)\]
Taking $X = x-X_0$
\[m\ddot{X} + V''(X_0)X = 0\]
This looks like the simple harmonic oscillator with $k = V''(X_0)$.

 Since $\omega_0 = \sqrt{\dfrac{k}{m}}$, the frequency of small oscillation is $\omega_0 = \sqrt{\dfrac{V''(X_0)}{m}}$
\[\implies T_{osc} = \frac{2\pi}{\omega_0} = 2\pi \sqrt{\dfrac{m}{V''(X_0)}}\]~

\begin{example}[Lennard-Jones Potential]

Used to model interactions between neutral atoms or molecules and Molecular dynamics simulations.
\end{example}


\lecturemarker{22}{22 Oct}







%!TEX root = M1P1.tex
\stepcounter{lecture}
\setcounter{lecture}{2}

\pagebreak

\sektion{Series}~

\begin{definition}\lecturemarker{9}{5 Oct}
An (infinite) series is an expression \[\displaystyle{\sum_{n=1}^{\infty} a_n} \text{ or } a_1  + a_2 + \dots \] where $(a_i)_{i\geq 1}$ is a sequence.
\end{definition}

\subsektion{Convergence of Series}

\begin{definition}
We say that the series $\sum a_n = A \in \mathbb{R}$ (or ``converges to $A \in \RR$'') iff the sequence of partial sums $S_n:= \sum_{i=1}^n a_i \in \RR$ converges to $A \in \RR$; $S_n \to A$ as $n \to \infty$. 
\end{definition}



\[
\begin{tikzcd}[row sep=2cm]
 &  \mbox{Sequence of partial sums $(s_n)$}\arrow[<->]{dr}{s_n = \sum_{i=1}^n a_i} \\ 
\mbox{Sequence $(a_n)$} \arrow[<->]{ur}{a_n = s_n - s_{n-1}} \arrow[<->]{rr}{\text{Equivalent Information}} && \mbox{Series $\sum a_n$}
\end{tikzcd}
\]~\\



\begin{example}
$a_n = x^n,~n \geq 0$. Consider $\sum_{n=0}^{\infty} a_n = \sum_{n=0}^{\infty} x^n$.

Define $s_n = \sum_{i=0}^n x^i = 1 + x + \dots + x^n$ then $xS_n = x + \dots + x^n + x^{n+1} \implies S_n - xS_n = 1 - x^{n+1}$
\[\implies S_n = \begin{cases}
 \frac{1-x^{n+1}}{1-x} & x \neq 1\\
 n+1 & x = 1	
 \end{cases}
\]
So for $|x| < 1$, we see that 
\[S_n = \frac{1}{1-x} - \frac{x^{n+1}}{1-x} \to \frac{1}{1-x} \text{ as } n \to \infty\]

(Question Sheet 3: proves that $r^n \to 0$ if $|n| < 1$)
\end{example}~

So we have proved that $(s_n)$ is convergent and $\sum x^n = \frac{1}{1-x} \iR$ for $|x| < 1$. 

For $|x| \geq 1$, $a_n = x^n$ does not $\to 0$ as $n \to \infty$. So $\sum a_n = \sum x^n$ is \emph{not} a real number (does not converge) by the following result:\\


\begin{theorem}
$\sum_{n=0}^\infty a_n$ is convergent $\implies a_n \to 0$	
\end{theorem}

\begin{proof}
$S_n - S_{n-1} = a_n$. If $S_n \to S$ then $S_{n-1} \to S$ (Ex). So by the algebra of limits $a_n$ is convergent and $\lim_{n\to \infty} a_n = \lim_{n\to\infty} S_n - \lim_{n\to \infty} S_{n-1} = S - S = 0$.	
\end{proof}

\begin{proof}[Proof from first principles]
Fix $\epsilon >0$. $s_n \to s$, so 
\[\exists N\iN \text{ s.t. } \forall n \geq N,~|s_n - s| < \epsilon\]
\[\begin{aligned}
\implies |a_n| &= |s_n - s_{n-1}| \\
&\leq |s_n - s| + |s_{n-1} - s| \\ 
&< \epsilon + \epsilon \text{, for } n-1 \geq N.	
\end{aligned}
\]
So $\forall n \geq N+1$, $|a_n| < 2\epsilon$. 
\end{proof}\vspace*{5pt}

\begin{remark}
Converse is \emph{not} true. E.g. $a_n = \frac{1}{n} \to 0$, but $\sum \frac{1}{n}$ is \emph{not} convergent. 	
\end{remark}~

\begin{example}
$\displaystyle{\sum_{n=1}^{\infty} \frac{1}{n^2}}$ is convergent\footnote{Famously to $\pi^2/6$ - see \href{https://en.wikipedia.org/wiki/Basel_problem}{Basel Problem}.} 	

\begin{proof}(Trick) First do $\sum_{n=1}^\infty \frac{1}{n(n+1)}$ and use $\frac{1}{n(n+1)} = \frac{1}{n} - \frac{1}{n+1}$

\[\begin{aligned}
S_n &= \sum_{i=1}^n \frac{1}{i} - \frac{1}{i+1} \\
&= \textstyle{(1-\frac{1}{2}) + (\frac{1}{2} - \frac{1}{3}) + \dots + (\frac{1}{n} - \frac{1}{n+1})}\\
&= 1 - \frac{1}{n+1} \to 1 \text{ as } n \to \infty	
\end{aligned}
\]

$\implies \sum_{n=1}^\infty \frac{1}{n(n+1)}$ is convergent to $1$.

Ao now compare the partial sums $\sigma_n$ of $\sum \frac{1}{n^2}$ to those of $\sum \frac{1}{n(n+1)} = 1$

\[\begin{aligned}
\sigma_n = \sum_{i=1}^n \frac{1}{i^2} &= 1 + \sum_{j=1}^{n-1} \frac{1}{(j+1)^2} \\
&\leq 1 + \sum_{j=1}^{n-1}\frac{1}{j(j+1)}\\
&= 1 + s_{n-1}	
\end{aligned}
\]

$s_{n-1}$ is a bounded (by $1$) monotonically increasing sequence (because $\frac{1}{n(n+1)} >0$), convergent to $1$. So $s_{n-1} < 1~\forall n \implies \sigma_n < 2 \implies$ bounded above monotonic increasing sequence $\implies \sigma_n$ is convergent $\implies \sum \frac{1}{n^2}$ is convergent.	
\end{proof}
\end{example}~


Similarly $\sum \frac{1}{n^k}$ is convergent for $k \geq 2$ because $\frac{1}{n^k} \leq \frac{1}{n^2}$. In fact $\zeta(k) = \sum \frac{1}{n^k}$ is convergent for $k \in (1,\infty)$... See later!\\


\begin{theorem}[Algebra of Limits for Sequences]

If $\sum a_n = A \iR$ and $\sum b_n = B \iR$, then $\sum (\lambda a_n + \mu b_n) = \lambda A + \mu B \iR$. 
\end{theorem}

Put differently, if $\sum a_n$, $\sum b_n$ converge, then so does $\sum (\lambda a_n + \mu b_n)$ and it equals $\lambda \sum a_n + \mu \sum b_n$. 


\begin{proof}
Partial sums (to $n$ terms) of $\sum (\lambda a_n + \mu b_n)$ is 
\[\sum_{i=1}^{n} (\lambda a_i + \mu b_i) = \lambda \sum_{i=1}^{n} \lambda a_i + \sum_{i=1}^{n} \mu b_i \to \lambda \sum_{i=1}^{\infty} a_n + \mu \sum_{i=1}^{\infty} \mu b_n \]
as $n \to \infty$ by the algebra of limits for sequences. So the partial sums converge.
\end{proof}~



\lecturemarker{10}{5 Oct}
\begin{theorem}[Comparison Test]
If $0 \leq a_n \leq b_n$ and $\sum_{n=1}^{\infty} b_n$ converges, then $\sum_{n=1}^{\infty} a_n$ converges.	 (and $0 \leq \sum_{n=1}^{\infty}a_n \leq \sum_{n=1}^{\infty}b_n$)
\end{theorem}

\begin{proof}
Call the partial sums $A_n$, $B_n$ respectively. Then
\[0 \leq A_n \leq B_n \leq \sum_{i=1}^{\infty} b_i = \lim_{n\to \infty}B_n\]	

So $A_n$ is bounded and monotonically increasing $\implies$ convergent.
 
(Question Sheet 3 shows that if $A_n \leq B_n$ and $A_n \to A$, $B_n \to B$, then $A \leq B$)
\end{proof}~


\begin{proposition}
Suppose $a_n \geq 0 ~\forall n$. Then $\sum_{n=1}^{\infty} a_n$ converges iff $S_N = \sum_{n=1}^{N}a_N$ is bounded above and $\sum_{n=1}^{\infty} a_n$ diverges to $\infty$ (i.e. $S_n \to +\infty$ as $N \to \infty$) iff $S_N = \sum_{n=1}^{N} a_n$ is an unbounded sequence. 
\end{proposition}

\begin{proof}
$a_n \geq 0 \iff (S_n)$ is monotonic increasing. So $(S_n)$ bounded $\iff$ convergent.

$S_N$ unbounded $\iff \forall R>0,~ \exists N \in \NN$ such that $\forall n \geq N,~ S_n > R \iff S_n \to +\infty$. 
\end{proof}~

Ex: (Converse of Comparison Test)
If $0 \leq a_n \leq b_n$ then $\sum a_n$ diverges to $\infty\implies \sum b_n$ diverges to $\infty$\\


\begin{example}
$\sum_{n=1}^{\infty} \frac{1}{n^{\alpha}},~ \alpha > 1$ is convergent.
\begin{proof}
(Trick!) Arrange the partial sum as follows:
\[\begin{aligned}
1 + \frac{1}{2^\alpha} + \frac{1}{3^\alpha} + \dots  = 1 + \left(\frac{1}{2^\alpha} + \frac{1}{3^\alpha}\right) &+ \left(\frac{1}{4^\alpha} + \dots +\frac{1}{7^\alpha}\right)  \\ 
&+ \left(\frac{1}{8^\alpha} + \dots + \frac{1}{15^\alpha}\right) \\
&+ \left(\frac{1}{16^\alpha} + \dots + \frac{1}{31^\alpha}\right) \\
&+ \dots  \end{aligned}\]
Note that the $k$th bracketed term:
\[\left(\frac{1}{(2^k)^\alpha} + \dots +\frac{1}{(2^{k+1}-1)^\alpha}\right ) \leq \frac{1}{2^{k\alpha}} + \dots + \frac{1}{2^{k\alpha}} = \frac{2^k}{2^{k\alpha}} = \frac{1}{2^{k(\alpha-1)}}\]

So any partial sum is less than some finite sum of these bracketed terms, i.e. for some sufficiently large $N$: \[ S_N < \sum_{k=0}^{N} \frac{1}{2^{k(\alpha -1)}} = \frac{1-\frac{1}{2^{(N+1)(\alpha -1)}}}{1-\frac{1}{2^{(\alpha-1)}}} \leq \frac{1}{1-\frac{1}{2^{\alpha-1}}}\] because $\alpha >1$, so $\left|\frac{1}{2^{\alpha-1}}\right| < 1$, so denominator $>0$. 

So partial sums are bounded above $\implies$ convergent. 
\end{proof}	
\end{example}~

\begin{definition}
Say that the series $\sum_{n=1}^{\infty}a_n$ is \emph{absolutely convergent} if and only if the series $\sum_{n=1}^{\infty} |a_n|$ is convergent	
\end{definition}~

\begin{example}
$\sum_{n=1}^{\infty} \frac{(-1)^{n+1}}{n}$ is \emph{not} absolutely convergent, but it is convergent.


\textit{Rough Working.} $1 - \frac{1}{2} + \frac{1}{3} - \frac{1}{4} + \frac{1}{5} - \frac{1}{6} + \dots = (1-\frac{1}{2}) + (\frac{1}{3} - \frac{1}{4}) + (\frac{1}{5} -\frac{1}{6}) + \dots$, the $k$th bracket $\frac{1}{2k-1} - \frac{1}{2k} = \frac{1}{2k(2k-1)}$. This is positive and $\leq \frac{1}{2k(2k-2)} = \frac{1/4}{k(k-1)}$, seen earlier sum of these is convergent.

So cancellation between consecutive terms is enough to make series converge by comparison with $\sum \frac{1}{k(k-1)}$.

\begin{proof}
Fix $\epsilon >0.$ Then use 2 things\begin{enumerate}
\item[(1)] $\sum \frac{1}{2k(2k-1)}$	is convergent
\item[(2)] $\frac{(-1)^{n+1}}{n}\to 0$
\end{enumerate}
By (1) $\exists N_1$ such that $\forall n \geq N_1,~ \sum_{n=1}^{\infty} \frac{1}{k(k-1)} < \epsilon$

By (2) $\exists N_2$ such that $\forall n \geq N_2,~ \left|\frac{(-1)^{n+1}}{n}\right| < \epsilon$

Set $N =$ max$(N_1,N_2)$. Then $\forall n \geq N$, we have:
\[S_n = \left(1-\frac{1}{2}\right) + \left(\frac{1}{3} - \frac{1}{4}\right) + \dots \left(\frac{1}{2j-1} - \frac{1}{2j} \right) + \delta = \sum_{k=1}^{j} \frac{1}{2k(2k-1)} + \delta\] 
where $\delta = \begin{cases}
 	\frac{(-1)^{n+1}}{n} & \text{ if } n \text{ is odd.}\\
 	0 & \text{ if } n \text{ is even.}
 \end{cases}
$ $~~\left(j = \lfloor \frac{n}{2} \rfloor\right)$   $j = \begin{cases}
 	\frac{n-1}{2} & \text{ if } n \text{ is odd.}\\
 	\frac{n}{2} & \text{ if } n \text{ is even.}
 \end{cases}
$
\[\implies S_n = \sum_{k=1}^{\infty} \frac{1}{2k(2k-1)} - \sum_{k=\lfloor \frac{n}{2} \rfloor + 1}^{\infty} \frac{1}{2k(2k-1)} + \delta\]
\[\text{So } \left|S_n - \sum_{k=1}^{\infty} \frac{1}{2k(2k-1)} \right| \leq \sum_{k=\lfloor \frac{n}{2} \rfloor + 1}^{\infty} \frac{1}{2k(2k-1)} + \frac{1}{n} < \epsilon + \epsilon\] for all $n \geq 2N$ (so that $\lfloor \frac{n}{2} \rfloor + 1 >N$) 
\end{proof}
\end{example}\vspace*{10pt}

\lecturemarker{11}{5 Oct}
\begin{theorem}
	If $(a_n)$ is absolutely convergent, then it is convergent.
\end{theorem}

\begin{proof}
Let $S_n = \sum_{i=1}^{n} |a_i|$, $\sigma+n = \sum_{i=1}^n a_i$ be the partial sums.

We're assuming that $S_n$ converges. Therefore $S_n$ is Cauchy: \[ \forall \epsilon >0~ \exists N_{\epsilon}\text{ such that }n > m \geq N_{\epsilon} \implies |S_n - S_m| < \epsilon \iff |a_{m+1} + \dots + |a_n| < \epsilon\]

i.e. the terms in the tail of the series contribute little to the sum 

$\implies |a_{m+1} + \dots + a_n| < \epsilon$ by the triangle inequality $\implies |\sigma_n - \sigma_m| < \epsilon \implies (\sigma_n)$ is Cauchy $\implies \sum a_i$ is convergent.
\end{proof}~

\begin{example}
$\sum_{n=1}^{\infty} z_n$ is convergent for $|z| < 1$, divergent for $|z| \geq 1$
\begin{proof}
$\sum_{n=1}^{\infty} z_n$ is absolutely convergent because we showed that $\sum_{n=1}^{\infty} |z|^n$ converges to $\frac{1}{1 - |z|}$ for $|z| < 1$	

For $|z| \geq 1$, the individual terms $z^n$ have $|z^n| \geq 1$, so $z^n \not\to 0$, so $\sum z^n$ divergent.
\end{proof}
\end{example}


\subsektion{$*$Re-arrangement of Series$*$}
\emph{This section was non-examinable in 2015}

\textbf{Beware.} Do not rearrange series and sum them in a different order unless you can prove the result is the same.

\begin{example}
$\sum (-1)^{n+1} = 1 - 1 + 1 - 1 + \dots$

either this $``='' (1-1) + (1-1) + \dots = 0$

or this $``='' 1 - (1-1) + (1-1) + \dots = 1$	
\end{example}

A better (convergent) example\\

\begin{example}
$a_n:= 1 - \frac{1}{2} + \frac{1}{3} - \frac{1}{4} + \dots 	= \log 2$

(See later for proof of result, it's the series for $\log(1+x) = x - \frac{x^2}{2} + \dots$ putting $n=1$, which is on our radius of convergence!)

Reorder the sum as follows:
\[\begin{array}{cccccccc}
	1 & ~ & +\frac{1}{3} & ~ & +\frac{1}{5} & ~ & +\frac{1}{7} & \dots \\
	& -\frac{1}{2} & ~ & -\frac{1}{4} & ~ & -\frac{1}{6} & ~ & \dots \\
	&&&&&&&\\
= 	1 & ~ & +\frac{1}{3} & ~ & +\frac{1}{5} & ~ & +\frac{1}{7} & \dots \\
	-\frac{1}{2} [& 1 & ~ & +\frac{1}{2} & ~ & +\frac{1}{3} & ~ & \dots ]\\

\end{array}\]
Terms with even denominator appear only in bottom row ($\times -\frac{1}{2}$)

Terms with odd denominator appear in the top row ($\times 1$) + bottom row $\times -\frac{1}{2} \implies (\times \frac{1}{2})$ in total. 

So $a = \frac{1}{2}[1-\frac{1}{2} + \frac{1}{3} - \frac{1}{4} + \frac{1}{5} -\frac{1}{6} + \dots] \implies a = a/2,~\cont$ (But clearly $a \geq \frac{1}{2} > 0$)

\end{example}

This happened because when I reordered I went along the bottom row twice as fast as I went along the top row. Since the top and bottom row diverges to $\infty$, I'm computing $\infty - \infty$, and originally I did this like $(a+n) - n$ as $n \to \infty$. Now I'm doing it like $(a + n) - (n + \frac{a}{2})$ as $n \to \infty$. 


In fact I can rearrange the sum to converge to anything I like.\\

\begin{example}
Rearrange $a_n = \dfrac{(-1)^{n+1}}{n} \to 42$. 

We reorder the sum as follows
\begin{enumerate}
\item Take only off terms $a_{2n+1} > 0$ until their sum is $>42$. We can do this as $1 + \frac{1}{3} + \dots$ diverges to $\infty$!
\item Now take only even terms $a_{2n} < 0$ until sum gets $<42$
\item Repeat (i) and (ii) to fade.
\end{enumerate}


We can do each step because $\sum a_{2n+1}$ diverges to $\infty$ and $\sum a_{2n}\to -\infty$. We use all the terms eventually (so this is really a reordering of the whole sum)

Why? If not then we must eventually only take terms of one type (w.l.o.g. the even -ve terms) but these sum to $-\infty,~\cont$. At point they reach $<42$ we switch back to odd +ve terms.

Finally proof that the reordered sum converges to $42$ 
\[a_n \to 0 \text{ so }\forall \epsilon >0,~\exists N \iN \text{ s.t. } n \geq N \implies |a_n| < \epsilon ~(*)\]

So now we go to a point in the reordering where we have used all $a_i$ up to $N$ and then further to the point where the partial sum crosses $42$. At this point, $(*)$ holds, so I'm within $\epsilon$ of $42$. from this point on the sum is always within $\epsilon$ of $42$ by design and by $(*)$. 

\[\implies |s_n - 42| < \epsilon \text{ from this point on}\qed\]
\end{example}


More generally \lecturemarker{12}{9 Feb}
 if $(a_n)$ is a sequence whose terms tend to zero, $a_n \to 0$ and such that: \begin{itemize}
 \item $\displaystyle{\sum_{\substack{n \text{ s.t.} \\a_n \geq 0}} a_n}$ diverges ($\to \infty$)
 \item $\displaystyle{\sum_{\substack{n \text{ s.t.}\\ a_n < 0}}  a_n}$ diverges ($\to -\infty$)	
 \end{itemize}
 then I can rearrange the series $\sum a_n$ (1) to make it converge to \emph{any} number I like $\iR$ or (2) to  make it diverge to $\infty$ or (3) to $-\infty$. 
 
 For (1), the Algorithm is same as for $\sum \frac{(-1)^n}{n}$	
 \begin{enumerate}
 \item Pick +ve terms until partial sums are $>$ my fixed real number, $a$
 \item Now pick -ve terms until partial sum is $< a$
 \item Go back to (i) and repeat.
 \end{enumerate}
 
 If however $a_n \to 0$ and 
 \begin{itemize}
 	\item $\displaystyle{\sum_{\substack{n \text{ s.t.} \\a_n \geq 0}} a_n} \to \infty$
 \item $\displaystyle{\sum_{\substack{n \text{ s.t.}\\ a_n < 0}}  a_n}$ converges
 \end{itemize}
 Then however I rearrange $\sum a_n$ it will always diverge to $+ \infty$
 
 Similarly if $a_n \to 0$ and 
 
  \begin{itemize}
 	\item $\displaystyle{\sum_{\substack{n \text{ s.t.} \\a_n \geq 0}} a_n}$ converges
 \item $\displaystyle{\sum_{\substack{n \text{ s.t.}\\ a_n < 0}}  a_n} \to -\infty$
 $\implies  \sum a_n$ diverges to $-\infty$ (however rearranged)
 \end{itemize}
 
\textbf{ Final case:} $a_n \to 0$ and 
 \begin{itemize}
 	\item $\displaystyle{\sum_{\substack{n \text{ s.t.} \\a_n \geq 0}} a_n}$ converges
 \item $\displaystyle{\sum_{\substack{n \text{ s.t.}\\ a_n < 0}}  a_n}$ converges
 \end{itemize}
This is the \emph{good case} where \emph{however} you rearrange, $\sum a_n$ is \emph{absolutely convergent} to the same limit, $\sum_{a_n \geq 0} a_n + \sum_{a_n < 0} a_n$. 
We will prove this next time.\\

\begin{remark} Rearrange partial sums only. $a+ b = b+a$ is fine. Infinite sums are tricky!	
\end{remark}


\begin{definition}[Rearrangement of a Sequence] 
	If $M: \NN \to \NN$ is a bijection (i.e. a reordering!) then define $b_m:= a_{M(m)}.$ Then $(b_m)_{m \geq 1}$ is a rearrangement of $(a_n)$.
\end{definition}

e.g. if $M(1), M(2), M(3), M(4),\dots$ is $5,1,6,2,\dots$ then $b_1,b_2,b_3,b_4,\dots$ is $a_5,a_1,a_6,a_2,\dots$.\\


\begin{theorem}
Suppose that $\sum a_n$ is absolutely convergent. Then
\begin{itemize}
\item[(1)] $\sum_{a_n \geq 0} a_n$ is convergent to $A$ (say)	
\item[(2)] $\sum_{a_n < 0} a_n$ is convergent to $B$ (say)	
\item[(3)] $\sum a_n = A+B$
\item[(4)] $\sum b_m = A+B$ where $(b_m)$ is any rearrangement of $(a_n)$
\end{itemize}
\end{theorem}

\begin{proof} \lecturemarker{13}{9 Feb}
Key Idea: $\sum |a_n|$ is convergent so has a small ``tail'', so by the triangle inequality $\sum a_n$ has an even smaller tail so should converge. 

But what to? No idea, so we use the Cauchy criterion! 

(1) $s_n = \sum_{i=1}^n a_i,~\sigma_n = \sum_{i=1}^n |a_i|$. $\sigma_n$ convergent $\implies \sigma_n$ is Cauchy. 
\[\forall \epsilon > 0~\exists N \iN \text{ s.t. } \forall n,m \geq N,~|\sigma_n - \sigma_m | < \epsilon\]
w.l.o.g. $n \geq m$, this says 
\[\sum_{i=m+1}^n |a_i| < \epsilon \implies \left|\sum_{i=m+1}^m a_i \right| < \epsilon \iff |s_n - s_m| < \epsilon\]

So $(s_n)$ is Cauchy $\implies s_n$ is convergent. 

(2) $\sum_{a_n \geq 0} a_n$ is also convergent because the partial sums are monotonic increasing, bounded above by $\sum |a_n|$. Similarly $\sum_{a_n < 0} a_n$ is decreasing, $\geq -\sum |a_n|$, so also cvgt. 

(3) Let $A = \sum_{a_n \geq 0} a_n$ and $B = \sum_{a_n < 0} a_n$. Then $\forall \epsilon >0$
\[\exists N_1 \text{ s.t. } n \geq N_1 \implies \left| \sum_{a_n \geq 0}^{\text{first } n \text{ terms}} - A\right| < \epsilon \]
\[\exists N_2 \text{ s.t. } n \geq N_2 \implies \left| \sum_{a_n < 0}^{\text{first } n \text{ terms}} - B\right| < \epsilon \]

Let $N$ be max$(I,J)$ where $I$ is the $N_i$th $a_i \geq 0$ (the $N_i$th positive term) and $a_J$ the $N_J$th -ve term. Then $\forall n \geq N$

\[\begin{aligned}\left|\sum_{i=1}^n - (A+B)\right| 
&\leq \left|\sum_{a_i \geq 0}^n a_i - A\right| + \left|\sum_{a_i < 0}^n a_i - B\right| < \epsilon + \epsilon = 2\epsilon 
\end{aligned}
\]
So $\sum_{i=1}^n \to A+B$ as $n \to \infty$.\\

(4) Finally $(b_m)$ is a rearrangement of $(a_n)$. We want to show that $\sum b_m$ converges to $A+B$ as well. 

Pick $M \iN$ such that $b_1,b_2,\dots,b_M$ contains all of $P_1,P_2,\dots,P_I$ and $N_1,N_2,\dots,N_J$ where $P_i$ is the $i$th $a_i \geq 0$ and $N_J$ is the $j$th $a_j <0$. 

[i.e. we're far enough down the rearranged series to have included all significant $a_i \geq 0$ and $a_i <0$ which sum to $<\epsilon$ by (1) and (2)]

Then $\forall m \geq M$ we have

\[\begin{aligned}
\left|\sum_{i=1}^m b_i - (A+B)\right| 
&\leq \left|\sum_{b_i \geq 0}^m b_i - A\right| + \left|\sum_{b_i < 0}^m b_i - B\right|\\
&\leq \left|\sum_{a_k \geq 0}^I a_k + \delta - A\right| + \left|\sum_{a_k < 0}^J a_k +\delta' - B\right|\\
&< \epsilon + \epsilon = 2\epsilon 
\end{aligned}
\]
(where $\delta =$ sum of $a_k \geq 0$ with $k >I$ and $\delta' =$ sum of $a_k < 0$ with $k > J$) \end{proof}

\subsektion{Tests for convergence}

\setcounter{equation}{4}
We already met the first test:
\begin{theorem}[Comparison I]
If $0 \leq a_n \leq b_n$ and $\sum_{n=1}^{\infty} b_n$ converges, then $\sum_{n=1}^{\infty} a_n$ converges.	 (and $0 \leq \sum_{n=1}^{\infty}a_n \leq \sum_{n=1}^{\infty}b_n$)
\end{theorem}

Recall proof from earlier: $s_n = \sum a_i$ is monotonic increasing and bounded above by $\sum b_i \in \RR$.\\

\setcounter{equation}{17}
\begin{theorem}[Comparison II - Sandwich Test]
	Suppose $c_m \leq a_n \leq b_n$ and $\sum c_n,~\sum b_n$ are both convergent. Then $\sum a_n$ is convergent.
\end{theorem}
\begin{proof}
Use Cauchy. $\forall \epsilon >0,~ \exists N \in \NN$ such that $\forall n,m > N$
\[\left|\sum_{i=m+1}^n b_i\right| < \epsilon,~\left|\sum_{i=m+1}^n c_i\right| < \epsilon\] since the partial sums of $b_i,~c_i$ are Cauchy. Therefore
\[-\epsilon <\sum_{i=m+1}^n c_i \leq \sum_{i=m+1}^n a_i \leq \sum_{i=m+1}^n b_i < \epsilon   \]
\[\implies \left|\sum_{i=1}^{n} a_i - \sum_{i=1}^m a_i\right| < \epsilon \implies \left(\sum_{i=1}^{n} a_i \right) \text{ is Cauchy.} \qedhere\]
\end{proof}


\begin{theorem}[Comparison III]
 \lecturemarker{14}{12 Feb}
If $\frac{a_n}{b_n}\to l \in \RR$ then $\sum b_n$ absolutely convergent $\implies \sum a_n$ is absolutely convergent.
\end{theorem}
\begin{proof}
Pick $\epsilon = 1$, then $\exists N \in \NN$ such that $\forall n \geq N$:
\[\left|\frac{a_n}{b_n} - l \right| < 1 \implies \left|\frac{a_n}{b_n}\right| < |l| + 1 \implies |a_n| < (|l| + 1)|b_n|\]
So now by the comparison test $\sum_{n \geq N} |b_n|$ convergent $\implies \sum_{n \geq N} |a_n|$ convergent $\implies \sum_{n\geq 1} |a_n|$ convergent. 	
\end{proof}

We have used the obvious fact that if $\sum_{n \geq N} c_n$ is convergent then $\sum_{n \geq 1} c_n$ is also convergent (and vice-versa). Ex: proof this!\\

\begin{theorem}[Alternating Series Test.]
Given an alternating sequence $a_n$ where $a_{2n} \geq 0$, $a_{2n+1} \leq 0~ \forall n$. Then $|a_n|$ monotonic decreasing to $0 \implies \sum a_n$ convergent
\end{theorem}

\begin{proof}
Write $a_n = (-1)^nb_n,~b_n\geq 0 ~\forall n	$. Consider the partial sums $S_n = \sum_{i=1}^{n} (-1)^nb_n$.\\

  Observe that: \begin{enumerate}
 \item[(1)]$S_i \leq S_{2n}~\forall i \geq 2n$
 \item[(2)]$S_i \geq S_{2n+1}~\forall i\geq 2n+1$
 \end{enumerate}
 Since if $i=2j$ is even, then
  \[\begin{aligned}
	S_{2j} &= S_{2n} + a_{2n+1} + \dots + a_{2j}\\ 
	&= S_{2n} + \underbrace{(-b_{2n+1} + b_{2n+2})}_{\leq 0} + \dots + \underbrace{(-b_{2j-1} + b_{2j})}_{\leq 0} \leq S_2n
\end{aligned}
\]
 
  If $i= 2j+1$ is odd, then similarly:
   \[\begin{aligned}
	S_{2j} = S_{2n} + \underbrace{(-b_{2n+1} + b_{2n+2})}_{\leq 0} + \dots + \underbrace{(-b_{2j-1} + b_{2j})}_{\leq 0} - b_{2j+1} \leq S_2n
\end{aligned}
\]
  
 So now $\forall \epsilon >0,~ \exists N \in \mathbb{N}$ such that $\forall n \geq N,~|b_n| < \epsilon$. So $\forall n,m\geq 2n$, we have: \[S_{2N+1} \leq S_n,~S_m \leq S_{2N}\] 
 \[\begin{aligned}\text{So } |S_n - S_m| &\leq |S_{2N+1} - S_{2N}|\\
 &= b_{2n+1} < \epsilon	
\end{aligned}
\]
\end{proof}~

\begin{theorem}[Ratio Test]
If $a_n$ is a sequence such that $\left|\frac{a_{n+1}}{a_n}\right| \to r < 1$, then $\sum a_n$ is absolutely convergent.	
\end{theorem}
\begin{proof}
Fix $\epsilon = \frac{1-r}{2} > 0$. Then $\exists N \in \mathbb{N}$ such that $\forall n \geq N$
\[\left|\frac{a_{n+1}}{a_n} - r\right| < \epsilon \implies |a_{n+1}| < (r + \epsilon)|a_n|\]
Set $\alpha := r + \epsilon = \frac{1 + r}{2} < 1$. 

Inductively
\[|a_{N+m}| < \alpha|a_{N+m-1}| < \dots < \alpha^m|a_N|\]	
So $\forall k \geq N$ \[|a_k| <  \alpha^{k-N}|a_N| = C\alpha^k\]

Then \[C\sum_{k=N}^{n} \alpha^k = \frac{C(\alpha^N-\alpha^n)}{1-\alpha} \to \frac{C'}{1-\alpha} \text{ as } n \to \infty \text{, since } \alpha < 1\]

So by the comparison test $\sum_{k\geq N} |a_k|$ is convergent $\implies \sum_{k\geq 1} |a_k|$ is convergent
\end{proof}

The point is that the ratio test, when it applies, says that $a_n \approx r^n$ i.e. decays exponentially. But many convergent series like $\sum \frac{1}{n^2}$ do not decay so fast.\\

\begin{example}
$a_n = \dfrac{100^n(\cos n\theta + i\sin n\theta}{n!} = \dfrac{(100e^{i\theta})^n}{n!}$

Then 
\[\left|\frac{a_{n+1}}{a_n}\right| = \frac{(100e^{i\theta})^{n+1}/(n+1)!}{(100e^{i\theta})^{n}/n!} = \frac{100}{n+1} \to 0\]
So by the ratio test, $\sum a_n$ is absolutely convergent $\implies \sum a_n$ is convergent.	
\end{example}



\begin{theorem}[Root Test] \lecturemarker{15}{16 Feb}
If $\lim_{n\to \infty} |a_n|^{1/n} = r < 1$, then $\sum a_n$ is absolutely convergent.	
\end{theorem}
\begin{proof}
Fix $\epsilon = \frac{1-r}{2} > 0$. Then $\exists N \in \mathbb{N}$ such that $\forall n \geq N$
\[\left||a_n|^{1/n} - r\right| < \epsilon \implies |a_n|^{1/n} < r + \epsilon\]
Set $\alpha := r + \epsilon = \frac{1 + r}{2} < 1$, so that $|a_n| < \alpha^n$. Then
\[ \sum_{k=1}^n\alpha^k = \frac{\alpha(1-\alpha^n)}{1-\alpha}  \to \frac{\alpha}{1-\alpha} \text{ as } n \to \infty \text{ since } \alpha < 1\]

So by the comparison test $\sum_{k\geq 1} |a_k|$ is convergent.
\end{proof}


\subsektion{Power Series}\vspace*{5pt}
\begin{theorem}[Radius of Convergence]
	Consider the series $\sum a_nz^n ~(*),~ z,a_n \iC$. 
	
	Then $\exists R \in [0,\infty]$ such that $|z| < R \implies (*)$ is aboslutely convergent, $|z| > R \implies (*)$ divergent
\end{theorem}

\begin{proof}
Define $R =$ sup $S = \{|z| : a_nz^n \to 0\}$	 or $R = \infty$ if the set is unbounded. (1) Suppose $|z| < R$. $|z|$ not an upperbound for $S \implies \exists w$ such that $|w| > |z|$ and $a_nw^n \to 0.$ Then \[|a_nz^n| = |a_nw^n|\left|\frac{z}{w}\right|^n \leq A\left|\frac{z}{w}\right|^n\] Since $\left|\frac{z}{w}\right| < 1 \implies \sum|a_nz^n|$ cvgt. Similarly $|z| >R  \implies$ $\sum |a_nz^n|$ divergent. 

(2) Suppose $|z| > R$. Then $a_nz^n \not\to 0$ as $n \to \infty \implies \sum a_nz^n$ does not converge.
\end{proof}

\begin{clicker}
What is the radius of convergence for $\sum \frac{z^n}{n}$? 

\textbf{Answer:} $R = 1$, in fact the series
\begin{enumerate}
\item $\sum z^n$
\item $\sum \frac{z^n}{n}$
\item $\sum \frac{z^n}{n^2}$	
\end{enumerate}
all have this $R$.

\begin{proof}
The ratio test gives  $\left|\dfrac{z^{n+1}}{z^n}\cdot f(n)\right|$ where $f$ is a rational function of $n$ of degree $0$. $= |zf(n)| \to |z|$ as $n \to \infty$. So convergent for $|z| < 1$ and divergent for $|z| > 1$.  	
\end{proof}

But notice different behaviours on $|z| = 1$. \begin{enumerate}
 \item Never converges on $|z| = 1$ as $z^n \not\to 0$
 \item Convergent for some $|z| = 1$ (in fact $z \neq 1$), divergent for others
 \item Also convergent $\forall z$ with $|z|  =1$ (comparison with $\sum \frac{1}{n^2}$)	
 \end{enumerate}
\end{clicker}






\subsubsektion{Products of Series}
Consider 
\[\begin{aligned}
\sum_{n=0}^{\infty} a_nz^n \sum_{n=0}^{\infty}b_nz^n &= (a_0 + a_1z + a_2z^2 + \dots) (b_0 + b_1z + b_2z^2 + \dots)\\
&``='' a_0b_0 + (a_0b_1 + a_1b_0)z + (a_0b_2 + a_1b_1 + a_2b_0)z^2 + \dots \\
&= \sum_{n=0}^{\infty} c_nz^n 	
\end{aligned}
\]
where $c_0 = a_0b_0$, $c_1 = a_0b_1 + a_1b+0, \dots c_n = a_0b_n + a_1b_{n-1} + \dots + a_nb_0$. 

So we set $c_n = \sum_{i=0}^{n} a_i b_{n-i}$ and ask when is the product $\sum a_nz^n \sum b_nz^n$ equal to $\sum c_nz^n$? We can also do this without the $z^n$'s:\\

\begin{definition}
Given series $\sum a_n,~\sum b_n$, their \emph{Cauchy Product} is the series $\sum c_n$ where $c_n := \sum_{i=0}^n a_ib_{n-i}$.	
\end{definition}


\begin{theorem}[Cauchy Product]  \lecturemarker{16}{19 Feb}
If $\sum a_n, \sum b_n$ are absolutely convergent, then $\sum c_n$ is absolutely convergent to $(\sum a_n) \cdot (\sum b_n)$	
\end{theorem}
\textit{Proof.}
See handout on blackboard. Non-examinable. \\

\begin{corollary}
If $\sum A_nz^n$ and $\sum B_nz^n$ have radius of convergence $R_A$ and $R_B$ respectively, then $\sum c_nz^n$ has radius of convergence $R_C \geq \mathrm{min}\{R_A,R_B\}$.	
\end{corollary}
\begin{proof}
By the previous theorem, for $|z| < \mathrm{min}\{R_A,R_B\} ~(*)$ we have $\sum A_nz^n$ and $\sum B_nz^n$ absolutely convergent $\implies \sum c_nz^n$ absolutely convergent to their product. 

In fact $|c_nz^n| \to 0$ so $|z| < R_c$. So by $(*)$, $R_c \geq \mathrm{min}\{R_A,R_B\}$.
\end{proof}~

\begin{example}
$\sum z^n$ has $R_A = 1$, $1-z$ has $R_B = \infty$ So their cauchy product $\sum c_nz^n$ has $R_c \geq 1$. 

\emph{Ex:} Check $c_0 = 1, c_n = 0~\forall n \geq 1$, so in fact $R_c = \infty$	.

But we only know that $\sum c_n z^n = 1 = (\sum z^n)(1-z)$ when $|z| < 1 = \mathrm{min}\{R_A,R_B\}$. 
\end{example}~

\subsubsektion{Exponential Power Series}\vspace*{5pt}

\begin{definition}[Exponential Series]
\[E(z):= \sum_{n=0}^\infty \frac{z^n}{n!},~z\iC\]

Ratio test: $|a_{n+1}/a_n| = \frac{z}{n+1} \to 0$ as $n \to \infty ~\forall z \iC\implies E(z)$ is absolutely convergent $\forall z\iC$.  
\end{definition}~

\begin{proposition}
$E(z)E(w) = E(z + w)$
\end{proposition}
\begin{proof}
By Cauchy product theorem
\[E(z)E(w) = \sum_{n=0}^\infty c_n\]
where $c_n = \sum_{i=0}^n \dfrac{z^i}{i!}\dfrac{w^{n-i}}{(n-i)!} \implies c_n = \dfrac{(z+w)^n}{n!}$.
\end{proof}

\begin{corollary}
$E(z) \neq 0$ and $\dfrac{1}{E(z)} = E(-z)$
\end{corollary}
\begin{proof}
$E(z)E(-z) = E(0) = 1$.	
\end{proof}\vspace*{5pt}

\begin{definition}
$e:= E(1) = \sum \frac{1}{n!} \in (-0,\infty)$	
\end{definition}\vspace*{5pt}

\begin{corollary}
$E(n) = e^n$ for $n \iN$	
\end{corollary}
\begin{proof}
	$E(n) = E(1 + (n-1)) = E(1)E(n-1) = \dots = (E(1))^n$.
\end{proof}

\begin{proposition}
$E(q) = e^q$ for $q \iQ$ (recall rational powers of $a \iR$ were defined in M1F)	
\end{proposition}
\begin{proof}
	Suppose $q > 0$; write $q = \frac{m}{n}$, $m,n \iN$. Then 
	\[\textstyle{E(q) = E(\underbrace{\frac{1}{n} + \dots + \frac{1}{n}}_{n \text{ times}}) = E(\frac{1}{n})^m}\]
	But \[E(\frac{1}{n})^n = E(\underbrace{\frac{1}{n} + \dots + \frac{1}{n}}_{m \text{ times}}) = E(1) = e\]
	
	\[ \implies E(\frac{1}{n}) = e^{1/n}\text{ and }E(q) = E(\frac{1}{n})^m = e^{m/n} = e^q\]
	
	If $q = \frac{-m}{n}$ then $E(q) = 1/E(m/n) = \frac{1}{e^{m/n}} = e^{-m/n} = e^q$.
\end{proof}

So we know that $E(x) = e^x ~\forall x \iQ$. Later we define $e^x~\forall x\iR$ by \emph{continuity} and we will show $E(x)$ is also continuous and so $E(x) = e^x~\forall x\iR$. \\
 
 
Some \lecturemarker{17}{20 Feb} useful properties of $E(x)$:
\begin{enumerate}
\item $x \geq 0 \implies E(x) \geq 1$ and $x > 0 \implies E(x) > 1$ (obvious from series)
\item $E(x) > 0~\forall x \iR$
\item $E(x)$ is strictly increasing for $x\iR$: $x < y \implies E(y) = E(x)E(y-x) > E(x).1$
\item $|x| < 1$ then $|E(x) - 1| < \frac{|x|}{1-|x|}$
\item $\RR \ni x \mapsto E(x)$ is a continuous bijection onto $(0,\infty)$. (proven later)
\item So we can define $\log: (0,\infty) \to \RR$ as inverse of $E$, i.e. $y = \log x$ defined by $\iff x = e^y$ with the usual log properties
\end{enumerate}~

We can also define $a^x$ for $a \in(0,\infty),x\in \RR$ by $a^x = E(x\log a)$

Ex: If $x \iQ$ this agrees with Corti's definition. 

And trig functions $\cos \theta = \Re E(i\theta)$, $\sin \theta = \Im E(i\theta)$ etc. 

Ex: $E(i\theta + i\phi) = E(i\theta)E(i\phi)$ implies what?
	


\pagebreak
\documentclass[twoside]{scrartcl}

\usepackage[blacklinks]{drangreport}
% \usepackage[caecilia]{drangreport}

\usepackage{pgfplots}

%Thomas' contradiction symbol
\newcommand*{\cont}{%
  \mathsf{X}\mkern-8.5mu\mathsf{X}%
}
\renewcommand{\thempfootnote}{$\dagger$}

%Change font of proof to scshape, not sure if it looks better...
\expandafter\let\expandafter\oldproof\csname\string\proof\endcsname
\let\oldendproof\endproof
\renewenvironment{proof}[1][\proofname]{%
  \oldproof[\scshape #1]%
}{\oldendproof}


\title{Analysis I}


\hypersetup{pdfinfo={
Title={Analysis I},
Author={Karim Bacchus},
Keywords ={Analysis I, M1P1, Lecture Notes, Imperial College, Maths},
}}


\begin{document}


\NotesTitle{1st}{Spring 2015}{Analysis I}{Prof.~R.}{Thomas}
{
Caveat Lector: unofficial notes, \emph{not} endorsed by Imperial College.\\[0.1cm]
Comments and corrections should be sent to \href{mailto:kb514@ic.ac.uk}{\texttt{kb514@ic.ac.uk}}
}
{

\section*{About these notes}

These notes are not affiliated with Richard Thomas (or even \emph{Professor} Richard Thomas!)\\ The \LaTeX{} source is available on \texttt{github} - it would be great if someone would update changes to the courses so they'll still be useful to later years. Notes for other courses are available on \texttt{dropbox}:

\begin{center}
\begin{tabular}{l | l}
\textbf{1st / 2nd Year} & \textbf{3rd / 4th Year}\\
Foundations of Analysis & Galois Theory\\
Analysis I & Algebraic Number Theory \\
Analysis II & Analytical Number Theory \\
Complex Analysis & Elliptic Curves  \\
Geometry \& Linear Algebra & Modular Forms \\
Algebra I & Algebra III \\
Algebra II & Commutative Algebra \\
Methods I & Lie Algebras  \\
Methods II & Measure \& Integration  \\
Multivariable Calculus & Functional Analysis\\
Differential Equations & Algebraic Topology \\
Metric Spaces \& Topology  & Differential Topology \\
Intro to Numerical Analysis & Complexity \\
Mechanics & General Relativity \\
\end{tabular}
\end{center}

\begin{flushright}
- Karim Bacchus 	
\end{flushright}



\pagebreak 
\thispagestyle{empty}


\section*{Syllabus}  

\textit{A rigorous treatment of the concept of a limit, as applied to sequences, series and functions.}
\begin{itemize}
\item  Real and complex sequences. Convergence, divergence and divergence to infinity. Sums and products of convergent sequences. The Sandwich Test. Sub-sequences, monotonic sequences, Bolzano-Weierstrass Theorem. Cauchy sequences and the general principle of convergence.

\item Real and complex series. Convergent and absolutely convergent series. The Comparison Test for non-negative series and for absolutely convergent series. The Alternating Series Test.  Rearranging absolutely convergent series. Radius of convergence of power series. The exponential series.

\item Limits and continuity of real and complex functions. Left and right limits and continuity. Maxima
and minima of real valued continuous functions on a closed interval. Inverse Function Theorem for
strictly monotonic real functions on an interval. 

\item  An introduction to differentiability: definitions, examples, left and right derivative.
\end{itemize}
\subsection*{Appropriate books}

{\shortskip

K.~G.~ Binmore, \textit{Mathematical Analysis, A Straightforward Approach} (Cambridge University Press).

}

}





\TableofContents
%!TEX root = M1P1.tex

\stepcounter{lecture}

\pagebreak
\setcounter{section}{-1}

\setcounter{lecture}{0}

\setcounter{page}{4}
\sektion{Preliminaries}
\lecturemarker{1}{5 Oct}


M1F stuff:

\begin{itemize}
\item $\forall - \text{ for any, \textbf{fix any}, for all, every...}$
\item $\exists - \text{ there exists}$
\item $\mathbb{N} = \{1,2,3,\dots\}$ 
\end{itemize}

\begin{theorem}[Triangle Inequality]
(See Question Sheet 1)
	\[|a+b| \leq |a| + |b|\]
\end{theorem}

\begin{corollary}
\[\left||a| - |b|\right|\leq |a-b|	\]
\end{corollary}
\begin{proof}
\[
\begin{aligned}
|a-b| < \epsilon &\iff b-\epsilon < a < b + \epsilon\\
&\iff a \in (b-\epsilon, b+\epsilon)\\
&\iff b \in (a-\epsilon, a+\epsilon)	\\
&\implies \left||a| - |b|\right|< \epsilon
\end{aligned} \] 
\end{proof}

Lots of other versions, see Question Sheet 1 - \emph{don't try to memorise them!}\\



\begin{clicker}
Fix $a \in \mathbb{R}$. What does the statement 
\[\forall \epsilon >0,~|x-a|<\epsilon ~(*)\]
mean for the number $x$? 

\textbf{Answer:} $x = a$. 
\begin{proof}
Assume $x \neq a$. Take $\epsilon := \frac{1}{2}|x-a| > 0$. Then $(*)$ does not hold.	
\end{proof}

\end{clicker}







\stepcounter{lecture}
\setcounter{lecture}{1}

\sektion{Sequences}
\lecturemarker{2}{5 Oct}
\label{sub:sequences}

A sequence $(a_n)_{n\geq 1}$ of real (or complex, etc.) numbers is an infinite list of numbers $a_1,~a_2,~a_3,\dots$ all in $\mathbb{R}$ (or $\mathbb{C}$, etc.) Formally:\\

\begin{definition}
	A \emph{sequence} is a function $a:\mathbb{N} \to \mathbb{R}$
\end{definition}

\textbf{Notation:} We let $a_n \in \mathbb{R}$ denote $a(n)$ for $n \in \mathbb{N}$. The data $(a_n)_{n=1,2,\dots}$ is equivalent to the function $a:\mathbb{N} \to \mathbb{R}$ because a function $a$ is determined by its values $a_n$ over all $n \in \mathbb{N}$. 

We will denote $a$ by $a_1,~a_2,\dots$ or $(a_n)_{n\in\mathbb{N}}$ or $(a_n)_{n\geq 1}$ or even just $(a_n)$.

\begin{remark}
$a_i$'s could be repeated, so $(a_n)$ is \emph{not} equivalent to the set $\{a_n : n \in \mathbb{N}\}\subset \mathbb{R}$. E.g. $(a_n) = 1,~0,~1,~0,\dots$ is different from $(b_n) = 1,~0,~0,~1,~0,~0,~1,\dots$	
\end{remark}

We can describe a sequence in may ways, e.g. formula for $a_n$ as above $a_n = \frac{1-(-1)^n}{2}$, or a recursion e.g. $c_1 = 1 = c_2$, $c_n = c_{n-1} + c_{n-2}$ for $n\geq 3$, or a summation (see next section) e.g. $d_n = \sum_{i=1}^n \frac{1}{i} = 1 + \frac{1}{2} + \frac{1}{3} + \dots +\frac{1}{n}$.

\subsektion{Convergence of Sequences}
We want to \emph{rigorously} define $a_n \to a \in \mathbb{R}$, or ``$a_n$ converges to $a$ as $n \to \infty$" or ``$a$ is the limit of $(a_n)$". 

Idea: $a_n$ should get closer and closer to $a$. Not necessarily monotonically, e.g.:
\[a_n = 
\begin{cases}
\frac{1}{n} & n \text{ odd}\\
\frac{1}{2n} & n \text{ even}	
\end{cases}
\hspace*{20pt}\text{ we want } a_n \to 0\]


\begin{center}
\begin{tikzpicture}
\begin{axis}[axis lines=middle,
     x label style={at={(axis description cs:0.95,-0.15)}},
     y label style={at={(axis description cs:-0.1,0.9)}},
    xlabel={$n$},
    ylabel={$a_n$},
    ticks=none,
  ymin = 0,
  ymax = 0.4,
  xmin = 1,
  xmax = 10,
  width=10cm,height=5cm]
   \addplot[samples at={3,5,7,9}, only marks, mark=x, mark size=3pt]{1/x};
    \addplot[samples at={2,4,6,8},only marks,mark=x,mark size=3pt]{(1/(2*x))};
  \end{axis}
\end{tikzpicture}
\end{center}
Also notice that $\frac{1}{n}$ gets closer and closer to $-1$! So we want to say instead that $a_n$ gets \emph{as close as we like to $a$}. We will measure this with $\epsilon >0$. We phrase ``$a_n$ gets \emph{arbitrarily} close to $a$" by ``$a_n$ gets to within $\epsilon$ of $a$ for \emph{any} $\epsilon >0$".\\

\begin{definition}[Mestel]
	$u_n \to u$ if $\forall n$ sufficiently large, $|u_n - u|$ is \emph{arbitrarily small.} 
	
Define a real number $b \in \mathbb{R}$ to be arbitrarily small if it is smaller than any $\epsilon >0$ i.e. $\forall \epsilon >0,~|b| < \epsilon$.
\end{definition}

Definition Mestel says that once $n$ is large enough, $|u_n - u|$ is less than every $\epsilon >0$, i..e it's zero, i.e. $u_n = u$. We want to \emph{reverse} the order of specifying $n$ and $\epsilon$. 

i.e. we want to say that to get \emph{arbitrarily close to the limit $a$} (i.e. $|a_n - a| < \epsilon$), we need to go sufficiently far down the sequence. Then if I change $\epsilon >0$ to be smaller, I simply go further down the sequence to get within $\epsilon$ of $a$. 

\begin{center}
\begin{tikzpicture}
\begin{axis}[
 axis line style={red},
axis lines=middle,
     x label style={at={(axis description cs:1.05,0.33)}},
     y label style={at={(axis description cs:-0.1,1.1)}},
    xlabel={\small $n$},
    ylabel={$a_n \to 0$},
  ymin = -0.4,
  ymax = 0.7,
  xmin = 1,
  xmax = 20,
    ytick = {0.3,0.12},
   yticklabels={$\epsilon_1$,$\epsilon_2$},
   xtick = 0,
  width=12cm,height=8cm]
   \addplot[samples at={2,...,20}, only marks, mark=x, mark size=2pt,  mark options={red},]{1/x};
   \draw (axis cs:20,0.3) -- (axis cs:0,0.3);
      \draw (axis cs:20,0.12) -- (axis cs:0,0.12);
     \draw (axis cs:20,-.1) -- (axis cs:4,-.1) node at (axis cs:10,-0.15){\small $n$ suff. large for $|a_n - 0| < \epsilon_1$};
          \draw (axis cs:4,-.06) -- (axis cs:4,-.1);
          \draw (axis cs:20,-.06) -- (axis cs:20,-.1);
          \draw (axis cs:20,-.25) -- (axis cs:9,-.25) node at (axis cs:15,-0.3){\small $n$ suff. large for $|a_n - 0| < \epsilon_2$};
           \draw (axis cs:9,-.2) -- (axis cs:9,-.25);
          \draw (axis cs:20,-.2) -- (axis cs:20,-.25);

  \end{axis}
\end{tikzpicture}
\end{center}

There will not be a ``$n$ sufficiently large" that works for all $\epsilon$ at once! (unless $a_n = a$ eventually.)

But for \emph{any} (fixed) $\epsilon>0$ we want there to be an $n$ sufficiently large such that $|a_n - a| < \epsilon$. So we change ``$\exists n$ such that $\forall \epsilon$" to ``$\forall \epsilon,~\exists n.$". \emph{This allows $n$ to depend on $\epsilon$.}~\\

\begin{definition}[Nestel]
	$a_n \to a$ if $\forall \epsilon >0,~\exists n \in \mathbb{N}$ such that $|a_n - a| < \epsilon$. 
\end{definition}

e.g. 
\[
a_n = \begin{cases}
 0 & n\text{ even}\\
 1 & n\text{ odd}	
 \end{cases}
 \text{ satisfies } a_n \to 0 \text{ according to Prof. Nestel.}
\]

We want to modify this to say eventually $|a_n - a| < \epsilon$ \emph{and it stays there!}\vspace*{5pt}

\textbf{Ignore Mestel and Nestel's definition!}\lecturemarker{3}{5 Oct}\\

\begin{definition}[Convergence]
We say that $a_n \to a$ iff 
\[\forall \epsilon >0,~\exists N \in \mathbb{N} \text{ such that } ``n \geq N \implies |a_n - a| < \epsilon"\]	
\end{definition}

This says that \emph{however close} ($\forall \epsilon>0$) I want to get to the limit $a$, there's a point in the sequence ($\exists N \in \mathbb{N}$) beyond which ($n \geq N$) my $a_n$ is indeed that close to the limit $a$ ($|a_n - a| <\epsilon$).\\ 

\begin{remark}
$N$ depends on $\epsilon$! $N = N(\epsilon)$	
\end{remark}

Equivalently:
\[\forall \epsilon >0,~ \exists N \in \mathbb{N} \text{ such that} ``\forall n \geq N,~|a_n -a|<\epsilon"\]
or equivalently
\[\forall \epsilon >0,~\exists N_\epsilon\in\mathbb{N} \text{ such that } |a_n - a| < \epsilon,~\forall n \geq N_\epsilon\]

\begin{clicker}Given a sequence of real numbers $(a_n)_{n\geq 1}$. Conisder 
\[\boxed{\forall n \geq 1,~\exists \epsilon >0 \text{ such that } |a_n|< \epsilon}\]
This means? 

\textbf{Answer:} It always holds. \begin{proof}
 Fix any $n \in \mathbb{N}$. Take $\epsilon = |a_n| + 1$. 	
 \end{proof}
 
What about
 \[\boxed{\exists \epsilon >0 \text{ such that } \forall n \geq 1,~|a_n| < \epsilon} \]
 
 \textbf{Answer:} $(a_n)$ is bounded.
 
 
\begin{center}
\begin{tikzpicture}
\begin{axis}[
 axis line style={red},
axis lines=middle,
     x label style={at={(axis description cs:1.05,0.4)}},
     y label style={at={(axis description cs:-0.1,1.1)}},
    xlabel={\small $n$},
    ylabel={$a_n$},
  ymin = -0.4,
  ymax = 0.4,
  xmin = 1,
  xmax = 10,
     ytick = {0.3, -.3},
   yticklabels={$\epsilon$,$-\epsilon$},
   xtick = 0,
  width=10cm,height=5cm]
   \addplot[samples at={5,7,9}, only marks, mark=x,  mark options={red}, mark size=3pt]{(-1)^x*1/x};
    \addplot[samples at={2,3,4,6,8},only marks, mark options={red},mark=x,mark size=3pt]{(1/(2*x))};
       \draw (axis cs:20,0.3) -- (axis cs:0,0.3);
      \draw (axis cs:20,-0.3) -- (axis cs:0,-0.3);
  \end{axis}
\end{tikzpicture}
\end{center}


  \begin{proof}
$\iff a_n \in (-\epsilon,\epsilon)~ \forall n \iff |a_n|$ is bounded by $\epsilon$. \end{proof}

\end{clicker}~

\begin{definition}
If $a_n$ does not converge to $a$ for any $a\in \mathbb{R}$, we say that $a_n$ \emph{diverges.}
\end{definition}~

\begin{example}
I claim that $\frac{1}{n} \to 0$ as $n \to \infty$

\textit{Rough working:} Fix $\epsilon >0$. I want to find $N \in \mathbb{N}$ such that $|a_n - a| = |\frac{1}{n} - 0| = \frac{1}{n} < \epsilon$ for all $n \geq N$. 
\begin{center}
\begin{tikzpicture}
\begin{axis}[
 axis line style={red},
axis lines=middle,
     x label style={at={(axis description cs:1.05,0.1)}},
     y label style={at={(axis description cs:-0.1,1.1)}},
    xlabel={\small $n$},
    ylabel={$a_n$},
  ymin = -0.1,
  ymax = 0.4,
  xmin = 1,
  xmax = 10,
     ytick = {0.15},
   yticklabels={$\epsilon$},
   xtick = 0,
  width=10cm,height=5cm]
   \addplot[samples at={5,7,9}, only marks, mark=x,  mark options={red}, mark size=3pt]{(-1)^x*1/x};
    \addplot[samples at={2,3,4,6,8},only marks, mark options={red},mark=x,mark size=3pt]{(1/(2*x))};
       \draw (axis cs:20,0.15) -- (axis cs:0,0.15);
        \draw (axis cs:4,0.12) -- (axis cs:4,-0.1) node at (axis cs:4.2,-0.05) {$N$};
  \end{axis}
\end{tikzpicture}
\end{center}


Since $a_n = \frac{1}{n}$ is monotonic, it is \emph{sufficient} to ensure that $\frac{1}{N} < \epsilon \iff N > \frac{1}{\epsilon}$ (This \emph{implies} $\frac{1}{n} \leq \frac{1}{N} < \epsilon,~\forall n \geq N$).

\begin{proof}
Fix $\epsilon >0$. 
 Pick any $N \in \mathbb{N}$ such that $N > \frac{1}{\epsilon}$. (This is the Archimedean axiom of $\mathbb{R}$. Notice $N$ depends on $\epsilon$!!). Then $n \geq N \implies |\frac{1}{n}-0| = \frac{1}{n} \leq \frac{1}{N} < \epsilon$.
\end{proof}


\end{example}


\textbf{Method to prove $a_n \to a$}
\begin{enumerate}
\item[(I)] Fix $\epsilon > 0$
\item[(II)] Calculate $|a_n - a|$
\item[(II$'$)] Find a good estimate $|a_n - a| < b_n$
\item[(III)] Try to solve $a_n - a < b_n < \epsilon ~~(*)$
\item[(IV)] Find $N \in \mathbb{N}$ s.t. $(*)$ holds whenever $n \geq N$
\item[(V)] Put everything together into a logical proof (usually involves rewriting everything in reverse order - see examples below)
\end{enumerate}~

\begin{example}
$a_n = \dfrac{n+5}{n+1}$\\

\emph{Rough Working}
\[ |a_n - 1| = \left| \dfrac{n+5}{n+1} -1 \right| = \dfrac{4}{n+1}\]

This is $< \epsilon \iff n+1 > 4/\epsilon \iff n > 4/\epsilon$, so take $N \geq 4/\epsilon$. 

\begin{proof}
Fix $\epsilon > 0$. Pick $N$ such that $N \geq 4/\epsilon$. Then $\forall n \geq N$, \[|a_n - 1| = \frac{4}{n+1} \leq \frac{4}{N+1} < \frac{4}{N} \leq \epsilon\qedhere\]
\end{proof}
\end{example}~

\begin{example}$a_n = \dfrac{n+2}{n-2} \to 1$\\

\emph{Rough Working} 
\[|a_n -1| = \left|\frac{n+2}{n-2} - 1\right| = \frac{4}{n-2}\]

We want $\frac{4}{n-2} < \epsilon$. We want implications in the $\impliedby$ direction (i.e. $\frac{4}{n-2} < \epsilon \impliedby n \geq N$) \emph{not} $\implies$ direction. i.e. $\frac{4}{n-2} < \epsilon \implies \frac{4}{n} < \epsilon$.

But if we take $N = \frac{4}{\epsilon}$, we need the \emph{opposite} implication, we \emph{need} $\frac{4}{n-2} < \epsilon$. We \emph{need} to estimate $\frac{4}{n-2} < b_n$, and then solve $b_n < \epsilon$. So we make denominator smaller. 

To make $n-2$ smaller, make $2$ bigger! e.g. $\frac{n}{2}>2$ for $n>4$. Then $\frac{4}{n-2} < \frac{4}{n-n/2} = \frac{8}{n}$

Also want $b_n = \frac{8}{n} < \epsilon \iff n > 8/\epsilon$. So take $N > \mathrm{max}(8/\epsilon, 4)$.

\begin{proof}
Fix $\epsilon > 0.$	Choose $N \iN$ such that $N > \mathrm{max}(8/\epsilon, 4)$. Then $n \geq N \implies n > 8/\epsilon~ (1)$ and $n > 4 ~(2) \implies$

\[\left|\frac{n+2}{n-2} - 1\right| = \frac{4}{n-2} \underbrace{<}_{(2)} \frac{4}{n-n/2} = \frac{8}{n} \underbrace{<}_{(1)} \epsilon\qedhere\]

\end{proof}

	
\end{example}
\pagebreak


\lecturemarker{4}{5 Oct}
We can also define limits for \emph{complex sequences}.\\

\begin{definition}
$a_n \in \mathbb{C},~\forall n \geq 1$. We say $a_n \to a \in \mathbb{C}$ iff
\[\forall \epsilon >0,~\exists N \in \mathbb{N} \text{ such that } n \geq N \implies |a_n - a| < \epsilon\]	

(i.e. $\sqrt{\Re(a_n-a)^2 + \frak{I}(a_n-a)^2} < \epsilon$) 

This is equivalent (see problem sheet!) to $(\frak{R}a_n) \to \frak{a}$ \emph{and} $(\frak{I}a_n) \to \frak{I}a$
\end{definition}~

\begin{example}
Prove $a_n = \dfrac{e^{in}}{n^3-n^2-6} \to 0$ as $n \to \infty$\\

\emph{Rough Working} 
\[|a_n - a| = \left|\dfrac{e^{in}}{n^3-n^2-6} \right| = \left|\frac{1}{n^3-n^2-6}\right|\]

\emph{Estimate} $\left|\dfrac{1}{n^3-n^2-6}\right| < \dfrac{1}{c_n}$ by making $c_n$ smaller than $n^3 - n^2 - 6$ (But not too small! We want $c_n \to \infty$). So let $c_n = n^3 - \text{\emph{ something bigger than} } n^2 + 6$.

Take off $\frac{n^3}{2}$ to make the expression simple. For $n \geq 4$, we have $\frac{n^3}{2} > n^2 + 6$. 

So for $n \geq 4$
\[\left|\frac{1}{n^3-n^2-6}\right| < \frac{1}{n^3- n^3/2} = \frac{2}{n^3} \]
and this is $< \epsilon$ for $n > \sqrt[3]{\frac{2}{\epsilon}}$.

\begin{proof}
$\forall \epsilon > 0$, choose $N \geq \mathrm{max}(4,\sqrt[3]{2/\epsilon})$	. Then $\forall n \geq N$
\[|a_n - 0| = \left| \frac{1}{n^3-n^2 - 6}\right| < \frac{1}{n^3- n^3/2} = \frac{2}{n^3} \leq \frac{2}{N^3} \leq \epsilon \qedhere \]
\end{proof}
\end{example}~


\begin{example}
Set $\delta = 10^{-1000000}, ~a_n = (-1)^n\cdot\delta$. Prove that $a_n$ does not converge. 

We want to show that the following is false: 
\[\exists a \text{ s.t. } \forall \epsilon > 0,~\exists N \in \mathbb{N} \text{ s.t. } n \geq N \implies |a_n - a| <\epsilon\]
i.e. we need to prove
\[\forall a,~ \exists \epsilon > 0 \text{ s.t. } \forall N \in \mathbb{N},~\exists n \geq N \text{ s.t. } |a_n-a| \geq \epsilon\]

\emph{Rough:} 	
Assume for contadiction that $a_n \to a$, i.e. $\forall \epsilon > 0,~\exists N \in \mathbb{N} \text{ s.t. } n \geq N \implies |a_n - a| <\epsilon$


\begin{center}
\begin{tikzpicture}
\begin{axis}[
 axis line style={red},
axis lines=middle,
     x label style={at={(axis description cs:1.05,0.4)}},
     y label style={at={(axis description cs:-0.1,1.1)}},
    xlabel={\small $n$},
    ylabel={$a_n$},
  ymin = -0.4,
  ymax = 0.5,
  xmin = 1,
  xmax = 10,
     ytick = {0.35, -0.05,0.1},
   yticklabels={$a+\epsilon$,$a-\epsilon$, $a$},
   xtick = 0,
  width=10cm,height=5cm]
   \addplot[samples at={3,5,7,9}, only marks, mark=x,  mark options={red}, mark size=3pt]{0.2};
    \addplot[samples at={2,4,6,8,},only marks, mark options={red},mark=x,mark size=3pt]{-0.2};
       \draw (axis cs:20,0.35) -- (axis cs:0,0.35); %a+e line
      \draw (axis cs:20,-0.05) -- (axis cs:0,-0.05); %a-e line
       \draw[dashed] (axis cs:20,0.1) -- (axis cs:0,0.1); %a line
         \fill[gray!40,nearly transparent] (axis cs:0,-.05)-- (axis cs: 0,0.35) -- (axis cs: 10,0.35) -- (axis cs: 10,-0.05) -- cycle;
  \end{axis}
   \draw[->] (9,2.1) -- (8,2.1);
      \draw[->] (9,0.8) -- (8,0.8);
      \node[draw,text width=2cm] at (10.2,2.1) {\small Eventually all pts are within $\epsilon$ of $a$};
        \node[draw,text width=2cm] at (10.2,0.8) {\small Agh!};
\end{tikzpicture}
\end{center}


For small enough $\epsilon >0$, the fact that $a$ is within $\epsilon$ of $\delta ~ (a_{2n})$ and $-\delta ~(a_{2n+1})$ will be a contradiction. 


\begin{proof}
Fix $a\iR$. Take $\epsilon = \delta$ (or $\epsilon < \delta$ will do). 

 Then if $\exists N$ s.t. $\forall n \geq N$, $|a_n - a| <\epsilon$ this implies
\begin{enumerate}
\item $|a_{2N} - a | < \epsilon \iff a \in (\delta - \epsilon, \delta + \epsilon)\implies a > \delta - \epsilon =0$ 
\item $|a_{2N+1} - a | < \epsilon\iff a \in (-\delta - \epsilon,-\delta + \epsilon) \implies a < -\delta + \epsilon = 0,~\cont$ 
\end{enumerate}
(or use triangle inequality: \[|\delta - (-\delta)| \leq |\delta - a| + |a -(-\delta)| < \epsilon + \epsilon \implies 2\delta < 2\epsilon = 2\delta ~\cont\text{)}\]
So $a_n \not\to a$, but this holds $\forall a \in \RR$, so $a_n$ does not converge.
\end{proof}
\end{example}~

\begin{clicker}
Fix $(a_n)_{n\geq 1},~a_n \in \RR$. Then \[\forall n,~\exists \epsilon >0 \text{ s.t. } |a_n| < \epsilon \text{ means?}\]

\textbf{Answer:} Nothing. This is always true. Take $\epsilon = |a_n| + 1$	
\end{clicker}~

\lecturemarker{5}{5 Oct}

\begin{theorem}[Uniqueness of Limits]
Limits are unique. If $a_n \to a$ and $a_n \to b$, then $a=b$	
\end{theorem}

\emph{Idea:} For $n$ large, $a_n$ should be close to $a$ and to $b$. So $a$ arbitrarily close to $b \implies a = b$. 

\begin{proof}[Proof 1]~
\begin{enumerate}
\item $\forall \epsilon,~\exists N_a$ s.t. $\forall n \geq N_a,~|a_n - a| < \epsilon$ 

\item $\forall \epsilon,~\exists N_b$ s.t. $\forall n \geq N_b,~|a_n - b| < \epsilon$
\end{enumerate}

Set $N = \mathrm{max}(N_a,N_b)$. Then $\forall n \geq N$, (i) and (ii) hold, so
\[|a - b| = |(a-a_n) + (a_n - b)| \leq |a - a_n| + |a_n - b| < 2\epsilon \implies |a-b| = 0!\qedhere\]
(recall! if not, set $\epsilon = \frac{1}{2}|a-b| >0$ to get a contradiction)
\end{proof}~


\begin{proof}[Proof 2]
By contradiction. Assume $a \neq b$. 

\begin{center}
\begin{tikzpicture}
\begin{axis}[
 axis line style={red},
axis lines=middle,
     x label style={at={(axis description cs:1.05,0.4)}},
     y label style={at={(axis description cs:-0.1,1.1)}},
    xlabel={\small $n$},
    ylabel={$a_n$},
  ymin = -0.4,
  ymax = 0.5,
  xmin = 1,
  xmax = 10,
     ytick = {0.3, 0.2,0.1,-0.1,-0.2,-0.3},
   yticklabels={$a+\epsilon$,$a$, $a-\epsilon$,$b+\epsilon$, $b$, $b-\epsilon$},
   xtick = 0,
  width=12cm,height=6cm]
       \draw (axis cs:20,0.3) -- (axis cs:0,0.3); %a+e line
      \draw (axis cs:20,0.1) -- (axis cs:0,0.1); %a-e line
       \draw[dashed] (axis cs:20,0.2) -- (axis cs:0,0.2); %a line
         \fill[gray!40,nearly transparent] (axis cs:0,.1)-- (axis cs: 0,0.3) -- (axis cs: 10,0.3) -- (axis cs: 10,0.1) -- cycle;
         
          \draw(axis cs:20,-0.3) -- (axis cs:0,-0.3); %b+e line
      \draw (axis cs:20,-0.1) -- (axis cs:0,-0.1); %b-e line
       \draw[dashed]  (axis cs:20,-0.2) -- (axis cs:0,-0.2); %b line
         \fill[gray!40,nearly transparent] (axis cs:0,-.1)-- (axis cs: 0,-0.3) -- (axis cs: 10,-0.3) -- (axis cs: 10,-0.1) -- cycle;
  \end{axis}
\end{tikzpicture}
\end{center}

Eventually $a_n$ is in \emph{both} corridors. So if I choose $\epsilon$ sufficiently small so that corridors don't overlap to get a contradiction.

Set $\epsilon = \frac{|a-b|}{2} > 0$. Then $\exists N_a, N_b$ such that $\forall n \geq N_a,N_b$, we have 
\[|a_n - a| < \epsilon \text{ and } |a_n - b| < \epsilon\]
w.l.o.g. $a > b$. Then $a_n > a - \epsilon$ and $a_n < b$
\[\begin{aligned}
	&\implies b + \epsilon > a-\epsilon\\
	&\implies 2\epsilon > a-b = 2\epsilon~ \cont 
\end{aligned}\]
\end{proof}

\begin{clicker}
Prove $\frac{1}{n-2} \to 0$. Student Answer:
Fix $\epsilon >0$.
\begin{enumerate}
	\item We want $|\frac{1}{n-2} - 0| = \frac{1}{n-2} < \epsilon$
	\item $\implies n-2 > 1/\epsilon$
	\item $\implies n > 2 + 1/\epsilon$
	\item $\implies n > 1/\epsilon ~(*)$
	\item So take $N > 1/\epsilon$, then
	\item $\forall n \geq N$, $n > 1/\epsilon$ which is $(*)$
	\item So $\frac{1}{n-2} \to 0$
	\item (This is correct)
\end{enumerate}

\textbf{Answer:} (iv) is wrong. 
	
\end{clicker}


\begin{theorem}[Algebra of Limits]\label{thm1}
$a_n \to a$ and $b_n \to b$ then:\begin{enumerate}
\item $a_n+b_n \to a+b$
\item $a_nb_n \to ab$
\item $\frac{a_n}{b_n} \to \frac{a}{b} ~(b \neq 0)$
\end{enumerate}
\end{theorem}
\begin{proof}[Proof of (i)]
Fix any $\epsilon >0$. Then $\exists N_a \in \mathbb{N}$ such that $\forall n\geq N_a,~ |a_n	 - a| < \epsilon/2$ 

and $\exists N_b \in \mathbb{N}$ such that $\forall n \geq N_b,~ |b_n - b| < \epsilon/2$. Set $N =$ max$\{N_a,N_b\}$, so 
\[\begin{aligned}|(a_n + b_n) - (a+b)| &\leq |a_n -a| + |b_n -b| \\
	 &< \epsilon/2 + \epsilon/2 = \epsilon \qedhere
\end{aligned}\]
\end{proof}

\begin{proof}[Proof of (ii)]
\emph{Rough working:} 	
\[\begin{aligned}|a_nb_n -ab|  &= |(a_n -a)b - a_nb + a_nb_n|\\ 
&\leq |a_n-a||b| + |a_n|	|b_n-b|
\end{aligned}
\]

We can easily make $|a_n - a| < \epsilon/2$ if I take $|a_n - a| < \frac{\epsilon}{2|b|}$.

We need to show that $|a_n| < A$, so that I can take $|b_n - b| < \frac{\epsilon}{2A}$.\\

\begin{lemma}
If $a_n \to a$, then $(a_n)$ is bounded: $\exists A \in \RR \text{ s.t. }|a_n| < A,~\forall n$.
\end{lemma}
\begin{proof}[Proof of Lemma]~

\begin{center}
\begin{tikzpicture}
\begin{axis}[
 axis line style={red},
axis lines=middle,
     x label style={at={(axis description cs:1.05,0.4)}},
     y label style={at={(axis description cs:-0.08,1.02)}},
    xlabel={\small $n$},
    ylabel={$a_n$},
  ymin = -0.5,
  ymax = 0.5,
  xmin = 1,
  xmax = 10,
     ytick = {0.25, -0.05,0.1},
   yticklabels={$a+1$,$a-1$, $a$},
   xtick = 0,
  width=12cm,height=8cm]
   \addplot[samples at={3,5,7,9}, only marks, mark=x,  mark options={red}, mark size=3pt]{1/x};
    \addplot[samples at={2,4,6,8},only marks, mark options={red},mark=x,mark size=3pt]{-1/(x^2)+0.1};
       \draw (axis cs:20,0.25) -- (axis cs:0,0.25); %a+e line
      \draw (axis cs:20,-0.05) -- (axis cs:0,-0.05); %a-e line
       \draw[dashed] (axis cs:20,0.1) -- (axis cs:0,0.1); %a line
         \fill[gray!40,nearly transparent] (axis cs:0,-.05)-- (axis cs: 0,0.25) -- (axis cs: 10,0.25) -- (axis cs: 10,-0.05) -- cycle;
            \draw (axis cs:4,0.0375) -- (axis cs:4,-0.1) node at (axis cs:4,-0.15) {$N$};
                  \node[text width=3cm] at (axis cs:2.8,-0.35) {\small Finitely many points $<N$};
            
  \end{axis}
\end{tikzpicture}
\end{center}

Fix $\epsilon =1$. Then $\exists N \in \NN$ such that $\forall n \geq N,~|a_n - a| < 1 \implies |a_n| < 1 + |a|$. 

Then $(a_n)$ is bounded by max$\{a_1,a_2,\dots,a_{N-1},a+1\}$.
\end{proof}


Fix $\epsilon >0$. Then $\exists N_a$ such that $\forall n \geq N_a$, $|a_n - a| < \dfrac{\epsilon}{2(|b| + 1)}$ (we add $1$ in case $|b| = 0$)
and $\exists N_b$ such that $\forall n \geq N_b$, $|b_n - b| < \dfrac{\epsilon}{2A}$. 

Set $N = \text{max}(N_a,N_b)$. Then $\forall n \geq N$
\[\begin{aligned}|a_nb_n -ab| &\leq |a_n-a||b_n| + |b_n-b||a|\\ 
&< \frac{\epsilon}{2}\frac{|b|}{|b|+1} + A\frac{\epsilon}{2A}\\ 
& < \epsilon/2 + \epsilon/2 = \epsilon\qedhere	
\end{aligned}
\]
\end{proof}

See exercise sheet for proof of \ref{thm1}iii.\\

\begin{theorem}
If $(a_n)$ is bounded above \emph{and} monotonically increasing then $a_n$ is \emph{convergent}.
\end{theorem}

\emph{Idea:} 
\begin{center}
\begin{tikzpicture}
\begin{axis}[
 axis line style={red},
axis lines=middle,
     x label style={at={(axis description cs:1.05,0.1)}},
     y label style={at={(axis description cs:-0.1,1.1)}},
    xlabel={\small $n$},
    ylabel={$a_n$},
  ymin = -0.1,
  ymax = 0.42,
  xmin = 1,
  xmax = 20,
     ytick = {0.26,0.38,0.32},
   yticklabels={$a-\epsilon$,$a+\epsilon$,$a$},
   xtick = 0,
  width=12cm,height=7cm]
   \addplot[samples at={1,2,3,4,5,6,7,8,9,10,11,12,13,14,15,16,17,18,19}, only marks, mark=x,  mark options={red}, mark size=3pt]{0.35-1/x};
       \draw (axis cs:20,0.38) -- (axis cs:0,0.38);
              \draw (axis cs:20,0.26) -- (axis cs:0,0.26);
                     \draw[dashed] (axis cs:20,0.32) -- (axis cs:0,0.32);
         \fill[gray!40,nearly transparent] (axis cs:0,0.26)-- (axis cs: 0,0.38) -- (axis cs: 20,0.38) -- (axis cs: 20,0.26) -- cycle;
                     \draw (axis cs:12,0.262) -- (axis cs:12,-0.02) node at (axis cs:12,-0.05) {$N$};
  \end{axis}
\end{tikzpicture}
\end{center}
Eventually we get in the epsilon corridor (shaded area) because $a-\epsilon$ is \emph{not} an upper bound. We stay in there because monotonic and bounded by $a$.
\begin{proof}
Fix $\epsilon >0$. $a-\epsilon$ is \emph{not} an upper bound for $\{a_n : n \iN\}$ (because $a$ is the \emph{smallest} upper bound). So $\exists N \in \mathbb{N}$ such that $a_N > a-\epsilon$. Monotonic so $\forall n \geq N$ we have \[a\geq a_n\geq a_N > a-\epsilon \implies |a_n - a| < \epsilon\qedhere\]	
\end{proof}~

\begin{remark}\lecturemarker{6}{5 Oct}
 Now it's easier to handle things like $a_n = \dfrac{n^2 + 5}{n^3 - n + 6}$.

Dividing by $n^3$, we get $a_n = \dfrac{1/n + 5/n^3}{1 - 1/n^2 + 6/n^3}$.

 Use the fact that $1/n \to 0$ as $n \to \infty$ (Recall proof: $\forall \epsilon >0$, let $N_\epsilon  > 1/\epsilon$, then $n\geq N_{\epsilon} \implies n > 1/\epsilon \implies 1/n < \epsilon$), and the algebra of limits to deduce that 
\[a_n \to \dfrac{0 + 5.0^3}{1 - 0^2 + 6.0^3} = 0.\]
\end{remark}




\subsektion{Cauchy Sequences}
Gives a way of proving convergence \emph{without} knowing the limit.\\

\begin{definition}
	A sequence is Cauchy iff
	\[\forall \epsilon >0,~\exists N \in \mathbb{N} \text{ s.t. } \forall n,m \geq N,~ |a_n - a_m| < \epsilon\]
\end{definition}~

\begin{remark}
$m,n \geq N$ are arbitrary. It is not enough to say that $\forall \epsilon >0,~\exists N \in \NN$ such that $n \geq N \implies |a_n - a_{n+1}| < \epsilon$. See ex sheet. 	
\end{remark}

\begin{proposition}	
If $a_n \to a$ then $(a_n)$ is Cauchy. 
\end{proposition}
\begin{proof}
$a_n \to a\implies \forall \epsilon >0,~\exists N \text{ s.t. } n \geq N \implies |a_n - a| < \epsilon/2 ~(1)$


So $m \geq N \implies |a_m - a| < \epsilon /2 ~(2)$. So \[m,n \geq N \implies |a_n - a_m| \leq |a_n - a| + |a_m - a| < \underbrace{\epsilon/2}_{(1)} + \underbrace{\epsilon/2}_{(2)}=\epsilon\qedhere\]
\end{proof}

We want to prove converse: Cauchy $\implies$ Convergence.

We need a candidate for the limit $a$

\begin{center}
\begin{tikzpicture}
\begin{axis}[
 axis line style={red},
axis lines=middle,
     x label style={at={(axis description cs:1.05,0.4)}},
     y label style={at={(axis description cs:-0.1,1.1)}},
    xlabel={\small $n$},
    ylabel={$a_n$},
  ymin = -0.5,
  ymax = 0.5,
  xmin = 1,
  xmax = 10,
     ytick = {0.15, 0.05},
   yticklabels={$a_{N_{\epsilon}}+\epsilon$,$a_{N_{\epsilon}}-\epsilon$},
   xtick = 0,
  width=10cm,height=5cm]
   \addplot[samples at={3,5,7,9}, only marks, mark=x,  mark options={red}, mark size=3pt]{1/x};
    \addplot[samples at={2,4,6,8},only marks, mark options={red},mark=x,mark size=3pt]{-1/(x^2)+0.1};
       \draw (axis cs:20,0.15) -- (axis cs:0,0.15); %a+e line
      \draw (axis cs:20,0.05) -- (axis cs:0,0.05); %a-e line
       %\draw[dashed] (axis cs:20,0.1) -- (axis cs:0,0.1); %a line
         \fill[gray!40,nearly transparent] (axis cs:0,.05)-- (axis cs: 0,0.15) -- (axis cs: 10,0.15) -- (axis cs: 10,0.05) -- cycle;
            \draw (axis cs:6,0.04) -- (axis cs:6,-0.1) node at (axis cs:6,-0.15) {$N_\epsilon$};
     
            
  \end{axis}
\end{tikzpicture}
\end{center}

We will produce an auxiliary sequence which is \emph{monotonic} (+ bounded) $\implies$ convergence. $b_n:= \mathrm{sup}\{a_i : i \geq n\}$. Then picture shows that $b_{N_{\epsilon}} \in (a_{N_{\epsilon}} - \epsilon, a_{N_{\epsilon}} + \epsilon]$ and $b_n$'s are monotonically \emph{decreasing} because $b_{n+1} = \mathrm{sup}\{a_i : i \geq n+1\}$, a subset of $\{a_i : i \geq n\}$. 

So $b_n$s converge to $\mathrm{inf}\{b_n: n \iN\}$. We will show that $a_n$'s converge to same number, $a$, using Cauchy condition. 

\begin{lemma}
$(a_n)$ is Cauchy $\implies (a_n)$ is bounded
\end{lemma}

\begin{proof}
Pick $\epsilon =1$, then $\exists N$ such that $\forall n,m \geq N$, $|a_n - a_m|  < 1$. In particular $|a_n| < 1 + |a_N|~\forall n \geq N$ (take $m = N$), so 
\[|a_n| \leq \mathrm{max}\{|a_1|,|a_2|,\dots|a_{N-1}|,1+|a_N|\}~\forall N \iN\qedhere\]	
\end{proof}


\begin{theorem}
$(a_n)$ is a Cauchy sequence of real numbers $\implies a_n$ convergent. 
\end{theorem}

\begin{corollary}
$(a_n)$ Cauchy $\iff (a_n)$ convergent. (Ex: Show not true in $\QQ$!) 	
\end{corollary}

\begin{proof}
$(a_n)$ Cauchy $\implies$ bounded. So we can define $b_n = \mathrm{sup}\{a_i : i \geq n\}$. Then define $a = \mathrm{inf}\{b_n : n \iN\}$ and we prove that $a_n \to a$.

Fix $\epsilon > 0$. $\exists N \iN$ s.t. $n,m \geq N  \implies |a_n - a_m| < \epsilon/2 \iff a_n - \epsilon/2 < a_m < a_n + \epsilon /2$.
Take supremum over all $m \geq i \geq N$
\[\begin{aligned}
\implies a_n - \epsilon/2 &< \mathrm{sup}	\{a_m: m \geq i\} \leq a_n + \epsilon/2\\
\text{ i.e. } a_n - \epsilon/2 &< b_i \leq a_n + \epsilon/2\\
\implies a_n - \epsilon/2 &\leq \equalto{\mathrm{inf}	\{b_i: i \geq N\}}{a} \leq a_n + \epsilon/2\\
\iff |a-a_n| &\leq \epsilon/2 < \epsilon \quad \forall n \geq N.\end{aligned}\]
(We used: $S \subseteq \RR$ is bounded satisfying $x < M~\forall x \in S$. Then $\mathrm{sup}S \leq M$.)
\end{proof}

\pagebreak
\lecturemarker{7}{5 Oct}

\begin{example}
Prove that if $\left|a_{n+1}/a_n\right|\to L$, $L < 1$, then $a_n \to 0$	

\emph{Idea:} $a_N \approx c.L^n$ for $n >> 0$, $L<1 \implies a_n \to 0$. 

To turn this in to a proof, we want $\left|a_{n+1}/a_n\right|$ to be less than $\alpha <1$! We can't take $\alpha = L$! We can take $\alpha = L + \epsilon$ (because $\left|a_{n+1}/a_n\right|$ is \emph{not} equal to $L$; it just tends to it). So we need $L + \epsilon < 1$, so take $\epsilon = \frac{1-L}{2}$.

\begin{proof}
Fix $\epsilon = \frac{1-L}{2} > 0$ (because $L < 1$). $\exists N \iN$ such that $\forall n \geq N$
\[\left|\frac{a_{n+1}}{a_n} - L\right| < \epsilon \implies \left|\frac{a_{n+1}}{a_n}\right| < L + \epsilon = L + \frac{1-L}{2} = \frac{1+L}{2} < 1\]
So inductively we find that
\[|a_{N+k}| \leq \frac{1+L}{2} |a_{N+k-1}| \leq \left(\frac{1+L}{2}\right)^2 |a_{N+k-2}| \leq \dots \leq \left(\frac{1+L}{2}\right)^k |a_{N}| ~(*)\]
[Ex sheet: $\alpha^k \to 0$ as $k \to \infty$ if $|\alpha| < 1$]

Applying this to $\alpha = \frac{1+L}{2} < 1$. $\exists M > 0$ s.t. $\forall m \geq M$

\[\left(\frac{1+L}{2}\right)^M < \frac{\epsilon}{1 + |a_N|}\]
(as before we add $1$ in denominator in case $|a_N| = 0$)

So by $(*)$ we have $|a_{N+m}| < \dfrac{\epsilon|a_N|}{1 + |a_N|} < \epsilon~\forall m \geq M$. Rewriting this: 
\[\forall n \geq N+M,~ |a_n| < \epsilon\qedhere\]
\end{proof}
\end{example}~

\subsektion{Subsequences}

\begin{definition}
A \emph{subsequence} of $(a_n)$ is a new sequence $b_i = a_{n(i)}~ \forall i \iN$ where $n(1) < n(2) < \dots < n(i) < \dots ~\forall i \implies n(i) \geq i$ (Ex: prove this by induction)

[Formally $n(i)$ is a function $\NN \to \NN$ with $i \mapsto n(i)$ which is strictly monotonically increasing.] ``Just go down the sequence faster, missing some terms out''
\end{definition}~

\begin{example}
$a_n = (-1)^n$ has subsequences:
\begin{itemize}
	\item $b_n = a_{2n}$, so $b_n = 1~\forall n \implies b_n \to 1$
	\item $c_n = a_{2n+1}$, so $c_n = -1~\forall n \implies c_n \to -1$
	\item $d_n = a_{3n}$, so $d_n = (-1)^n (=a_n!)$ doesn't converge. 
	\item $e_n = a_{n+17}$, so $e_n = (-1)^{n+1} = -a_n$ doesn't converge.
\end{itemize}

\end{example}


Next we work up to 


\begin{theorem}[Bolzano-Weierstrass]
If $(a_n)$ is a \emph{bounded} sequence of real numbers then it has a \emph{convergent subsequence}.
\end{theorem}


\begin{proof}[Cheap proof]

Use ``peak points'' of $(a_n)$
\begin{center}
\begin{tikzpicture}
\begin{axis}[
 axis line style={red},
axis lines=middle,
     x label style={at={(axis description cs:1.05,0)}},
     y label style={at={(axis description cs:-0.05,1.1)}},
    xlabel={\small $n$},
    ylabel={$a_n$},
  ymin = 0,
  ymax = 8,
  xmin = 1,
  xmax = 12,
    ytick = 0,
   xtick = 0,
  width=10cm,height=6cm]
\addplot[color=red,mark=x] coordinates {
		(2,1)
		(3,2)
		(4,7)
		(5,3.4)
		(6,4)
		(7,2)
		(8,5.3)
		(9,3.5)
		(10,2.8)
	};
          \draw[->] (axis cs:4,7) -- (axis cs:12,7);
          \draw[->] (axis cs:8,5.3) -- (axis cs:12,5.3);
        \node at (axis cs: 4,7.4) {\small peak};
        \node at (axis cs: 8,5.7) {\small peak};
  \end{axis}
\end{tikzpicture}
\end{center}

We say that $a_j$ is a \emph{peak point} iff $a_k < a_j ~\forall k > j$.
Either
\begin{enumerate}
	\item $(a_n)$ has a finite no. of peak points
	\item $(a_n)$ has an infinite no. of peak points
\end{enumerate}

\textbf{Case (i):} Pick $n(1) \geq \mathrm{max}(j_1,\dots,j_k)$ where $a_{j1},\dots,a_{jk}$ are the finite no. of peak points. 

``Go beyond the (finitely many) peak points''.

$a_{n(1)}$ is not a peak point $\implies \exists n(2) > n(1)$ s.t. $a_{n(2)} \geq a_{n(1)}$. 

Similarly $a_{n(2)}$ not a peak point $\implies \exists n(3) > n(2)$ s.t. $a_{n(3)} \geq a_{n(2)}$.

Recursively no peak pints beyond $n(1) \implies$ we get $n(i) > n(i-1) > \dots > n(1)$ s.t. $a_{n(i)} \geq a_{n(i-1)}~\forall i$.

\begin{center}
\begin{tikzpicture}
\begin{axis}[
 axis line style={red},
axis lines=middle,
     x label style={at={(axis description cs:1.05,0)}},
     y label style={at={(axis description cs:-0.05,1.1)}},
    xlabel={\small $n$},
    ylabel={$a_n$},
  ymin = 0,
  ymax = 8,
  xmin = 1,
  xmax = 13,
     ytick = 0,
   xtick = {4,8,11},
   xticklabels = {$n(1)$,$n(2)$,$n(3)$},
  width=10cm,height=7cm]
\addplot[color=red,mark=x] coordinates {
		(4,2.2)
		(5,2)
		(6,1.2)
		(7,3)
		(8,4.5)
		(9,2.4)
		(10,3.6)
		(11,6)
		(12,5.5)
	};
	\addplot[only marks, color=red,mark=o] coordinates {
		(4,2.2)
		(8,4.5)
		(11,6)
	};
        \draw (axis cs:4,6) -- (axis cs:4,-0.1);
  \end{axis}
\end{tikzpicture}
\end{center}

i.e. $a_{n(i)}$ is a monotonically increasing subsequence of $a_n$. $(a_n)_{n\geq 1}$ bounded $\implies (a_{n(i)})_{i\geq 1}$ is bounded $\implies a_{n(i)}$ is convergent (to $\mathrm{sup}\{a_{n(i)} : i \iN\}$.


\textbf{Case (ii):} $\exists$ infinitely many peak points. Call these peak points $a_{n(1)}, a_{n(2)},\dots$ where $n(1) > n(2) > \dots$

\begin{center}
\begin{tikzpicture}
\begin{axis}[
 axis line style={red},
axis lines=middle,
     x label style={at={(axis description cs:1.05,0)}},
     y label style={at={(axis description cs:-0.05,1.1)}},
    xlabel={\small $n$},
    ylabel={$a_n$},
  ymin = 0,
  ymax = 8,
  xmin = 1,
  xmax = 13,
     ytick = 0,
   xtick = {4,8,11},
   xticklabels = {$n(1)$,$n(2)$,$n(3)$},
  width=10cm,height=6cm]
\addplot[color=red,mark=x] coordinates {
		(12,1)
		(11,2)
		(10,1.2)
		(9,3)
		(8,4.5)
		(7,2.4)
		(6,3.6)
		(5,3.2)
		(4,6)
	};
	\addplot[only marks, color=red,mark=o] coordinates {
		(11,2)
		(8,4.5)
		(4,6)
	};
        \draw (axis cs:4,6) -- (axis cs:4,-0.1);
  \end{axis}
\end{tikzpicture}
\end{center}

$a_{n(i+1)} \leq a_{n(i)}$ because $n(i+1) > n(i)$ and $a_{n(i)}$ is a peak point $\implies (a_{n(i)})_{i\geq 1}$ is monotonically decreasing and bounded $\implies$ convergent (to $\mathrm{inf}\{a_{n(i)} : i \iN\}$.
\end{proof}




\begin{proposition}\label{prop1}\lecturemarker{8}{5 Oct}
If $a_n \to a$ as $n\to \infty$ then any subsequence $a_{n(i)} \to a$ as $i \to \infty$	
\end{proposition}

\begin{proof}
\[\forall \epsilon >0,~\exists N \iN \text{ s.t. } \forall n \geq N,~|a_n - a| < \epsilon ~(*)\]	
But $\forall i \geq N$, then $n(i) \geq i \geq N \implies$ by $(*),~|a_{n(i)} - a| < \epsilon$.
\end{proof}

This gives us another proof that $(-1)^n$ is not convergent, because if $(-1)^n \to a$, then by Prop \ref{prop1}, $(-1)^{2n} \to a$ and $(-1)^{2n+1} \to a \implies a = 1$ and $a = 1,~\cont$\\

We also get another proof of ``Cauchy $\implies$ convergence'' using BW (Bolzano-Weierstrass). If $a_n$ is Cauchy ($\forall \epsilon >0 ~\exists N \iN$ s.t. $\forall n,m \geq N~|a_n -a_m | < \epsilon$), then $a_n$ is convergent $(\exists a$ s.t. $a_n \to a$)

\begin{proof}
	We know that $a_n$ is bounded (by $\mathrm{max}\{|a_1|,|a_2|,\dots,|a_{N-1}|,|a_N| + 1)\}$. So by BW, $\exists$ a convergent subsequence $a_{n(i)},~i \geq 1$ s.t. $a_{n(i}) \to a$ as $i \to \infty$ for some $a \iR$. 
	
	So fix $\epsilon >0$. We have:
	\begin{enumerate}
	\item $\exists N_1$ s.t. $\forall n,m \geq N_1,~|a_n - a_m| < \epsilon$
	\item $\exists N_2$ s.t. $\forall i \geq N_2,~|a_{n(i)} -a | < \epsilon$	
	\end{enumerate}

\begin{center}
\begin{tikzpicture}
\begin{axis}[
 axis line style={red},
axis lines=middle,
     x label style={at={(axis description cs:1.05,0.1)}},
     y label style={at={(axis description cs:-0.08,1.02)}},
    xlabel={\small $n$},
    ylabel={$a_n$},
  ymin =-0.2,
  ymax = 1.1,
  xmin = 1,
  xmax = 10,
     ytick = {0.6},
   yticklabels={$a$},
   xtick = 0,
  width=12cm,height=6cm]
   \addplot[samples at={3,5,7,9}, only marks, mark=x,  mark options={red}, mark size=3pt]{1/x+0.5};
    \addplot[samples at={2,4,6,8},only marks, mark options={red},mark=otimes,mark size=3pt]{-1/(x^2)+0.6};
       \draw[dashed] (axis cs:20,0.6) -- (axis cs:0,0.6); %a line
            \draw (axis cs:5.5,1) -- (axis cs:5.5,-0.1) node  at (axis cs:5.5,-0.15) {$N =n(\mathrm{max}\{N_1,N_2\})$}; %N line
                  \node[text width=4cm] at (axis cs:8,0.9) {\small (i) other $a_n$ ($\times$) close to subseq. ($\otimes$)};
          \node[text width=4cm] at (axis cs:8,0.35) {\small (ii) subseq. ($\otimes$) close to $a$};
            
  \end{axis}
\end{tikzpicture}
\end{center}

Set $N = n(\mathrm{max}\{N_1,N_2\}) \geq \mathrm{max}\{N_1,N_2\} \geq N_1$. Then $\forall n \geq N$ we have
\[\begin{aligned}|a_n - a| &= |(a_n - a_N) + (a_N - a)| \\
&\leq |a_n - a_N| + |a_N - a|\\
 &< \epsilon + \epsilon = 2\epsilon	
\end{aligned}\]\end{proof}\vspace*{5pt}


\textbf{Aside:} Fix $c >0$. Then $a_n \to a$ iff 
\[\boxed{\forall \epsilon >0,~\exists N_{\epsilon} \iN \text{ s.t. } n \geq N_\epsilon \implies |a_n - a| < c\epsilon (*)}\]
Ex: Show $\implies$

\begin{proof}[Proof $\impliedby$] Fix $\epsilon >0$. Set $e' = \epsilon/c >0$. Then $(*) \implies $
\[\exists N_\epsilon \iN \text{ s.t. } n \geq N_\epsilon \implies |a_n - a| < c\epsilon' = \epsilon\qedhere\]
\end{proof}

\textbf{Beware!} Do not let $c$ depend on $\epsilon$ (Nor $N!$), e.g. if we let $c = \frac{1}{\epsilon}$ then $(*)$ becomes $\forall \epsilon >0,~\exists N \iN$ s.t. $\forall n \geq N,~|a_n - a| < 1$ and $a_n = \frac{1}{2} \forall n,~ a = 0$ satisfies this!\\

We can also go the other way round: Cauchy theorem $\implies$ BW. 

\begin{proof}[Proof 2 of BW]



Take a bounded sequence $(a_n)$. We want to find a convergent subsequence. 

Given $a_n \in [-R,R]~ \forall n$, repeatedly subdivide to make this interval smaller. So either
\begin{enumerate}
\item $\exists$ infinite number of $a_n$'s in $[-R,0]$
\item $\exists$ infinite number of $a_n$'s in $[0,R]$
\end{enumerate}

Pick one of these intervals with inifnite number of $a_n$'s; call it $[A_1,B_1]$, length $2R/2$. 

Now subdivide again; call $[A_2,B_2]$ one of the intervals $[A_1,\frac{A_1+B_1}{2}]$ or $[\frac{A_1+B_1}{2},B_1]$ with infinitely many $a_n$'s in it with length $2R/2^2$ etc. 

We get a sequence of intervals $[A_n,B_n]$ of length $2R/2^n$ each containing an infinite number of $a_n$s which are nested: $[A_{k+1},B_{k+1}] \subseteq [A_k,B_k]$


Now we use a \emph{diagonal argument}. Let $b_i = a_{n(i)}$ be an elements of the sequence in $[A_i,B_i]$ s.t. $n(i) > n(i-1)$. (This is possible because $\exists$ infinite no. of elements of sequence in $[A_i,B_i]$. 

 \textbf{Claim:} $b_i = a_{n(i)}$ is convergent.

Fix $\epsilon >0$. Take $N_{\epsilon} > \frac{2R}{\epsilon}$, so that $\frac{2R}{2^{N_{\epsilon}}} < \frac{2R}{N_{\epsilon}} < \epsilon$. Then $\forall i, j \geq N_{\epsilon}$ we have
\[|b_i - b_j| < \frac{2R}{2^{N_{\epsilon}}} < \epsilon\]
beacause $b_i,b_j \in [A_{N_{\epsilon}},B_{N_{\epsilon}}] \implies (b_i)$ Cauchy $\implies$ convergent.
	\end{proof}\vspace*{5pt}
	

		% Sequences
%!TEX root = M1P1.tex
\stepcounter{lecture}
\setcounter{lecture}{2}

\pagebreak

\sektion{Series}~

\begin{definition}\lecturemarker{9}{5 Oct}
An (infinite) series is an expression \[\displaystyle{\sum_{n=1}^{\infty} a_n} \text{ or } a_1  + a_2 + \dots \] where $(a_i)_{i\geq 1}$ is a sequence.
\end{definition}

\subsektion{Convergence of Series}

\begin{definition}
We say that the series $\sum a_n = A \in \mathbb{R}$ (or ``converges to $A \in \RR$'') iff the sequence of partial sums $S_n:= \sum_{i=1}^n a_i \in \RR$ converges to $A \in \RR$; $S_n \to A$ as $n \to \infty$. 
\end{definition}



\[
\begin{tikzcd}[row sep=2cm]
 &  \mbox{Sequence of partial sums $(s_n)$}\arrow[<->]{dr}{s_n = \sum_{i=1}^n a_i} \\ 
\mbox{Sequence $(a_n)$} \arrow[<->]{ur}{a_n = s_n - s_{n-1}} \arrow[<->]{rr}{\text{Equivalent Information}} && \mbox{Series $\sum a_n$}
\end{tikzcd}
\]~\\



\begin{example}
$a_n = x^n,~n \geq 0$. Consider $\sum_{n=0}^{\infty} a_n = \sum_{n=0}^{\infty} x^n$.

Define $s_n = \sum_{i=0}^n x^i = 1 + x + \dots + x^n$ then $xS_n = x + \dots + x^n + x^{n+1} \implies S_n - xS_n = 1 - x^{n+1}$
\[\implies S_n = \begin{cases}
 \frac{1-x^{n+1}}{1-x} & x \neq 1\\
 n+1 & x = 1	
 \end{cases}
\]
So for $|x| < 1$, we see that 
\[S_n = \frac{1}{1-x} - \frac{x^{n+1}}{1-x} \to \frac{1}{1-x} \text{ as } n \to \infty\]

(Question Sheet 3: proves that $r^n \to 0$ if $|n| < 1$)
\end{example}~

So we have proved that $(s_n)$ is convergent and $\sum x^n = \frac{1}{1-x} \iR$ for $|x| < 1$. 

For $|x| \geq 1$, $a_n = x^n$ does not $\to 0$ as $n \to \infty$. So $\sum a_n = \sum x^n$ is \emph{not} a real number (does not converge) by the following result:\\


\begin{theorem}
$\sum_{n=0}^\infty a_n$ is convergent $\implies a_n \to 0$	
\end{theorem}

\begin{proof}
$S_n - S_{n-1} = a_n$. If $S_n \to S$ then $S_{n-1} \to S$ (Ex). So by the algebra of limits $a_n$ is convergent and $\lim_{n\to \infty} a_n = \lim_{n\to\infty} S_n - \lim_{n\to \infty} S_{n-1} = S - S = 0$.	
\end{proof}

\begin{proof}[Proof from first principles]
Fix $\epsilon >0$. $s_n \to s$, so 
\[\exists N\iN \text{ s.t. } \forall n \geq N,~|s_n - s| < \epsilon\]
\[\begin{aligned}
\implies |a_n| &= |s_n - s_{n-1}| \\
&\leq |s_n - s| + |s_{n-1} - s| \\ 
&< \epsilon + \epsilon \text{, for } n-1 \geq N.	
\end{aligned}
\]
So $\forall n \geq N+1$, $|a_n| < 2\epsilon$. 
\end{proof}\vspace*{5pt}

\begin{remark}
Converse is \emph{not} true. E.g. $a_n = \frac{1}{n} \to 0$, but $\sum \frac{1}{n}$ is \emph{not} convergent. 	
\end{remark}~

\begin{example}
$\displaystyle{\sum_{n=1}^{\infty} \frac{1}{n^2}}$ is convergent\footnote{Famously to $\pi^2/6$ - see \href{https://en.wikipedia.org/wiki/Basel_problem}{Basel Problem}.} 	

\begin{proof}(Trick) First do $\sum_{n=1}^\infty \frac{1}{n(n+1)}$ and use $\frac{1}{n(n+1)} = \frac{1}{n} - \frac{1}{n+1}$

\[\begin{aligned}
S_n &= \sum_{i=1}^n \frac{1}{i} - \frac{1}{i+1} \\
&= \textstyle{(1-\frac{1}{2}) + (\frac{1}{2} - \frac{1}{3}) + \dots + (\frac{1}{n} - \frac{1}{n+1})}\\
&= 1 - \frac{1}{n+1} \to 1 \text{ as } n \to \infty	
\end{aligned}
\]

$\implies \sum_{n=1}^\infty \frac{1}{n(n+1)}$ is convergent to $1$.

Ao now compare the partial sums $\sigma_n$ of $\sum \frac{1}{n^2}$ to those of $\sum \frac{1}{n(n+1)} = 1$

\[\begin{aligned}
\sigma_n = \sum_{i=1}^n \frac{1}{i^2} &= 1 + \sum_{j=1}^{n-1} \frac{1}{(j+1)^2} \\
&\leq 1 + \sum_{j=1}^{n-1}\frac{1}{j(j+1)}\\
&= 1 + s_{n-1}	
\end{aligned}
\]

$s_{n-1}$ is a bounded (by $1$) monotonically increasing sequence (because $\frac{1}{n(n+1)} >0$), convergent to $1$. So $s_{n-1} < 1~\forall n \implies \sigma_n < 2 \implies$ bounded above monotonic increasing sequence $\implies \sigma_n$ is convergent $\implies \sum \frac{1}{n^2}$ is convergent.	
\end{proof}
\end{example}~


Similarly $\sum \frac{1}{n^k}$ is convergent for $k \geq 2$ because $\frac{1}{n^k} \leq \frac{1}{n^2}$. In fact $\zeta(k) = \sum \frac{1}{n^k}$ is convergent for $k \in (1,\infty)$... See later!\\


\begin{theorem}[Algebra of Limits for Sequences]

If $\sum a_n = A \iR$ and $\sum b_n = B \iR$, then $\sum (\lambda a_n + \mu b_n) = \lambda A + \mu B \iR$. 
\end{theorem}

Put differently, if $\sum a_n$, $\sum b_n$ converge, then so does $\sum (\lambda a_n + \mu b_n)$ and it equals $\lambda \sum a_n + \mu \sum b_n$. 


\begin{proof}
Partial sums (to $n$ terms) of $\sum (\lambda a_n + \mu b_n)$ is 
\[\sum_{i=1}^{n} (\lambda a_i + \mu b_i) = \lambda \sum_{i=1}^{n} \lambda a_i + \sum_{i=1}^{n} \mu b_i \to \lambda \sum_{i=1}^{\infty} a_n + \mu \sum_{i=1}^{\infty} \mu b_n \]
as $n \to \infty$ by the algebra of limits for sequences. So the partial sums converge.
\end{proof}~



\lecturemarker{10}{5 Oct}
\begin{theorem}[Comparison Test]
If $0 \leq a_n \leq b_n$ and $\sum_{n=1}^{\infty} b_n$ converges, then $\sum_{n=1}^{\infty} a_n$ converges.	 (and $0 \leq \sum_{n=1}^{\infty}a_n \leq \sum_{n=1}^{\infty}b_n$)
\end{theorem}

\begin{proof}
Call the partial sums $A_n$, $B_n$ respectively. Then
\[0 \leq A_n \leq B_n \leq \sum_{i=1}^{\infty} b_i = \lim_{n\to \infty}B_n\]	

So $A_n$ is bounded and monotonically increasing $\implies$ convergent.
 
(Question Sheet 3 shows that if $A_n \leq B_n$ and $A_n \to A$, $B_n \to B$, then $A \leq B$)
\end{proof}~


\begin{proposition}
Suppose $a_n \geq 0 ~\forall n$. Then $\sum_{n=1}^{\infty} a_n$ converges iff $S_N = \sum_{n=1}^{N}a_N$ is bounded above and $\sum_{n=1}^{\infty} a_n$ diverges to $\infty$ (i.e. $S_n \to +\infty$ as $N \to \infty$) iff $S_N = \sum_{n=1}^{N} a_n$ is an unbounded sequence. 
\end{proposition}

\begin{proof}
$a_n \geq 0 \iff (S_n)$ is monotonic increasing. So $(S_n)$ bounded $\iff$ convergent.

$S_N$ unbounded $\iff \forall R>0,~ \exists N \in \NN$ such that $\forall n \geq N,~ S_n > R \iff S_n \to +\infty$. 
\end{proof}~

Ex: (Converse of Comparison Test)
If $0 \leq a_n \leq b_n$ then $\sum a_n$ diverges to $\infty\implies \sum b_n$ diverges to $\infty$\\


\begin{example}
$\sum_{n=1}^{\infty} \frac{1}{n^{\alpha}},~ \alpha > 1$ is convergent.
\begin{proof}
(Trick!) Arrange the partial sum as follows:
\[\begin{aligned}
1 + \frac{1}{2^\alpha} + \frac{1}{3^\alpha} + \dots  = 1 + \left(\frac{1}{2^\alpha} + \frac{1}{3^\alpha}\right) &+ \left(\frac{1}{4^\alpha} + \dots +\frac{1}{7^\alpha}\right)  \\ 
&+ \left(\frac{1}{8^\alpha} + \dots + \frac{1}{15^\alpha}\right) \\
&+ \left(\frac{1}{16^\alpha} + \dots + \frac{1}{31^\alpha}\right) \\
&+ \dots  \end{aligned}\]
Note that the $k$th bracketed term:
\[\left(\frac{1}{(2^k)^\alpha} + \dots +\frac{1}{(2^{k+1}-1)^\alpha}\right ) \leq \frac{1}{2^{k\alpha}} + \dots + \frac{1}{2^{k\alpha}} = \frac{2^k}{2^{k\alpha}} = \frac{1}{2^{k(\alpha-1)}}\]

So any partial sum is less than some finite sum of these bracketed terms, i.e. for some sufficiently large $N$: \[ S_N < \sum_{k=0}^{N} \frac{1}{2^{k(\alpha -1)}} = \frac{1-\frac{1}{2^{(N+1)(\alpha -1)}}}{1-\frac{1}{2^{(\alpha-1)}}} \leq \frac{1}{1-\frac{1}{2^{\alpha-1}}}\] because $\alpha >1$, so $\left|\frac{1}{2^{\alpha-1}}\right| < 1$, so denominator $>0$. 

So partial sums are bounded above $\implies$ convergent. 
\end{proof}	
\end{example}~

\begin{definition}
Say that the series $\sum_{n=1}^{\infty}a_n$ is \emph{absolutely convergent} if and only if the series $\sum_{n=1}^{\infty} |a_n|$ is convergent	
\end{definition}~

\begin{example}
$\sum_{n=1}^{\infty} \frac{(-1)^{n+1}}{n}$ is \emph{not} absolutely convergent, but it is convergent.


\textit{Rough Working.} $1 - \frac{1}{2} + \frac{1}{3} - \frac{1}{4} + \frac{1}{5} - \frac{1}{6} + \dots = (1-\frac{1}{2}) + (\frac{1}{3} - \frac{1}{4}) + (\frac{1}{5} -\frac{1}{6}) + \dots$, the $k$th bracket $\frac{1}{2k-1} - \frac{1}{2k} = \frac{1}{2k(2k-1)}$. This is positive and $\leq \frac{1}{2k(2k-2)} = \frac{1/4}{k(k-1)}$, seen earlier sum of these is convergent.

So cancellation between consecutive terms is enough to make series converge by comparison with $\sum \frac{1}{k(k-1)}$.

\begin{proof}
Fix $\epsilon >0.$ Then use 2 things\begin{enumerate}
\item[(1)] $\sum \frac{1}{2k(2k-1)}$	is convergent
\item[(2)] $\frac{(-1)^{n+1}}{n}\to 0$
\end{enumerate}
By (1) $\exists N_1$ such that $\forall n \geq N_1,~ \sum_{n=1}^{\infty} \frac{1}{k(k-1)} < \epsilon$

By (2) $\exists N_2$ such that $\forall n \geq N_2,~ \left|\frac{(-1)^{n+1}}{n}\right| < \epsilon$

Set $N =$ max$(N_1,N_2)$. Then $\forall n \geq N$, we have:
\[S_n = \left(1-\frac{1}{2}\right) + \left(\frac{1}{3} - \frac{1}{4}\right) + \dots \left(\frac{1}{2j-1} - \frac{1}{2j} \right) + \delta = \sum_{k=1}^{j} \frac{1}{2k(2k-1)} + \delta\] 
where $\delta = \begin{cases}
 	\frac{(-1)^{n+1}}{n} & \text{ if } n \text{ is odd.}\\
 	0 & \text{ if } n \text{ is even.}
 \end{cases}
$ $~~\left(j = \lfloor \frac{n}{2} \rfloor\right)$   $j = \begin{cases}
 	\frac{n-1}{2} & \text{ if } n \text{ is odd.}\\
 	\frac{n}{2} & \text{ if } n \text{ is even.}
 \end{cases}
$
\[\implies S_n = \sum_{k=1}^{\infty} \frac{1}{2k(2k-1)} - \sum_{k=\lfloor \frac{n}{2} \rfloor + 1}^{\infty} \frac{1}{2k(2k-1)} + \delta\]
\[\text{So } \left|S_n - \sum_{k=1}^{\infty} \frac{1}{2k(2k-1)} \right| \leq \sum_{k=\lfloor \frac{n}{2} \rfloor + 1}^{\infty} \frac{1}{2k(2k-1)} + \frac{1}{n} < \epsilon + \epsilon\] for all $n \geq 2N$ (so that $\lfloor \frac{n}{2} \rfloor + 1 >N$) 
\end{proof}
\end{example}\vspace*{10pt}

\lecturemarker{11}{5 Oct}
\begin{theorem}
	If $(a_n)$ is absolutely convergent, then it is convergent.
\end{theorem}

\begin{proof}
Let $S_n = \sum_{i=1}^{n} |a_i|$, $\sigma+n = \sum_{i=1}^n a_i$ be the partial sums.

We're assuming that $S_n$ converges. Therefore $S_n$ is Cauchy: \[ \forall \epsilon >0~ \exists N_{\epsilon}\text{ such that }n > m \geq N_{\epsilon} \implies |S_n - S_m| < \epsilon \iff |a_{m+1} + \dots + |a_n| < \epsilon\]

i.e. the terms in the tail of the series contribute little to the sum 

$\implies |a_{m+1} + \dots + a_n| < \epsilon$ by the triangle inequality $\implies |\sigma_n - \sigma_m| < \epsilon \implies (\sigma_n)$ is Cauchy $\implies \sum a_i$ is convergent.
\end{proof}~

\begin{example}
$\sum_{n=1}^{\infty} z_n$ is convergent for $|z| < 1$, divergent for $|z| \geq 1$
\begin{proof}
$\sum_{n=1}^{\infty} z_n$ is absolutely convergent because we showed that $\sum_{n=1}^{\infty} |z|^n$ converges to $\frac{1}{1 - |z|}$ for $|z| < 1$	

For $|z| \geq 1$, the individual terms $z^n$ have $|z^n| \geq 1$, so $z^n \not\to 0$, so $\sum z^n$ divergent.
\end{proof}
\end{example}


\subsektion{$*$Re-arrangement of Series$*$}
\emph{This section was non-examinable in 2015}

\textbf{Beware.} Do not rearrange series and sum them in a different order unless you can prove the result is the same.

\begin{example}
$\sum (-1)^{n+1} = 1 - 1 + 1 - 1 + \dots$

either this $``='' (1-1) + (1-1) + \dots = 0$

or this $``='' 1 - (1-1) + (1-1) + \dots = 1$	
\end{example}

A better (convergent) example\\

\begin{example}
$a_n:= 1 - \frac{1}{2} + \frac{1}{3} - \frac{1}{4} + \dots 	= \log 2$

(See later for proof of result, it's the series for $\log(1+x) = x - \frac{x^2}{2} + \dots$ putting $n=1$, which is on our radius of convergence!)

Reorder the sum as follows:
\[\begin{array}{cccccccc}
	1 & ~ & +\frac{1}{3} & ~ & +\frac{1}{5} & ~ & +\frac{1}{7} & \dots \\
	& -\frac{1}{2} & ~ & -\frac{1}{4} & ~ & -\frac{1}{6} & ~ & \dots \\
	&&&&&&&\\
= 	1 & ~ & +\frac{1}{3} & ~ & +\frac{1}{5} & ~ & +\frac{1}{7} & \dots \\
	-\frac{1}{2} [& 1 & ~ & +\frac{1}{2} & ~ & +\frac{1}{3} & ~ & \dots ]\\

\end{array}\]
Terms with even denominator appear only in bottom row ($\times -\frac{1}{2}$)

Terms with odd denominator appear in the top row ($\times 1$) + bottom row $\times -\frac{1}{2} \implies (\times \frac{1}{2})$ in total. 

So $a = \frac{1}{2}[1-\frac{1}{2} + \frac{1}{3} - \frac{1}{4} + \frac{1}{5} -\frac{1}{6} + \dots] \implies a = a/2,~\cont$ (But clearly $a \geq \frac{1}{2} > 0$)

\end{example}

This happened because when I reordered I went along the bottom row twice as fast as I went along the top row. Since the top and bottom row diverges to $\infty$, I'm computing $\infty - \infty$, and originally I did this like $(a+n) - n$ as $n \to \infty$. Now I'm doing it like $(a + n) - (n + \frac{a}{2})$ as $n \to \infty$. 


In fact I can rearrange the sum to converge to anything I like.\\

\begin{example}
Rearrange $a_n = \dfrac{(-1)^{n+1}}{n} \to 42$. 

We reorder the sum as follows
\begin{enumerate}
\item Take only off terms $a_{2n+1} > 0$ until their sum is $>42$. We can do this as $1 + \frac{1}{3} + \dots$ diverges to $\infty$!
\item Now take only even terms $a_{2n} < 0$ until sum gets $<42$
\item Repeat (i) and (ii) to fade.
\end{enumerate}


We can do each step because $\sum a_{2n+1}$ diverges to $\infty$ and $\sum a_{2n}\to -\infty$. We use all the terms eventually (so this is really a reordering of the whole sum)

Why? If not then we must eventually only take terms of one type (w.l.o.g. the even -ve terms) but these sum to $-\infty,~\cont$. At point they reach $<42$ we switch back to odd +ve terms.

Finally proof that the reordered sum converges to $42$ 
\[a_n \to 0 \text{ so }\forall \epsilon >0,~\exists N \iN \text{ s.t. } n \geq N \implies |a_n| < \epsilon ~(*)\]

So now we go to a point in the reordering where we have used all $a_i$ up to $N$ and then further to the point where the partial sum crosses $42$. At this point, $(*)$ holds, so I'm within $\epsilon$ of $42$. from this point on the sum is always within $\epsilon$ of $42$ by design and by $(*)$. 

\[\implies |s_n - 42| < \epsilon \text{ from this point on}\qed\]
\end{example}


More generally \lecturemarker{12}{9 Feb}
 if $(a_n)$ is a sequence whose terms tend to zero, $a_n \to 0$ and such that: \begin{itemize}
 \item $\displaystyle{\sum_{\substack{n \text{ s.t.} \\a_n \geq 0}} a_n}$ diverges ($\to \infty$)
 \item $\displaystyle{\sum_{\substack{n \text{ s.t.}\\ a_n < 0}}  a_n}$ diverges ($\to -\infty$)	
 \end{itemize}
 then I can rearrange the series $\sum a_n$ (1) to make it converge to \emph{any} number I like $\iR$ or (2) to  make it diverge to $\infty$ or (3) to $-\infty$. 
 
 For (1), the Algorithm is same as for $\sum \frac{(-1)^n}{n}$	
 \begin{enumerate}
 \item Pick +ve terms until partial sums are $>$ my fixed real number, $a$
 \item Now pick -ve terms until partial sum is $< a$
 \item Go back to (i) and repeat.
 \end{enumerate}
 
 If however $a_n \to 0$ and 
 \begin{itemize}
 	\item $\displaystyle{\sum_{\substack{n \text{ s.t.} \\a_n \geq 0}} a_n} \to \infty$
 \item $\displaystyle{\sum_{\substack{n \text{ s.t.}\\ a_n < 0}}  a_n}$ converges
 \end{itemize}
 Then however I rearrange $\sum a_n$ it will always diverge to $+ \infty$
 
 Similarly if $a_n \to 0$ and 
 
  \begin{itemize}
 	\item $\displaystyle{\sum_{\substack{n \text{ s.t.} \\a_n \geq 0}} a_n}$ converges
 \item $\displaystyle{\sum_{\substack{n \text{ s.t.}\\ a_n < 0}}  a_n} \to -\infty$
 $\implies  \sum a_n$ diverges to $-\infty$ (however rearranged)
 \end{itemize}
 
\textbf{ Final case:} $a_n \to 0$ and 
 \begin{itemize}
 	\item $\displaystyle{\sum_{\substack{n \text{ s.t.} \\a_n \geq 0}} a_n}$ converges
 \item $\displaystyle{\sum_{\substack{n \text{ s.t.}\\ a_n < 0}}  a_n}$ converges
 \end{itemize}
This is the \emph{good case} where \emph{however} you rearrange, $\sum a_n$ is \emph{absolutely convergent} to the same limit, $\sum_{a_n \geq 0} a_n + \sum_{a_n < 0} a_n$. 
We will prove this next time.\\

\begin{remark} Rearrange partial sums only. $a+ b = b+a$ is fine. Infinite sums are tricky!	
\end{remark}


\begin{definition}[Rearrangement of a Sequence] 
	If $M: \NN \to \NN$ is a bijection (i.e. a reordering!) then define $b_m:= a_{M(m)}.$ Then $(b_m)_{m \geq 1}$ is a rearrangement of $(a_n)$.
\end{definition}

e.g. if $M(1), M(2), M(3), M(4),\dots$ is $5,1,6,2,\dots$ then $b_1,b_2,b_3,b_4,\dots$ is $a_5,a_1,a_6,a_2,\dots$.\\


\begin{theorem}
Suppose that $\sum a_n$ is absolutely convergent. Then
\begin{itemize}
\item[(1)] $\sum_{a_n \geq 0} a_n$ is convergent to $A$ (say)	
\item[(2)] $\sum_{a_n < 0} a_n$ is convergent to $B$ (say)	
\item[(3)] $\sum a_n = A+B$
\item[(4)] $\sum b_m = A+B$ where $(b_m)$ is any rearrangement of $(a_n)$
\end{itemize}
\end{theorem}

\begin{proof} \lecturemarker{13}{9 Feb}
Key Idea: $\sum |a_n|$ is convergent so has a small ``tail'', so by the triangle inequality $\sum a_n$ has an even smaller tail so should converge. 

But what to? No idea, so we use the Cauchy criterion! 

(1) $s_n = \sum_{i=1}^n a_i,~\sigma_n = \sum_{i=1}^n |a_i|$. $\sigma_n$ convergent $\implies \sigma_n$ is Cauchy. 
\[\forall \epsilon > 0~\exists N \iN \text{ s.t. } \forall n,m \geq N,~|\sigma_n - \sigma_m | < \epsilon\]
w.l.o.g. $n \geq m$, this says 
\[\sum_{i=m+1}^n |a_i| < \epsilon \implies \left|\sum_{i=m+1}^m a_i \right| < \epsilon \iff |s_n - s_m| < \epsilon\]

So $(s_n)$ is Cauchy $\implies s_n$ is convergent. 

(2) $\sum_{a_n \geq 0} a_n$ is also convergent because the partial sums are monotonic increasing, bounded above by $\sum |a_n|$. Similarly $\sum_{a_n < 0} a_n$ is decreasing, $\geq -\sum |a_n|$, so also cvgt. 

(3) Let $A = \sum_{a_n \geq 0} a_n$ and $B = \sum_{a_n < 0} a_n$. Then $\forall \epsilon >0$
\[\exists N_1 \text{ s.t. } n \geq N_1 \implies \left| \sum_{a_n \geq 0}^{\text{first } n \text{ terms}} - A\right| < \epsilon \]
\[\exists N_2 \text{ s.t. } n \geq N_2 \implies \left| \sum_{a_n < 0}^{\text{first } n \text{ terms}} - B\right| < \epsilon \]

Let $N$ be max$(I,J)$ where $I$ is the $N_i$th $a_i \geq 0$ (the $N_i$th positive term) and $a_J$ the $N_J$th -ve term. Then $\forall n \geq N$

\[\begin{aligned}\left|\sum_{i=1}^n - (A+B)\right| 
&\leq \left|\sum_{a_i \geq 0}^n a_i - A\right| + \left|\sum_{a_i < 0}^n a_i - B\right| < \epsilon + \epsilon = 2\epsilon 
\end{aligned}
\]
So $\sum_{i=1}^n \to A+B$ as $n \to \infty$.\\

(4) Finally $(b_m)$ is a rearrangement of $(a_n)$. We want to show that $\sum b_m$ converges to $A+B$ as well. 

Pick $M \iN$ such that $b_1,b_2,\dots,b_M$ contains all of $P_1,P_2,\dots,P_I$ and $N_1,N_2,\dots,N_J$ where $P_i$ is the $i$th $a_i \geq 0$ and $N_J$ is the $j$th $a_j <0$. 

[i.e. we're far enough down the rearranged series to have included all significant $a_i \geq 0$ and $a_i <0$ which sum to $<\epsilon$ by (1) and (2)]

Then $\forall m \geq M$ we have

\[\begin{aligned}
\left|\sum_{i=1}^m b_i - (A+B)\right| 
&\leq \left|\sum_{b_i \geq 0}^m b_i - A\right| + \left|\sum_{b_i < 0}^m b_i - B\right|\\
&\leq \left|\sum_{a_k \geq 0}^I a_k + \delta - A\right| + \left|\sum_{a_k < 0}^J a_k +\delta' - B\right|\\
&< \epsilon + \epsilon = 2\epsilon 
\end{aligned}
\]
(where $\delta =$ sum of $a_k \geq 0$ with $k >I$ and $\delta' =$ sum of $a_k < 0$ with $k > J$) \end{proof}

\subsektion{Tests for convergence}

\setcounter{equation}{4}
We already met the first test:
\begin{theorem}[Comparison I]
If $0 \leq a_n \leq b_n$ and $\sum_{n=1}^{\infty} b_n$ converges, then $\sum_{n=1}^{\infty} a_n$ converges.	 (and $0 \leq \sum_{n=1}^{\infty}a_n \leq \sum_{n=1}^{\infty}b_n$)
\end{theorem}

Recall proof from earlier: $s_n = \sum a_i$ is monotonic increasing and bounded above by $\sum b_i \in \RR$.\\

\setcounter{equation}{17}
\begin{theorem}[Comparison II - Sandwich Test]
	Suppose $c_m \leq a_n \leq b_n$ and $\sum c_n,~\sum b_n$ are both convergent. Then $\sum a_n$ is convergent.
\end{theorem}
\begin{proof}
Use Cauchy. $\forall \epsilon >0,~ \exists N \in \NN$ such that $\forall n,m > N$
\[\left|\sum_{i=m+1}^n b_i\right| < \epsilon,~\left|\sum_{i=m+1}^n c_i\right| < \epsilon\] since the partial sums of $b_i,~c_i$ are Cauchy. Therefore
\[-\epsilon <\sum_{i=m+1}^n c_i \leq \sum_{i=m+1}^n a_i \leq \sum_{i=m+1}^n b_i < \epsilon   \]
\[\implies \left|\sum_{i=1}^{n} a_i - \sum_{i=1}^m a_i\right| < \epsilon \implies \left(\sum_{i=1}^{n} a_i \right) \text{ is Cauchy.} \qedhere\]
\end{proof}


\begin{theorem}[Comparison III]
 \lecturemarker{14}{12 Feb}
If $\frac{a_n}{b_n}\to l \in \RR$ then $\sum b_n$ absolutely convergent $\implies \sum a_n$ is absolutely convergent.
\end{theorem}
\begin{proof}
Pick $\epsilon = 1$, then $\exists N \in \NN$ such that $\forall n \geq N$:
\[\left|\frac{a_n}{b_n} - l \right| < 1 \implies \left|\frac{a_n}{b_n}\right| < |l| + 1 \implies |a_n| < (|l| + 1)|b_n|\]
So now by the comparison test $\sum_{n \geq N} |b_n|$ convergent $\implies \sum_{n \geq N} |a_n|$ convergent $\implies \sum_{n\geq 1} |a_n|$ convergent. 	
\end{proof}

We have used the obvious fact that if $\sum_{n \geq N} c_n$ is convergent then $\sum_{n \geq 1} c_n$ is also convergent (and vice-versa). Ex: proof this!\\

\begin{theorem}[Alternating Series Test.]
Given an alternating sequence $a_n$ where $a_{2n} \geq 0$, $a_{2n+1} \leq 0~ \forall n$. Then $|a_n|$ monotonic decreasing to $0 \implies \sum a_n$ convergent
\end{theorem}

\begin{proof}
Write $a_n = (-1)^nb_n,~b_n\geq 0 ~\forall n	$. Consider the partial sums $S_n = \sum_{i=1}^{n} (-1)^nb_n$.\\

  Observe that: \begin{enumerate}
 \item[(1)]$S_i \leq S_{2n}~\forall i \geq 2n$
 \item[(2)]$S_i \geq S_{2n+1}~\forall i\geq 2n+1$
 \end{enumerate}
 Since if $i=2j$ is even, then
  \[\begin{aligned}
	S_{2j} &= S_{2n} + a_{2n+1} + \dots + a_{2j}\\ 
	&= S_{2n} + \underbrace{(-b_{2n+1} + b_{2n+2})}_{\leq 0} + \dots + \underbrace{(-b_{2j-1} + b_{2j})}_{\leq 0} \leq S_2n
\end{aligned}
\]
 
  If $i= 2j+1$ is odd, then similarly:
   \[\begin{aligned}
	S_{2j} = S_{2n} + \underbrace{(-b_{2n+1} + b_{2n+2})}_{\leq 0} + \dots + \underbrace{(-b_{2j-1} + b_{2j})}_{\leq 0} - b_{2j+1} \leq S_2n
\end{aligned}
\]
  
 So now $\forall \epsilon >0,~ \exists N \in \mathbb{N}$ such that $\forall n \geq N,~|b_n| < \epsilon$. So $\forall n,m\geq 2n$, we have: \[S_{2N+1} \leq S_n,~S_m \leq S_{2N}\] 
 \[\begin{aligned}\text{So } |S_n - S_m| &\leq |S_{2N+1} - S_{2N}|\\
 &= b_{2n+1} < \epsilon	
\end{aligned}
\]
\end{proof}~

\begin{theorem}[Ratio Test]
If $a_n$ is a sequence such that $\left|\frac{a_{n+1}}{a_n}\right| \to r < 1$, then $\sum a_n$ is absolutely convergent.	
\end{theorem}
\begin{proof}
Fix $\epsilon = \frac{1-r}{2} > 0$. Then $\exists N \in \mathbb{N}$ such that $\forall n \geq N$
\[\left|\frac{a_{n+1}}{a_n} - r\right| < \epsilon \implies |a_{n+1}| < (r + \epsilon)|a_n|\]
Set $\alpha := r + \epsilon = \frac{1 + r}{2} < 1$. 

Inductively
\[|a_{N+m}| < \alpha|a_{N+m-1}| < \dots < \alpha^m|a_N|\]	
So $\forall k \geq N$ \[|a_k| <  \alpha^{k-N}|a_N| = C\alpha^k\]

Then \[C\sum_{k=N}^{n} \alpha^k = \frac{C(\alpha^N-\alpha^n)}{1-\alpha} \to \frac{C'}{1-\alpha} \text{ as } n \to \infty \text{, since } \alpha < 1\]

So by the comparison test $\sum_{k\geq N} |a_k|$ is convergent $\implies \sum_{k\geq 1} |a_k|$ is convergent
\end{proof}

The point is that the ratio test, when it applies, says that $a_n \approx r^n$ i.e. decays exponentially. But many convergent series like $\sum \frac{1}{n^2}$ do not decay so fast.\\

\begin{example}
$a_n = \dfrac{100^n(\cos n\theta + i\sin n\theta}{n!} = \dfrac{(100e^{i\theta})^n}{n!}$

Then 
\[\left|\frac{a_{n+1}}{a_n}\right| = \frac{(100e^{i\theta})^{n+1}/(n+1)!}{(100e^{i\theta})^{n}/n!} = \frac{100}{n+1} \to 0\]
So by the ratio test, $\sum a_n$ is absolutely convergent $\implies \sum a_n$ is convergent.	
\end{example}



\begin{theorem}[Root Test] \lecturemarker{15}{16 Feb}
If $\lim_{n\to \infty} |a_n|^{1/n} = r < 1$, then $\sum a_n$ is absolutely convergent.	
\end{theorem}
\begin{proof}
Fix $\epsilon = \frac{1-r}{2} > 0$. Then $\exists N \in \mathbb{N}$ such that $\forall n \geq N$
\[\left||a_n|^{1/n} - r\right| < \epsilon \implies |a_n|^{1/n} < r + \epsilon\]
Set $\alpha := r + \epsilon = \frac{1 + r}{2} < 1$, so that $|a_n| < \alpha^n$. Then
\[ \sum_{k=1}^n\alpha^k = \frac{\alpha(1-\alpha^n)}{1-\alpha}  \to \frac{\alpha}{1-\alpha} \text{ as } n \to \infty \text{ since } \alpha < 1\]

So by the comparison test $\sum_{k\geq 1} |a_k|$ is convergent.
\end{proof}


\subsektion{Power Series}\vspace*{5pt}
\begin{theorem}[Radius of Convergence]
	Consider the series $\sum a_nz^n ~(*),~ z,a_n \iC$. 
	
	Then $\exists R \in [0,\infty]$ such that $|z| < R \implies (*)$ is aboslutely convergent, $|z| > R \implies (*)$ divergent
\end{theorem}

\begin{proof}
Define $R =$ sup $S = \{|z| : a_nz^n \to 0\}$	 or $R = \infty$ if the set is unbounded. (1) Suppose $|z| < R$. $|z|$ not an upperbound for $S \implies \exists w$ such that $|w| > |z|$ and $a_nw^n \to 0.$ Then \[|a_nz^n| = |a_nw^n|\left|\frac{z}{w}\right|^n \leq A\left|\frac{z}{w}\right|^n\] Since $\left|\frac{z}{w}\right| < 1 \implies \sum|a_nz^n|$ cvgt. Similarly $|z| >R  \implies$ $\sum |a_nz^n|$ divergent. 

(2) Suppose $|z| > R$. Then $a_nz^n \not\to 0$ as $n \to \infty \implies \sum a_nz^n$ does not converge.
\end{proof}

\begin{clicker}
What is the radius of convergence for $\sum \frac{z^n}{n}$? 

\textbf{Answer:} $R = 1$, in fact the series
\begin{enumerate}
\item $\sum z^n$
\item $\sum \frac{z^n}{n}$
\item $\sum \frac{z^n}{n^2}$	
\end{enumerate}
all have this $R$.

\begin{proof}
The ratio test gives  $\left|\dfrac{z^{n+1}}{z^n}\cdot f(n)\right|$ where $f$ is a rational function of $n$ of degree $0$. $= |zf(n)| \to |z|$ as $n \to \infty$. So convergent for $|z| < 1$ and divergent for $|z| > 1$.  	
\end{proof}

But notice different behaviours on $|z| = 1$. \begin{enumerate}
 \item Never converges on $|z| = 1$ as $z^n \not\to 0$
 \item Convergent for some $|z| = 1$ (in fact $z \neq 1$), divergent for others
 \item Also convergent $\forall z$ with $|z|  =1$ (comparison with $\sum \frac{1}{n^2}$)	
 \end{enumerate}
\end{clicker}






\subsubsektion{Products of Series}
Consider 
\[\begin{aligned}
\sum_{n=0}^{\infty} a_nz^n \sum_{n=0}^{\infty}b_nz^n &= (a_0 + a_1z + a_2z^2 + \dots) (b_0 + b_1z + b_2z^2 + \dots)\\
&``='' a_0b_0 + (a_0b_1 + a_1b_0)z + (a_0b_2 + a_1b_1 + a_2b_0)z^2 + \dots \\
&= \sum_{n=0}^{\infty} c_nz^n 	
\end{aligned}
\]
where $c_0 = a_0b_0$, $c_1 = a_0b_1 + a_1b+0, \dots c_n = a_0b_n + a_1b_{n-1} + \dots + a_nb_0$. 

So we set $c_n = \sum_{i=0}^{n} a_i b_{n-i}$ and ask when is the product $\sum a_nz^n \sum b_nz^n$ equal to $\sum c_nz^n$? We can also do this without the $z^n$'s:\\

\begin{definition}
Given series $\sum a_n,~\sum b_n$, their \emph{Cauchy Product} is the series $\sum c_n$ where $c_n := \sum_{i=0}^n a_ib_{n-i}$.	
\end{definition}


\begin{theorem}[Cauchy Product]  \lecturemarker{16}{19 Feb}
If $\sum a_n, \sum b_n$ are absolutely convergent, then $\sum c_n$ is absolutely convergent to $(\sum a_n) \cdot (\sum b_n)$	
\end{theorem}
\textit{Proof.}
See handout on blackboard. Non-examinable. \\

\begin{corollary}
If $\sum A_nz^n$ and $\sum B_nz^n$ have radius of convergence $R_A$ and $R_B$ respectively, then $\sum c_nz^n$ has radius of convergence $R_C \geq \mathrm{min}\{R_A,R_B\}$.	
\end{corollary}
\begin{proof}
By the previous theorem, for $|z| < \mathrm{min}\{R_A,R_B\} ~(*)$ we have $\sum A_nz^n$ and $\sum B_nz^n$ absolutely convergent $\implies \sum c_nz^n$ absolutely convergent to their product. 

In fact $|c_nz^n| \to 0$ so $|z| < R_c$. So by $(*)$, $R_c \geq \mathrm{min}\{R_A,R_B\}$.
\end{proof}~

\begin{example}
$\sum z^n$ has $R_A = 1$, $1-z$ has $R_B = \infty$ So their cauchy product $\sum c_nz^n$ has $R_c \geq 1$. 

\emph{Ex:} Check $c_0 = 1, c_n = 0~\forall n \geq 1$, so in fact $R_c = \infty$	.

But we only know that $\sum c_n z^n = 1 = (\sum z^n)(1-z)$ when $|z| < 1 = \mathrm{min}\{R_A,R_B\}$. 
\end{example}~

\subsubsektion{Exponential Power Series}\vspace*{5pt}

\begin{definition}[Exponential Series]
\[E(z):= \sum_{n=0}^\infty \frac{z^n}{n!},~z\iC\]

Ratio test: $|a_{n+1}/a_n| = \frac{z}{n+1} \to 0$ as $n \to \infty ~\forall z \iC\implies E(z)$ is absolutely convergent $\forall z\iC$.  
\end{definition}~

\begin{proposition}
$E(z)E(w) = E(z + w)$
\end{proposition}
\begin{proof}
By Cauchy product theorem
\[E(z)E(w) = \sum_{n=0}^\infty c_n\]
where $c_n = \sum_{i=0}^n \dfrac{z^i}{i!}\dfrac{w^{n-i}}{(n-i)!} \implies c_n = \dfrac{(z+w)^n}{n!}$.
\end{proof}

\begin{corollary}
$E(z) \neq 0$ and $\dfrac{1}{E(z)} = E(-z)$
\end{corollary}
\begin{proof}
$E(z)E(-z) = E(0) = 1$.	
\end{proof}\vspace*{5pt}

\begin{definition}
$e:= E(1) = \sum \frac{1}{n!} \in (-0,\infty)$	
\end{definition}\vspace*{5pt}

\begin{corollary}
$E(n) = e^n$ for $n \iN$	
\end{corollary}
\begin{proof}
	$E(n) = E(1 + (n-1)) = E(1)E(n-1) = \dots = (E(1))^n$.
\end{proof}

\begin{proposition}
$E(q) = e^q$ for $q \iQ$ (recall rational powers of $a \iR$ were defined in M1F)	
\end{proposition}
\begin{proof}
	Suppose $q > 0$; write $q = \frac{m}{n}$, $m,n \iN$. Then 
	\[\textstyle{E(q) = E(\underbrace{\frac{1}{n} + \dots + \frac{1}{n}}_{n \text{ times}}) = E(\frac{1}{n})^m}\]
	But \[E(\frac{1}{n})^n = E(\underbrace{\frac{1}{n} + \dots + \frac{1}{n}}_{m \text{ times}}) = E(1) = e\]
	
	\[ \implies E(\frac{1}{n}) = e^{1/n}\text{ and }E(q) = E(\frac{1}{n})^m = e^{m/n} = e^q\]
	
	If $q = \frac{-m}{n}$ then $E(q) = 1/E(m/n) = \frac{1}{e^{m/n}} = e^{-m/n} = e^q$.
\end{proof}

So we know that $E(x) = e^x ~\forall x \iQ$. Later we define $e^x~\forall x\iR$ by \emph{continuity} and we will show $E(x)$ is also continuous and so $E(x) = e^x~\forall x\iR$. \\
 
 
Some \lecturemarker{17}{20 Feb} useful properties of $E(x)$:
\begin{enumerate}
\item $x \geq 0 \implies E(x) \geq 1$ and $x > 0 \implies E(x) > 1$ (obvious from series)
\item $E(x) > 0~\forall x \iR$
\item $E(x)$ is strictly increasing for $x\iR$: $x < y \implies E(y) = E(x)E(y-x) > E(x).1$
\item $|x| < 1$ then $|E(x) - 1| < \frac{|x|}{1-|x|}$
\item $\RR \ni x \mapsto E(x)$ is a continuous bijection onto $(0,\infty)$. (proven later)
\item So we can define $\log: (0,\infty) \to \RR$ as inverse of $E$, i.e. $y = \log x$ defined by $\iff x = e^y$ with the usual log properties
\end{enumerate}~

We can also define $a^x$ for $a \in(0,\infty),x\in \RR$ by $a^x = E(x\log a)$

Ex: If $x \iQ$ this agrees with Corti's definition. 

And trig functions $\cos \theta = \Re E(i\theta)$, $\sin \theta = \Im E(i\theta)$ etc. 

Ex: $E(i\theta + i\phi) = E(i\theta)E(i\phi)$ implies what?
	


\pagebreak		% Series
\stepcounter{lecture}
\setcounter{lecture}{3}

\pagebreak

\sektion{Continuity}
\label{sub:series}

\subsektion{Continuity and Limits}

\begin{definition}
	Given a function $f: \R \to \R$, we say that $f$ is \emph{continuous at $a \in \R$} if and only if 
	\[\forall \epsilon >0, ~\exists \delta >0 \text{ such that } |x-a| < \delta \implies |f(x) - f(a)| < \epsilon \]
\end{definition}

So $\delta$ depends on $a, \epsilon$. ``Once $x$ is close to $a$, then $f(x)$ is close to $f(a)$''.

More precisely: ``However close (i.e. within $\epsilon$) I want $f(x)$ to be to $f(a)$, I can arrange it by taking $x$ close (i.e. within $\delta$) to $a$''.\\

Equivalently: $\forall \epsilon > 0,~ \exists \delta >0 \text{ such that } |f(x) - f(a)| < \epsilon ~ \forall x \text{ with } |x-a| < \delta$\\

Or: $\forall \epsilon > 0, ~ \exists \delta > 0 \text{ such that } f(a- \delta, a + \delta) \subseteq (f(a)- \epsilon, f(a) + \epsilon)$

Where $S\subseteq R$ then $f(S)$ is the set $\{f(x):x \in S\}$\\

Or: $\forall \epsilon,~\exists \delta >0 \text{ such that } f^{-1}(f(a) - \epsilon, f(a) + \epsilon) \supseteq (a - \delta, a + \delta)$

Where $f: A \to B \subset T$ then $f^{-1}(T) = \{a \in A:f(a) \in T\}$ [Don't need $f^{-1}$ to exist !!]\\


\begin{example}
\[f(x) = \begin{cases}
 0 & x \leq 0 \\
 1 & x > 0	
 \end{cases}
\]	 Then $f$ is not continuous at $x = 0$

\begin{proof}
Take $\epsilon = 1$ (or $0 < \epsilon < 1$). Then if $f$ is continuous at $x = 0$ we know that $\exists \delta > 0$ such that $|f(x) - f(0)| < 1~ \forall x \in (0 - \delta, 0 + \delta)~(*)$. In particular, take $x = \delta/2$ to find that $|1-0| < 1$ by 	$(*)$.
\end{proof}
\end{example}

``Jump discontinuity'' is another type of discontinuity, 

\vspace*{100pt}

\begin{example}
\[f(x) = \begin{cases}
 \sin(\frac{1}{x}) & x \neq 0\\
 r & x = 0	
 \end{cases}
\] Then $f$ is discontinuous at $x = 0$ (for any $r$).	

\emph{Idea of proof}: If $f$ is continuous at $x = 0$, then $f(x) \in (r-\epsilon, r + \epsilon)$ is close to $f(0) = r$ for $x \in (- \delta, \delta).$ In particular, $f(x)$ and $f(y)$ are close to each other (within $2\epsilon$). But $f(x)$ could be $+1$ and $f(y)$ could be $-1$, $\cont$.
\begin{proof}
Fix $\epsilon \in (0,1]$. If $f$ is continuous at $0$, then $\exists \delta > 0$ such that $|f(x) - f(0)| < \epsilon~\forall x \in (\delta,\delta)$. In particular, $\forall x,y \in (-\delta, \delta), |f(x) - f(y)| < 2\epsilon \leq 2$, by the triangle inequality. 

Now choose $n \in \mathbb{N}$, $n > \frac{1}{\delta}$. Then take $x = \frac{1}{(4n+1)\pi/2} \in (0,\delta)$, $y = \frac{1}{(4n+3)\pi/2} \in (0,\delta)$. Then 
\[|\sin(1/x) - \sin(1/y)| = |1 - (-1)| = 2 ~\cont\qedhere\]
\end{proof}

\end{example}~


\begin{theorem}
$f,g: \RR \to \RR$ cts at $a \in \RR \implies (f + g)$ cts at $a$.
\end{theorem}
\begin{proof}
	Fix $\epsilon >0.$ \[\exists \delta_1 >0 \text{ such that }|x-a| < \delta_1 \implies |f(x) - f(a)| < \epsilon\] and\[\exists \delta_2 >0 \text{ such that }|x-a| < \delta_1 \implies |g(x) - g(a)| < \epsilon\]
	 Set $\delta =$ min$\{\delta_1,\delta_2\}.$ Then $\forall x$ such that $|x-a| < \delta:$
	\[|(f+g)(x) - (f+g)(a)| \leq |f(x) - f(a)| + |g(x) - g(a)| < 2\epsilon\qedhere\]
\end{proof}\vspace*{5pt}

\begin{theorem}
$f,g: \RR \to \RR$ cts at $a \in \RR \implies (fg)$ cts at $a$.	
\end{theorem}
\begin{proof}
	Take $\epsilon = 1$. $\exists \delta_1 > 0$ such that $|x-a| < \delta_1 \implies g(x) < 1 + g(a)$. \\
	Fix $\epsilon >0, \exists \delta_2 >0$ such that $|x-a| < \delta_1 \implies |f(x) - f(a)| < \epsilon$ and $\exists \delta_3 > 0$ such that $|x-a| < \delta_2 \implies |g(x) - g(a)| < \epsilon$.\vspace*{5pt}\\ Set $\delta =$ min$\{\delta_1,\delta_2,\delta_3\}.$ Then $\forall x$ such that $|x_a| < \delta$: \\
	$|f(x)g(x) - f(a)g(a)| \leq |g(x)||f(x) - f(a)| + |f(a)||g(x) - g(a)| < k\epsilon$.
\end{proof}\vspace*{5pt}


\begin{theorem}
	$f:\RR \to \RR$ cts at $a \in \RR$, $g:\RR \to \RR$ cts at $f(a) \in \RR$, then $g \circ f$ cts at $a$
\end{theorem}
\begin{proof}
Fix $\epsilon >0.~ \exists \delta > 0$ s.t. $|g - f(a)| < \delta \implies |g(y) - g(f(a))| < \epsilon$.\\
Also $\exists \eta > 0$ such that $|x-a| < \eta \implies |f(x) - f(a)| < \delta$.\\

\noindent Hence $|x-a| < \eta \implies |f(x) - f(a)| < \delta \implies |g(f(x)) - g(f(a))| < \epsilon.$
\end{proof}\vspace*{5pt}

\begin{theorem}
	$f: \RR \to \RR$ is cts at $a \in \RR$ iff $\forall$ sequences $x_n \to a$, $f(x_n) \to f(a)$
\end{theorem}
\begin{proof}
If $f$ is cts at $a$, fix $\epsilon >0$. $\exists \delta > 0$ such that $|x-a| < \delta \implies |f(x) - f(a)| < \epsilon$. Now $x_n \to a$, so $\exists N \in \mathbb{N}$ such that $n \geq N \implies |x_n - a| < \delta \implies |f(x_n) - f(a)| < \epsilon$.\\

\noindent Suppose $f$ is not cts at $a\in \RR$ for contradiction. Then $\exists \epsilon >0$ such that $\forall \delta >0,~\exists x \in (a-\delta,a+\delta)$ such that $|f(x) -f(a)| \geq \epsilon.$ Set $\delta = \frac{1}{n}$. $\exists x_n \in (a - \frac{1}{n},a + \frac{1}{n})$ such that $|f(x_n) - f(a)| \geq \epsilon$. So $|x_n-a| < \frac{1}{n}~\forall n \implies x_n \to a$. But $f(x_n) \not\implies f(a)$, a contradiction.  
\end{proof}~


\begin{theorem}
$f:[a,b] \to \RR$ cts $\implies f$ is bounded.	
\end{theorem}
\begin{proof}[Proof (1.)]
Suppose for contradiction $f$ is unbounded. Then $\forall N \in \mathbb{N}$, $N$ is not an unpperbound, so $\exists x_N \in [a,b]$ such that $|f(x_n)| > N$.\\
By BW Theorem, exists cvgt subsequence, $y_i := x_{N(i)}, y_i \to y \in [a,b]$. With $|f(y_i)| = |f(x_{N(i)} > N(i) \geq i ~(*)$. Fix $\epsilon =1$, then $\exists \delta > 0$ such that $\forall x \in (y-\delta, y + \delta): |f(x) - f(y)| < 1 \implies |f(x)| < |f(y)| +1.$ Since $y_i \to y$, $\exists N$ such that $\forall n \geq N~ |y_n - y| < \delta \implies y_n \in (y-\delta, y + \delta) \implies |f(y_n)| < |f(y)| + 1.$ By $(*)$, $n \leq |f(y_n)| < |f(y)| + 1~ \forall n \geq N$, contradicting the Archimedean Axiom.
	
\end{proof}\vspace*{5pt}

\begin{proof}[Proof (2.)]
	Suppose for contradiction $f$ is unbounded. Then $\forall N \in \mathbb{N}$, $N$ is not an unpperbound, so $\exists x_N \in [a,b]$ such that $|f(x_n)| > N$.\\
By BW Theorem, exists cvgt subsequence, $y_i := x_{N(i)}, y_i \to y \in [a,b]$. With $|f(y_i)| = |f(x_{N(i)} > N(i) \geq i ~(*)$. \\ 
$f$ is cts at $y \implies f(y_i) \to f(y)$, contradicting $(*)$.  
\end{proof}\vspace*{5pt}



\subsektion{Intermediate Value Theorem}



\begin{theorem}[Intermediate Value Theorem] If $f: [a,b] \to \RR$ cts, $c \in (f(a),f(b)),$ then $\exists x \in [a,b]$ such that $f(x) = c$
	
\end{theorem}
\begin{proof}
Consider $S_c = \{y \in [a,b] : f(y) \leq c\}$. Define $x:=$ sup$S_c$ ($S_c \neq \emptyset$ since $a \in S_c$ and bounded above by $b$ so sup exists)\\
\textbf{Claim: $f(x) = c$}. \textit{Proof:} \begin{enumerate}
 \item Suppose $f(x) < c$. Take $\epsilon = c-f(x) > 0$. $f$ is cts at $x$, so $\exists \delta > 0$ such that $\forall y \in (x,x+\delta) \cap [a,b], ~|f(y) - f(x)| < \epsilon$. Hence $f(y) < f(x) + \epsilon = c$. So $y\in S_c \implies x \neq$ sup$S_c$.
 \item 	Suppose $f(x) > c$. Take $\epsilon = f(x)-c > 0$. $f$ is cts at $x$, so $\exists \delta > 0$ such that $\forall y \in (x-\delta,x) \cap [a,b], ~|f(y) - f(x)| < \epsilon$. Hence $f(y) > f(x) - \epsilon = c \implies x- \delta$ is an upperbound for $S_c$, so $x \neq$ sup$S_c$. \qedhere
 \end{enumerate}
\end{proof}\vspace*{5pt}

\subsektion{Extreme Value Theorem}\vspace*{5pt}
\begin{theorem}[Extreme Value Theorem] $f: [a,b] \to \RR$ cts $\implies f$ bounded and attains its bounds.
\end{theorem}

\begin{proof}[Proof (1.)]
	By boundedness theorem, $\exists \displaystyle{\text{sup}_{x \in [a,b]}} f(x) = s$. Suppose for contradiction $\not \exists c \in [a,b]$ such that $f(x) = s$. Then $s-f(x) > 0 ~\forall x \in [a,b],$ so $g(x) = \frac{1}{s-f(x)}: [a,b] \to \RR$ is well defined and cts. So $g(x)$ is bounded by $M > 0 \implies \frac{1}{s-f(x)} \leq M \implies f(x) \leq s - \frac{1}{M}$, so $s \neq$ sup$f(x)$, a contradiction.
\end{proof}\vspace*{5pt}
\begin{proof}[Proof (2.)]
	$\exists$ a sequence $x_n \in [a,b]$ such that $f(x) \to  \displaystyle{\text{sup}_{x \in [a,b]}} f(x) = s$. BW Theorem $\implies$ exists subsequence $y_i := x_{N(i)}$ such that $y_i \to c \in [a,b].$ $f$ is cts $\implies f(y_i) \to f(x)$. Since $f(y_i) \to s$, by uniqueness of limits, $f(c) = s$.
	\end{proof}\vspace*{5pt}


\begin{theorem}
$f:\RR \to \RR$ is bijective and cts $\implies f$ is strictly monotoinc	
\end{theorem}
\begin{proof}
Fix any interval $[a,b] \subseteq \RR$. $f$ is bijective, so $f(a) \neq f(b)$, w.l.o.g. $f(b) > f(a).$ Suppose for contradiction $\exists c \in (a,b)$ such that $f(c) \not\in (f(a),f(b)).$ w.l.o.g. take $f(c) > f(b)$. Then fix $d \in (f(b),f(c))$. By IVT: $f|_{[a,c]}, ~\exists x \in (a,c)$ such that $f(x) = d$. Also  $f|_{[c,b]}, ~\exists y \in (c,b)$ such that $f(y) = d$.\\
But $x \neq y$, a contradiction since $f$ is injective. Hence $\forall c \leq b$,$f(x) \leq f(b)$. $f$ injective $\implies f(c) < f(b)$. 
\end{proof}\vspace*{5pt}

\subsektion{Inverse Function Theorem}\vspace*{5pt}

\begin{theorem}
	$f: \RR \to \RR$ bijective and cts $\implies f^{-1}: \RR \to \RR$ cts.
\end{theorem}
\begin{proof}
By Theorem 3.8, $f$ is strictly monotonic, w.l.o.g. strictly increasing. Fix $y_0 \in \RR$. Let $f^{-1}(y_0) = x_0 \in \RR$. Fix $\epsilon > 0$.\\ Set $\delta =$ min$\{f(x_0 + \epsilon) - y_0, y_0 - f(x_0 - \epsilon)\}$. Then $|y -  y_0| < \delta \implies y \in (y_0 -\delta, y_0 + \delta) \subseteq (f(x_0 - \epsilon), f(x_0 + \epsilon))$. Applying $f^{-1}$ preserves order $\implies f^{-1}(y) \in (x_0 - \epsilon, x_0 + \epsilon\implies |f^{-1}(y) - f^{-1}(y_0)| < \epsilon$.  	
\end{proof}\vspace*{5pt}


\begin{theorem}
	$f:\RR^n \to \RR^m$ is cts at $\mathbf{a} = (a_1,\dots,a_n)$ if and only if $f_i:\RR^n \to \RR$ is cts at $a_i~\forall i$. (With $f = (f_1,\dots,f_m)$).
\end{theorem}
\begin{proof}
	Fix $\epsilon >0.$ Then $f$ is cts at $\mathbf{a} \implies \exists \delta >0 $ such that $|\mathbf{x} - \mathbf{a}| < \delta  \implies |f(\mathbf{x}) - f(\mathbf{a}) < \epsilon ~(*)$. Since $|f(\mathbf{x}) - f(\mathbf{a})| = \sqrt{\sum_{j=1}^m (f_j(\mathbf{x}) - f_j(\mathbf{a}))^2} \geq \sqrt{(f)i(\mathbf{x}) - f_i(\mathbf{a}))}^2 = |f_i(\mathbf{x} - f_i(\mathbf{a})|$, $(*) \implies |f_i(\mathbf{x}) - f(\mathbf{a})| < \epsilon$.\\
	
	Suppose $f_i$ cts at $a_i~\forall i$. Fix $\epsilon > 0$. Then $\exists \delta_i > 0$ such that $|\mathbf{x} - \mathbf{a}| < \delta_i \implies |f_i(\mathbf{x}) - f_i(\mathbf{a})| < \epsilon$. Set $\delta =$ min$\{(\delta_i\} >0$, so that $|\mathbf{x} - \mathbf{a}| < \delta \implies |f_i(\mathbf{x}) - f_i(\mathbf{a})| < \epsilon ~\forall i \implies |f(\mathbf{x}) - f(\mathbf{a})| = \sqrt{\sum_{i=1}^m (f_i(\mathbf{x}) - f_i(\mathbf{a}))^2} \leq \sqrt{\sum_{i=1}^m \epsilon^2} = \sqrt{m}.\epsilon$
\end{proof}



		% Continuity
\stepcounter{lecture}
\setcounter{lecture}{4}

\pagebreak

\sektion{Differentiation}
\label{sub:differentiation}

\subsektion{Differentiability}\vspace*{5pt}

\begin{definition}
$f$ is \emph{differentiable} at $a$ iff $\lim_{x\to a} \left|\frac{f(x)-f(a)}{x-a}-f'(a)\right|$ exists, i.e.
\[\forall \epsilon > 0,~\exists \delta > 0\text{ such that } 0<|x-a| < \delta \implies \left|\frac{f(x)-f(a)}{x-a}-f'(a)\right| < \epsilon.\]	
\end{definition}




\begin{theorem}	
$f$ differentiable at $a\in \RR \implies $ cts at $a$.
\end{theorem}
\begin{proof}[Proof (1.)] If $f$ is differentiable at $a$ then
	\[\begin{aligned}\forall \epsilon >0~ \exists \delta  > 0 \text{ such that } 0 < |x-a| < \delta &\implies \left|\frac{f(x) - f(a)}{x-a} - f'(a)\right| < \epsilon \\ 
	&\implies |f(x) - f(a)| < |x-a|(|f'(a)| + \epsilon).	
\end{aligned}
\] 
	Fix $\epsilon > 0$, set $\delta = \epsilon$. Then \[0 < |x-a| < \delta \implies  |f(x) - f(a)| < \epsilon(|f'(a)| + \epsilon) = k\epsilon\] (also true for $x = a \implies |f(x) - f(a)| = 0$.)
\end{proof}
\begin{proof}[Proof (2.)]
Note that $f(x) = f(a) + (x-a)\frac{f(x) -f(a)}{x-a}$, $x \neq a$. Taking  $\lim_{x\to a}$  \[\lim_{x\to a}f(x) = f(a) + 0.f'(a) \implies f \text{ cts at } a\qedhere\]\end{proof}\vspace*{5pt}

\subsektion{Rolle's Theorem}\vspace*{5pt}


\begin{theorem}[Rolle's Theorem]
	$f:[a,b] \to \RR$ cts on $[a,b]$, differentiable on $(a,b)$ such that $f(a) = f(b)$. Then $\exists c \in (a,b)$ such that $f'(c) = 0$.
\end{theorem}
\begin{proof}~
\begin{enumerate}
\item[Case 1.] $f$ is constant on $[a,b]$. Then set $c = \frac{a+b}{2},$ so $f'(c) = \lim_{x\to c}\frac{f(x) - f(c)}{x-c} = 0$.
\item[Case 2.] $f$ takes values $< f(a)$. Then replace $f$ by $-f$ and consider Case 3.
\item[Case 3.] $f$ takes values $> f(a)$. Therefore sup $\{f(x): x \in [a,b]\} > f(a)$ by EVT is realised by some $c \in (a,b)$. Now $f'(c) = \lim_{x\to c} \frac{f(x) - f(c)}{x-c}$. Consider\\ 

 \[x > c,~f(x) \leq f(c) \implies \frac{f(x) - f(c)}{x-c} \leq 0 \implies \lim_{x \to c^{+}} \frac{f(x) - f(c)}{x-c} \leq 0\]
 \[x < c,~f(x) \leq f(c) \implies \frac{f(x) - f(c)}{x-c} \geq 0 \implies \lim_{x \to c^{-}} \frac{f(x) - f(c)}{x-c} \geq 0 \] 
 Hence  $\dfrac{f(x) - f(c)}{x-c} = 0$.\qedhere 
\end{enumerate}
	
\end{proof}\vspace*{5pt}

\subsektion{Mean Value Theorem}\vspace*{5pt}


\begin{theorem}[Mean Value Theorem]
If $f:[a,b] \to \RR$ is cts on $[a,b]$ and differentiable on $(a,b)$, then $\exists c \in (a,b)$ such that $f'(c) = \dfrac{f(b) - f(a)}{b-a}$.
\end{theorem}
\begin{proof}
Let $g(x) = f(x) - \dfrac{f(b) - f(a)}{b-a}(x-a)$, which is cts on $[a,b]$ and diff'ble on $(a,b)$. $g(a) = f(a) = g(b)$. By Rolle's Theorem
\[\exists c \in (a,b)\text{ such that }g'(c) = 0 \implies g'(c) = f'(c) - \dfrac{f(b) - f(a)}{b-a}.\qedhere\]	
\end{proof}~

\subsektion{Rules for Differentiation}\vspace*{5pt}

\begin{theorem}[Product Rule] $f,g : \RR \to \RR$ differentiable at $a \in \RR$. Then $fg$ is differentiable at $a$ with $(fg)'(a) = f'(a)g(a) + f(a)g'(a)$
\end{theorem}
\begin{proof}
\[\begin{aligned}
	\frac{f(x)g(x) - f(a)g(a)}{x-a} &= \frac{(f(x)-f(a))g(x) + (g(x) - g(a))f(a)}{x-a} \\ 
	&= g(x) \frac{f(x)-f(a)}{x-a} + f(a)\frac{g(x)-g(a)}{x-a}
\end{aligned}
\]
Taking $\lim_{x\to a} \implies (fg)'(a) = g(a)f'(a) + f(a)g'(a)$ by cty of $g$ and algebra of limits.
\end{proof}\vspace*{5pt}

\begin{theorem}[Chain Rule] $g: \RR \to \RR$ diff'ble at $a \in \RR$, $f: \RR \to \RR$ diff'ble at $g(a) \in \RR$, then $f \circ g$ diff'ble at $a$ with $(f\circ g)'(a) =f'(g(a))g'(a)$
\end{theorem}
\begin{proof}
Define $F(g) = \begin{cases}
 	\frac{f(y)-f(b)}{g-b} ~y \neq b\\
 	f'(g) ~~ y = b
 \end{cases}$
 with $b = g(a)$. \vspace*{5pt}\\$f$ is diff'ble at $b \implies \lim_{y \to b} F(y) \to f'(b) = F(b)$. Hence $F$ is cts at $b$. $g$ is diff'ble at $a \implies$ cts at $a \implies F \circ g$ is cts at $a \implies F(g(x)) \to F(g(a)) = f'(b)$ as $x \to a$. Then\vspace*{5pt}:\\
 $(f \circ g)'(a) = \lim_{x \to a} \frac{f(g(x)) - f(g(a))}{x-a} = F(g(x))\frac{g(x) - g(a)}{x-a} = f'(b)g'(a) = f'(g(a))g'(a)$.
\end{proof}\vspace*{5pt}

\begin{theorem}
	If $f:\RR \to \RR$ is diff'ble at $a\in \RR$ with $f'(a) \neq 0$ and $f$ is bijective with inverse $g = f^{-1}$, then $g$ is diff'ble at $b = f(a)$ with $g'(b) = \frac{1}{f'(g(b))} = \frac{1}{f'(a)}.$
\end{theorem}
\begin{proof}
\textit{Lemma:} $f'(a) \neq 0 \implies \exists \delta > 0$ such that $f(x) \neq f(a)$ for $x \in (a-\delta,a + \delta)\backslash\{0\}$.  \\

\noindent So $\frac{g(y)-g(b)}{y-b} = \frac{x-a}{f(x) - f(a)} = 1/\frac{f(x)-f(a)}{x-a}$ with $x = g(y),~y \neq b$. As $y \to b$, $g(y) \to g(b) = a$ since $f$ diff'ble at $a \implies f$ cts at $a \implies g$ cts at $b \implies x \to a \implies$ RHS $\to \frac{1}{f'(a)}$.	
\end{proof}~\\



  \begin{center}
  \textsf{\textbf{- End of Analysis I -}}	
  \end{center}
  
  \stepcounter{lecture}
\setcounter{lecture}{1}
  
  
  		% Differentiation


\end{document}
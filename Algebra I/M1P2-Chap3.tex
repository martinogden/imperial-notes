%!TEX root = M1P2.tex

  \sektion{Ring Theory}
 Many  \lecturemarker{24}{10/03}
of the sets we've looked at so far have \emph{two} binary operations, $+$ and $\times$. We may be interested in studying both at the same time.\\
 
 For instance we might want to look at structures where equations like $x^2 + 1 = 2$, or $y^3 = x^2 + 2$ make sense, which involve both $+$ and $\times$.\\
 
\begin{definition} A \emph{ring}\index{Ring} is a set $R$ with two binary operations $+$ and $\times$ such that the following axioms hold:
 \begin{enumerate}
 \item $(R,+)$ is an abelian group.
 \item $(x \times y)\times z = x \times (y \times z)$ for all $x,y,z \in R$. \textsc{[Associativity]}
 \item $x \times (y + z) = (x \times y) + (x \times z)$, and $(x + y) \times z = (x \times z) + (y \times z)$ for all $x,y,z \in R$ \textsc{[Distributivity]} ``multiplication distributes over addition''\\ \vspace*{-10pt}
 
 $R$ is a \emph{ring with unity} if it also satisfies:\\  \vspace*{-12pt}

 \item  There exists $1 \in R$ such that $1 \times x = x \times 1 = x$. \\ \vspace*{-10pt}

$R$ is a \emph{commutative ring} if it also satisfies:\\ \vspace*{-12pt}
\item $x \times y = y \times x$ for all $x,y \in R$.\\ \vspace*{-10pt}

\boxed{\text{In this course, all rings are assumed to be commutative rings with unity.} }
 \end{enumerate}
 \end{definition}\vspace*{10pt}
 
\begin{examples} \begin{enumerate}
 \item $\Z, \Q, \mathbb{R}, \C$ with their usual $+$ and $\times$.
 \item $\mathbb{Z}_n$ for any $n > 0$, with $+$ and $\times$ defined on residue classes as before.
 \item A non-commutative ring (so Axiom 5 fails): Let $n >1$ and let $R =$ Mat$_n(\mathbb{R})$, the set of all $n \times n$ real matrices with the usual $+$ and $\times$.	
 \end{enumerate}
 \end{examples}\vspace*{10pt}
 
 
\begin{remarks}~ \begin{enumerate}
 \item We write $0$ for the identity of $(R,+)$.
 \item We can write $x.y$ or $xy$ for $x \times y$.
 \item Fact: $0x = 0$ for any $x \in R$. \begin{proof}
 $0x = 0x + 0$. But $0x = (0 + 0)x = 0x + 0x$ by Distributivity.	 So $0x + 0 = 0x + 0x$. Hence $0x = 0$ by left cancellation in $(R,+)$.
 \end{proof}
 \item It is possible to have a ring $R$ in which $0 = 1$. But in this case $R$ only has one element. This ring is the \emph{trivial ring}. We have seen it before - it is $\Z_1 = \{[0]_1\}$.
 \item The axioms do not say that multiplicative inverses exist. In fact if $R$ is non-trivial, then it has at least one element with no inverse, since $0$ is such an element.\begin{proof}
$0x = 0$ for any $x$. So $0x \neq 1$ for any $x$ unless $0 = 1$. But in this case $R$ is trivial.
\end{proof}
 \end{enumerate}
 \end{remarks}\vspace*{10pt}
 
\begin{definition}

 Let $F$ be a field (e.g. $\mathbb{R}$). The \emph{polynomial ring} $F[x]$ is the set of all polynomials in $x$ with coefficients from $F$, with the usual $+$ and $\times$ for polynomials. Example: $f_1(x)= x^2 + 1$, $f_2(x) = x-3$.
 \[(f_1 + f_2)(x) = x^2 + x -2 \quad (f_1f_2)(x) = x^3 -3x^2 + x -3\]\end{definition} \vspace*{5pt}
 
\begin{definition}
 A non-zero polynomial $f(x)$ has a \emph{degree}\index{Degree}, which is the largest power of $x$ occurring in $f(x)$. Write deg $f(x)$ for the degree. Then we have deg $f(x) \geq 0$ for all $f(x)$. deg $f(x) = 0$ if and only if $f(x)$ is a non-zero constant polynomial.\end{definition} \vspace*{5pt}
 
 We do not define the degree of the zero polynomial $f(x) = 0$. (People who do define it to be $-\infty$)\\
 
 \textit{Exercise:} Show that $F[x]$ is indeed a ring.\\
 
\begin{definition} Let $R$ be a ring with unity. A \emph{unit} in $R$ is an element $u$ such that there exists $w \in R$ with $uw = wu = 1$. So a unit in $R$ is an element with a multiplicative inverse.\end{definition} \vspace*{5pt}
 
\begin{examples}\begin{enumerate}
 \item If $R$ is $\Q,~\mathbb{R},~\C$, then all of its non-zero elements are units.	
 \item If $R = \Z$, then the units are $\{\pm 1\}$.
 \item If $p$ is prime and $R = \Z_p$, then all non-zero elements are units, since $\Z_p^*$ is a group under $\times.$
 \item  
Let   \lecturemarker{25}{11/03} $R = \Z_m$, with $n \neq 1$ and $n$ is not prime. The units in $\Z_m$ are the residue classes $[x]_m$ for $x \in \Z$ such that hcf$(x,m) =1$ (if hcf$(a,b) =1$, we say that $a,b$ are \emph{coprime}\index{Coprime})
 \begin{proof}
 Suppose that $[x]_m$ is a unit in $\Z_m$. Then there exists $[y]_m \in \Z_m$ such that $[x]_m[y]_m = [y]_m[x]_m = [1]_m$. But $[x]_m[y]_m = [xy]_m$, and so $xy \equiv 1 \mod m$. Hence $xy -1 = km$ for some $k\in \Z$. Now $xy - km = 1$. But $xy-km$ is divisible by hcf$(x,m)$. Hence hcf$(x,m) = 1$.\\
 
 Conversely, suppose hcf$(x,m) = 1$. Then Euclid's Algorithm tells us that there exists $y,z \in \Z$ such that $xy + mz = 1$. Now $xy \equiv 1 \mod m$ so $[x]_m[y]_m = [xy]_m = [1]_m$. Hence $[x]_m$ is a unit in $\Z_m$.	
 \end{proof}

\item What are the units in $F[x]$, where $F$ is a field? \\
\textit{Observation:} Let $f_1(x)$, $f_2(x)$ be non-zero polynomials. Then deg $(f_1(x) \times f_2(x)) =$ deg $f_1(x)~ +$ deg $f_2(x)$. In particular, it follows that deg $(f_1(x) \times f_2(x)) \geq$ deg $f_1(x)$.\\

Suppose that $f_1(x)$ is a unit in $F[x]$. Then there exists $f_2(x)$ such that $f_1(x) \times f_2(x) = 1$. So deg $f_1(x) \leq$ deg $1 = 0$. Hence deg $f_1(x) = 0$. Hence $f_1(x)$ is a constant polynomial.\\

Conversely if $f_1(x) = c$, a non-zero constant polynomial, then $f_2(x) = 1/c$ is an inverse for $f_1(x)$. So $f_1(x)$ is a unit. We have shown that units in $F[x]$ are precisely the non-zero constant polynomials.

\item A non-commutative example - the units in the ring Mat$_n(F)$. $(n \times n$ matrices over $F$) are the invertible matrices. There are the elements of $GL_n(F)$. 

 \end{enumerate}\end{examples}\vspace*{5pt}
 
\begin{definition} Given a ring $R$ with unity, we write $R^{\times}$ for the set of units in $R$.	
\end{definition}\vspace*{10pt}
 
\begin{theorem} For any ring $R$ with unity, $(R^{\times},\times)$ is a group. [Called the unit group of $R$.]	
\end{theorem}

 \begin{proof}
First check that $\times$ gives a binary operation on $R^{\times}$. Suppose that $u_1,u_2 \in R^{\times}$. Then there exists $w_1,w_2 \in R^{\times}$ such that $u_1w_1 = w_1u_1 = 1$ and $u_2w_2 = w_2u_2 = 1$. $(u_1u_2)(w_1w_2) = u_1(u_2w_2)w_1 = u_11w_1 = u_1w_1 = 1$. Also $(w_2w_1)(u_1u_2) = w_2(w_1u_1)u_2 = w_21u_2 = w_2u_2 = 1$. So $w_2w_1$ is an inverse for $u_1u_2$, and so $u_1u_2 \in R^{\times}.$\\

Now check the group axioms:\\
\textsc{Associativity} is given by the ring axioms.\\
\textsc{Identity:} 1 is a unit since $1\times 1 = 1$. So we have an identity in $R^{\times}$\\
\textsc{Inverses:} Let $u \in R^{\times}$. Then $u$ has an inverse $w$ in $R$. So $uw = wu = 1$. Now $u$ is an inverse for $w$, and so $w \in R^{\times}$. Hence $u$ has an inverse in $R^{\times}$.
 \end{proof}
 
  \textit{Remarks:}\begin{enumerate}
\item  The proof of Theorem 136 did not assume that $R$ is commutative.
\item Since all units lie in $R^{\times}$, and since any inverse to a unit is itself a unit, it follows that every unit in $R$ has a \emph{unique} inverse. (since this is true in the unit group.) So we can write $u^{-1}$ for the inverse of the unit $u$.
 \end{enumerate}\vspace*{5pt}
 
\begin{examples}\begin{enumerate}
 \item If $R$ is Mat$_n(F)$, then the unit group is $GL_n(F)$. 
 \item If $R$ is $\Q, ~\mathbb{R},~ \C,$ or $\Z_p$ where $p$ is prime, then $R^{\times} = R^* = R \backslash\{0\}$.
 \end{enumerate}
\end{examples}\vspace*{5pt}
 
\begin{definition} A \emph{field}\index{Field} is a commutative ring, $F$, such that $F^* =  F\backslash\{0\}$ (Every non-zero element is a unit.)	
\end{definition}
\vspace*{10pt}
 
 \subsektion{Arithmetic of Rings}

\begin{definition} Let $R$ be a ring. Let $a,b \in R$. Say that $a$ \emph{divides} $b$ if there exists $c \in R$ such that $ac = b$. We write $a ~|~ b$ to mean that $a$ divides $b$.	
\end{definition}\vspace*{10pt}

\begin{examples} \begin{enumerate}
 \item If $R = \Z$ then ``$a$ divides $b$'' means what we expect it to.
 \item If $R = \Q,~\mathbb{R},\C$ or any other field, then $a ~|~ b$ for any $a,b \in R$, except when $a = 0$ and $b \neq 0$.
 \item In a polynomial ring $F[x]$, ``divides'' means what it ought to. For example $x -1$ divides $x^2 -2x + 1$ in $\Q[x]$.
 \end{enumerate}
\end{examples}

 
 
 
\begin{proposition} \lecturemarker{26}{13/03}
\begin{enumerate}
 \item  If $a ~|~ b$ then $a ~|~ bc$ for all $c \in R$.
 \item If $a ~|~ b$ and $a ~|~ c$ then $a ~|~ b + c$
 \item If $a ~|~ b$ and $b ~|~ c$ then $a ~|~ c$.
 \item If $u \in R$ is a unit, then $u ~|~ b$ for all $b\in R$.
 \item If $a~|~b$ and if $u$ is a unit, then $au ~|~ b$.
 \item If $u$ is a unit and $w ~|~ u$ then $w$ is also a unit.
 \item $a~|~ 0$ for any $a \in R$.
 \item If $0~|~b$ then $b = 0$.
 \end{enumerate} \end{proposition}(\textit{Proof.} Left as exercise)\\
 
 
 \begin{definition}Let $R$ be a ring, and $r \in R$. We say that $r$ is a \emph{zero-divisor}\index{Zero-divisor} if $\exists s \in R$ such that $s \neq 0$, and $rs = 0$. A ring $R$ with no zero-divisors, apart from $0$, is called an \emph{integral domain}\index{Integral Domain}.
	 \end{definition}\vspace*{10pt}

\begin{examples}\begin{enumerate}
 \item $\Z,~\Q,~\mathbb{R},~\C,~\Z_p$ ($p$ prime), $F[x]$ (where $F = \Q,~\mathbb{R},~\C)$ are all integral domains.
 \item If $m >1$, and $m$ is not prime, then $\Z_m$ is \emph{not}  an integral domain, since if we have $m = ab$, $1<a<m$, then $[a]_m$ and $[b]_m$ are non-zero, but their product is $[m]_m = [0]_m$, and so $[a]_m$ has a zero-divisor.\\
 \end{enumerate}\end{examples}\vspace*{10pt}

\begin{proposition}A unit in a ring $R$ is not a zero-divisor (unless $R$ is trivial.)	
\end{proposition}

 \begin{proof}
 	Let $u$ be a unit. Suppose that $ua = 0$ for some $a \in R$. Then $u^{-1}(ua) = u^{-1}0 = 0$. So (since $u^{-1}(ua) = 1a = a$) we have $a = 0$. Hence $u$ is not a zero-divisor
 \end{proof}
 
  \textbf{Corollary 145.} Any field is an integral domain.
 \begin{proof}
 All non-zero elements in a field are units, and so they are not zero-divisors.  	
 \end{proof}\vspace*{10pt}

 

\begin{proposition} Let $R$ be an integral domain. Let $a,b \in R$ be such that $a ~|~ b$ and $b ~|~ a$. Then there exists a unit $u\in R$ such that $b = au$. %non ID's => \not\exists	
\end{proposition}

\begin{proof}
	Since $a~|~b$, there exists $c \in R$ with $b = ac$. And since $b~|~a$, there exists $d \in R$ such that $a = bd$. $(ac)d = bd = a$, and by associativity it follows that $a(cd) = a$. %not in a group, no cancellation 
	Hence $a(1-cd) = 0$. So one of $a$ or $1-cd = 0$. If $a = 0$, then $b = 0$ since $a~|~ b$, so $a.1 = b$. If $1-cd = 0$, then $cd = 1$, and so $d$ is an inverse for $c$, so $c$ is a unit, with $b = ac$.
\end{proof}\vspace*{10pt}

\begin{definition} Let $R$ be an integral domain. Let $r$ be a non-zero, non-unit in $R$. We say that $r$ is \emph{irreducible}\index{Irreducible} if it can't be written as $r = st$, where $s,t \in R$, and neither $s$ nor $t$ is a unit. Otherwise $r$ is \emph{reducible.}	
\end{definition}
\vspace*{10pt}

\begin{examples}\begin{enumerate}
 \item In $\Z$ the irreducible elements are the primes (positive and negative). Note that if $p$ is prime then $p$ has two factorisations, $p = 1\times p = (-1) \times (-p)$. Here $1$ and $-1$ are the units in $\Z$.
 \item In a field there are no irreducible (or reducible) elements, since every element is either $0$ or a unit.
 \item $f(x) = x-3$ is irreducible in $\Q[x]$ or $\mathbb{R}[x]$ or $\C[x]$. %=> a deg 0 => unit
 \item $f(x) = x^2 -2$ is irreducible in $\Q[x]$. But in $\mathbb{R}[x]$ or $\C[x]$, $f(x)$ is reducible since $x^2 - 2 = (x-\sqrt{2})(x + \sqrt{2})$, and neither of these factors are units.
 \item $f(x) = x^2 + 1$ is irreducible in $\Q[x]$ and $\mathbb{R}[x]$, but is reducible in $\mathbb{C}[x]$ as $x^2 + 1 = (x + i)(x -i)$.
 \end{enumerate}
 \end{examples}

 \textit{Remark.} In an integral domain, every element is exactly one of: zero, a unit, irreducible or reducible. \\

\begin{proposition} Let $F$ be $\Q,~\mathbb{R},~\C$ and let $f(x) \in F[x]$. Let $a \in F$. Then $f(a) = 0$ if and only if $x -a ~|~ f(x).$	
\end{proposition}

\begin{proof}
  \lecturemarker{27}{17/03}

Suppose that $x-a ~|~ f(x)$. Then $f(x) = (x-a)g(x)$, for some $g(x) \in F[x]$. Now $f(a) = (a-a)g(a) = 0$. \\ \vspace*{-5pt}

 For the converse, suppose that $f(a) = 0$. Let $f(x) = \alpha_0 + \alpha_1x + \dots + \alpha_nx^n$, where $\alpha_0, \dots, \alpha_n \in F$. Then $f(x) - f(a) = (\alpha_0 - \alpha_0) + (\alpha_1x -  \alpha_1a) + \dots +(\alpha_nx^n - \alpha_na^n)= \alpha_1(x-a) + \alpha_2(x^2 - a^2) + \dots + \alpha_n(x^n -\alpha^n).$ \\

Note that $x^k = a^k = (x-a)(x^{k-1}+ x^{k-2}a + \dots a^{k-1})$. So $x-a$ divides $x^k - a^k$ for all $k \in \mathbb{N}$. Hence $x-a$ divides $f(x) - f(a)$. But $f(a) = 0$ by assumption, and so $x-a$ divides $f(x)$.
\end{proof}\vspace*{5pt}
\begin{theorem}[Fundamental Theorem of Algebra] \index{Fundamental Theorem of Algebra} Let $f(x)\in \C[x]$ be a polynomial of degree greater than 0. Then $f(x)$ has a root in $\C$. (So there exists $a \in \C$ such that $f(a) = 0$.)	
\end{theorem}\vspace*{10pt}

 \textit{Proof. }
Next year's Complex Analysis course.	\\

\begin{corollary}The irreducible elements in $\C[x]$ are the linear polynomials, $\alpha_1x + \alpha_0$, where $\alpha_1 \neq 0$.	
\end{corollary}

\begin{proof}
First show that $\alpha_1x + \alpha_0$ is irreducible. Suppose that $\alpha_1x + \alpha_0 = f(x)g(x)$. Then deg $f(x) +$ deg $g(x) = 1$. So one of $f(x)$ or $g(x)$ has degree $0$, and so is a unit in $\C[x]$. Hence $\alpha_1x + \alpha_0$ is irreducible in $\C[x]$.\\

Conversely, suppose that $r(x)$ has degree $d$ greater than $1$. Then $r(x)$ has a root $a$ by the Fundamental Theorem. So $r(a) = 0$, and so $x-a$ divides $r(x)$ by Proposition 149. So $r(x) = (x-a)s(x)$ for some $s(x) \in \C[x]$. Now deg $s(x) = $ deg $r(x) -1 = d-1 > 0$, so neither $s(x)$ nor $r(x)$ are units in $\C[x]$. So $r(x)$ is reducible. 
\end{proof}\vspace*{5pt}

 \textit{Exercise:} What are the irreducible elements in the ring $\mathbb{R}[x]$?\\

\subsektion{The Rings $\Z[\sqrt{d}]$}

\begin{definition} Let $d \in \Z$ be a non-square. The ring $\Z[\sqrt{d}]$ is the set $\{x + y\sqrt{d} : x,y \in \Z\} \subseteq \C$, with the usual complex $+$ and $\times$.	
\end{definition}\vspace*{10pt}

Check that these give binary operations on our set:\vspace*{3pt}\\ We have $(x_1 + y_1\sqrt{d}) + (x_2 + y_2\sqrt{d}) = (x_1 + x_2) + (y_1 + y_2)\sqrt{d} \in \Z[\sqrt{d}]$,\vspace*{3pt}\\
and $(x_1 + y_1\sqrt{d})(x_2 + y_2\sqrt{d}) = (x_1x_2 + dy_1y_2) + (x_1y_2 + x_2y_1)\sqrt{d} \in \Z[\sqrt{d}]$.\\

 The ring axioms are left as an exercise.\\

 \textbf{Example:} $\Z[\sqrt{-1}] = \Z[i]$ is the set $\{x + yi : x,y \in \Z\}$, the ring of \emph{Gaussian integers.}\index{Gaussian integers}\\

\begin{proposition} Let $d$ be a non-square in $\Z$. If $x_1 + y_1\sqrt{d} = x_2 + y_2\sqrt{d}$ for $x_1, x_2, y_1, y_2 \in \Z$, then $x_1 = x_2$ and $y_1 = y_2$.	
\end{proposition}

\begin{proof}
Since $d$ is a non-square in $\Z$, we know that $\sqrt{d} \not\in \Q$. Suppose that $x_1 + y_1\sqrt{d} = x_2 + y_2\sqrt{d}$. Then $x_1 - x_2 = (y_2 - y_1)\sqrt{d}$. If $y_2 \neq y_1,$ then $\sqrt{d} = \frac{x_1 - x_2}{y_2 - y_1} \in \Q$, which is a contradiction. So $y_2 = y_1$. Now $x_1 -x_2 = 0$, and so $x_2 = x_1$, as well. 	
\end{proof}\vspace*{5pt}

\begin{definition} Define a function $N: \Z[\sqrt{d}] \to \Z$, by $N(x + y\sqrt{d}) = x^2 - dy^2.$ $N$ is the \emph{norm map}\index{Norm Map} on $\Z[\sqrt{d}].$	
\end{definition}\vspace*{10pt}

\begin{examples}\begin{enumerate}
\item In $\Z[i]$, the norm map is $x + iy \mapsto x^2 + y^2 = |x + iy|^2$.
\item More generally, if $d < 0$, then the norm map is $x+ y\sqrt{d} \mapsto x^2 - dy^2 = x^2 + |d|y^2 = |x + y\sqrt{d}|^2$ (since $x + y\sqrt{d} = x+y\sqrt{|d|}i$).
\item In $\Z[\sqrt{2}]$, the norm map is $x + y\sqrt{2} \mapsto x^2 - 2y^2$. So for example $N(1 + \sqrt{2}) = -1$. When $d > 0$, the norm map can take negative values.
 \end{enumerate}\end{examples}\vspace*{10pt}
 
\begin{proposition}\begin{enumerate}
 \item If $r \in \Z[\sqrt{d}],$ and if $N(r) = 0$, then $r = 0 = 0 + 0\sqrt{d}$
 \item 	If $r,s \in \Z[\sqrt{d}]$, then $N(rs) = N(r)N(s)$.
 \end{enumerate}\end{proposition}
\begin{proof} [Proof (1)] 
Let $r = x + y\sqrt{d}$, and suppose that $N(r) = 0$. Then $x^2 - dy^2 = 0$, and so $x^2 = dy^2$. Now if $y \neq 0$, then $d =\frac{x^2}{y^2} = \left(\frac{x}{y}\right)^2,$ and so $\sqrt{d} \in \Q$. But this is impossible. Hence $y = 0$. Now $N(r) = x^2$, so $x^2 = 0$ and so $x = 0$. Proof of (2) in Homework Sheet.
\end{proof}\vspace*{5pt}



% Recall N(x + y\sqrt{d}) = x^2 -dy^2
\begin{proposition} An   \lecturemarker{28}{18/03}
 element $r$ of $\Z[\sqrt{d}]$ is a unit if and only if $N(r) = \pm 1$. \end{proposition}
\begin{proof}
(only if) Suppose that $r$ is a unit. Then $r$ has an inverse $r^{-1}$. Now $rr^{-1} = 1$, and so $N(r)N(r^{-1}) = N(1)$. But $1 = 1 + 0 \sqrt{d}$, so $N(1) = 1$. Hence $N(r)$ has the inverse $N(r^{-1})$ in $\Z$, so $N(r)$ is a unit in $\Z$. So $N(r) = \pm 1$. \\

(if) Let $r = x + y\sqrt{d}$, and suppose that $N(r) = \pm 1$. Let $s = x - y\sqrt{d}$. Now $rs = x^2 - dy^2 = N(r) = \pm 1$. If $N(r) = 1$ then $s$ an inverse for $r$. If $N(r) = -1$  then $-s$ is an inverse for $r$. In either case, $r$ is a unit in $\Z[\sqrt{d}]$.
\end{proof}\vspace*{5pt}

 \textit{Remark.} If $d = -1$, the $\Z[\sqrt{d}] = \Z[i]$ has $4$ roots, $\pm 1, \pm i$. If $d < -1$, then $\Z[\sqrt{d}]$ has only two units, $\pm 1$. If $d > 0$ then $\Z[\sqrt{d}]$ has infinitely many units.\\

 \textit{Remark.} Elements that are irreducible in $\Z$ [primes] need not be irreducible in $\Z[\sqrt{d}]$.\\

\begin{examples} \begin{enumerate}
 \item $3$ is not irreducible in $\Z[\sqrt{3}]$ since $3 = \sqrt{3} \times \sqrt{3}$, and $\sqrt{3}$ is not a unit.
 \item $2$ is not irreducible in $\Z[i]$, since $2 = (1+i)(1-i)$. 
 \item \textit{Question:} Is $11$ irreducible in $\Z[\sqrt{3}]$?\vspace*{5pt}\\
\textit{Answer:} We have $N(11) = 11^2 = 121$. Suppose that $11 = ab$ in $\Z[\sqrt{-3}]$, where $a$ and $b$ are non-units. $N(a), N(b) \neq \pm 1$. We have $N(a)N(b) = N(11) = 121$. So we must have $N(a) = N(b) = \pm 11$. Since $N$ takes non-negative values on $\Z[\sqrt{-3}]$, we have $N(a) = 11$. Let $a = x + y\sqrt{d}$. Then $x^2 + 3y^2 = 11$. But it is easy to check that no such $x,y$ exist in $\Z$, which is a contradiction. So no such $a,b$ exist. So $11$ is irreducible in $\Z[\sqrt{-3}]$. %+3 harder 
 \end{enumerate}\end{examples}
 
 \subsektion{Highest Common Factor / Greatest Common Divisor}
\begin{definition} Let $R$ be a ring. Let $a,b,c \in R$. Then $c$ is a \emph{highest common factor}\index{Highest Common Factor} (hcf) or \emph{greatest common divisor} (gcd) for $a$ and $b$ if\begin{enumerate}
\item[(i)] $c ~|~ a$ and $c ~|~ b$	 ~($c$ is a common factor for $a$ and $b$.)
\item[(ii)] If $d ~|~ a$ and $d ~|~ b$ then $d ~|~ c$ for $d \in R$.
\end{enumerate}\end{definition}

 \textit{Remarks.} We do not claim that a highest common factor necessarily exists for all $a,b \in R$. There exists rings $R$ and elements $a,b$ where this doesn't happen. Where hcfs do exist, they are not usually unique. For instance if $R = \Z$, and $a = 4,~ b = 6$, then the hcfs of $a$ and $b$ are $2, -2$. \\

\begin{proposition}Let $R$ be an integral domain. Let $c$ be an hcf for $a$ and $b$ in $R$. Then $d$ is a highest common factor for $a$ and $b$ if and only if $d = cu$, for some unit $u \in R$. 	
\end{proposition}

\begin{proof}
	(only if) Suppose that $c$ and $d$ are both hcfs for $a$ and $b$. So by the 2nd condition in the definition of hcf, we have $c ~|~ d$ and $d ~|~c$. Now by Proposition 146, we have $d = cu$ for a unit $u$. \vspace*{-5pt}\\

	(if) Let $d = cu$, where $u$ is a unit. Since $c ~|~ a$ and $c~|~ b$, we have $cu ~|~ a$ and $cu ~|~ b$, by Proposition 141.5. So $d ~|~ a$ and $d ~|~ b$, so $d$ is a common factor for $a$ and $b$. To show the 2nd hcf condition holds, let $e$ be any common factor of $a$ and $b$. So $e ~|~ a$ and $e ~|~ b$. Then $e ~|~ c$, since $c$ is a hcf. So $e ~|~ cu $ by Proposition 141.3. Hence $e ~|~ d$. Hence $d$ is an hcf for $a$ and $b$.
\end{proof}\vspace*{10pt}


 \textit{Reminder.} Euclids   \lecturemarker{29}{20/03}
 Lemma: If we have any two $a,b\in \Z$, with $b \neq 0$, then there exists $q$ and $r \in \Z$, with  $0 \leq r < |b|$, such that $a = qb + r$.\\


\begin{lemma} Let $f(x)$ and $g(x)$ be polynomials with coefficients from $F,$ $(\Q,,\C)$, with $g(x) \neq 0$. There exist polynomials $q(x)$ and $r(x) \in F[x]$ such that $f(x) = q(x)g(x) + r(x)$, and either $r(x) =0$ or else deg $r(x) <$ deg $g(x)$. 	
\end{lemma}

 \textbf{Example.} Take $F = \Q$, $f(x) = x^4 + x^3 + 2x + 3$ and $g(x) = x^2 -1$. Dividing $f(x)$ by $g(x)$, we get $f(x) = g(x)(x^2 + x + 1) + (3x+4)$. So here $q(x) = x^2 + x + 1$, and $r(x) = 3x + 4$. Here deg $r(x) = 1 < 2 = $ deg $g(x)$.\\

\begin{proof}[Proof of Lemma 161.]
	Take deg $f(x) = m$ and deg $g(x) = n$. If $m < n$, take $q(x) =0$ and $r(x) = f(x)$, and this satisfies the lemma. So we will assume that $m \geq n$. Now we argue by induction on $m = $ deg $f$. The base case is $m = 0$ (and so $n = 0$ too). $f(x) = a$ and $g(x) = b$ for $a,b \in F$, with $b \neq 0$. Take $q(x) = \frac{a}{b}$, $r(x) = 0$, and this satisfies the lemma.\\  
	\textit{Inductive step:} Assume the lemma holds for all polynomials $f(x)$ of degree $<m$. Suppose deg $f(x) = m$. Put \[f(x) = a_mx^m + \dots + a_1x + a_0 \text{ and } g(x) = b_nx^n + \dots + b_1x + b_0\] Define $f'(x) = f(x) - \frac{a_m}{b_n}x^{m-n}g(x)$. The coefficient of $x^m$ in $f'(x)$ is $0$, so deg $f'(x) < m$. So the inductive assumption applies to $f'(x)$, and so there exists $q'(x)$ and $r(x)$ such that $f'(x) = q'(x)g(x) + r(x)$, and with $r(x) = 0$ or deg $r(x) <$ deg $g(x)$.\vspace*{5pt}\\ But now $f(x) = f'(x) + \frac{a_m}{b_n}x^{m-n}g(x) = (q'(x) + \frac{a_m}{b_n}x^{m-n})g(x) + r(x).$ So take $q(x) = q'(x) + \frac{a_m}{b_n}x^{m-n}$, and this satisfies the lemma.
\end{proof}\vspace*{5pt}

 \textbf{Example:} $f(x) = 3x^2 + 5x + 1$, $g(x)= 2x -1$. Calculate $q(x)$ and $r(x)$ using polynomial long division:\\

\polylongdiv{3x^2 + 5x + 1}{2x-1}

So $q(x) = \frac{3}{2}x + \frac{13}{14}, \quad r(x) = \frac{17}{4}$. \\
%Note that leading term of quotient is a_m / b_n, f'(x) is the first remainder

\begin{proposition} Let $R$ be a ring. Let $a,b,q,r \in R$ such that $a = bq + r$. Then $d$ is a highest common factor for $a$ and $b$ if and only if $d$ is a highest common factor for $b$ and $r$.	
\end{proposition}

\begin{proof}
We actually show that the common factors of $a$ and $b$ are the same as those of $b$ and $r$. Suppose $d$ divides $b$ and $r$. Then $d$ divides $bq$ and $r$. So $d$ divides $bq + r = a$. So $d$ divides $a$ and $b$.\\
Conversely, suppose $d$ divides $a$ and $b$. Note that $r = a - bq$. Now $d$ divides $a$ and $bq$, so $d$ divides $a - bq = r$.	
\end{proof}\vspace*{5pt}

 Propositions 161 and 162 give us a Euclidean Algorithm for polynomials. \\

\begin{example}[Euclidean Algorithm for polynomials.] \\$f(x) = x^3 - 2x^2 -5x + 6$, $g(x) = x^2 - 2x -3$.\\
First find $q(x)$ and $r(x)$:\\
\polylongdiv{x^3 - 2x^2 -5x + 6}{x^2 - 2x -3}~

 We have $q(x) = x$, $r(x) = -2x + 6$. Now we look for a hcf of $g(x)$ and $r(x).$ (A ``smaller'' problem than the original.) So applying Euclid's Algorithm again, so now quotient is $-\frac{1}{2}x - \frac{1}{2} ,$ now remainder is $0$. We've shown that $r(x) = -2x + 6$ divides $g(x)$. So $r(x)$ is a hcf for $r(x)$ and $g(x)$. So $r(x)$ is also a hcf for $f(x)$ and $g(x)$ by Proposition 162.\\

N.B. $-2$ is a unit in $\Q[x]$, with inverse $-\frac{1}{2}$. So $x-3$ is another hcf for $f(x)$ \& $g(x)$.
\end{example}


%What did we need to make Euclid's Algorithm? \begin{itemize}
% \item Prop 162... This holds in any ring.
%\item  Notion of ``smallness'' such that $r$ is ``smaller'' than $b$
% \item Some version of Euclid's Lemma using this notion of `smallness''
%\end{itemize}

\begin{definition} Let  \lecturemarker{30}{24/03}
 $R$ be an integral domain. A \emph{Euclidean function}\index{Euclidean Function} on $R$ is a function $f: R\backslash\{0\} \to \mathbb{N} \cup \{0\}$, which satisfies the following two conditions: \begin{enumerate}
 \item[(i)] $f(ab) \geq f(a)$ for $a,b \in R\backslash\{0\}$
 \item[(ii)] For all $a,b \in R,~b \neq 0$, there exists $q,r \in R$ such that $a = qb + r$, either $r = 0$, or else $f(r) < f(b)$.	
 \end{enumerate}\end{definition}

\begin{examples} \begin{enumerate}
 \item If $R = \Z$ then $f(n) = |n|$ is a Euclidean function. 
 \item If $R = F[x]$ then deg $f(x)$ is a Euclidean function.	
 \end{enumerate}\end{examples}
 \vspace*{5pt}

Euclidean functions are what we need for Euclidean algorithms: \vspace*{5pt}\\
\begin{algorithm}To find a hcf for $a$ and $b$ in a ring $R$ with a Euclidean function $f$.\begin{itemize}
\item[\textbf{Step 1.}] Find $q$ and $r$ such that $a = qb + r$ and $r$ is $0$ or $f(r) < f(b)$
\item[\textbf{Step 2.}]	If $r = 0$ then $b$ is a hcf for $a$ and $b$. Otherwise:
\item[\textbf{Step 3.}] Start the algorithm again, replacing $a$ with $b$ and $b$ with $r$.

\textit{Since $f(b) \in \mathbb{N}\cup \{0\}$, this algorithm must eventually terminate.}
\end{itemize}
\end{algorithm}\vspace*{10pt}

\begin{definition} An integral domain with at least one Euclidean function is called a \emph{Euclidean Domain}\index{Euclidean Domain}.\vspace*{5pt}\\
It can be very difficult to decide whether an Integral Domain is Euclidean.
\end{definition}\vspace*{10pt}

Can we find a Euclidean function on $\Z[\sqrt{d}]$? Sometimes!\\

\begin{proposition} Let $d \in \{-2,-1,2,3\}.$ Then the function $f(a) = |N(a)|$ is a Euclidean function on $\Z[\sqrt{d}]$.	
\end{proposition}

\begin{proof}
The first condition from Definition 164 is easy. Since $|N(b)| \geq 1$ for $b \neq 0$, we have $|N(ab)| = |N(a)||N(b)| \geq |N(a)|$.\\

For the second condition, let $a = x + y \sqrt{d}$ and $b = v + w\sqrt{d}$, with $b \neq 0$. In $\C$, calculate: 

\[\dfrac{a}{b} = \dfrac{x + y\sqrt{d}}{v + w\sqrt{d}} = \frac{(x + y\sqrt{d})(v - w\sqrt{d})}{v^2 - dw^2} = \frac{1}{N(b)}((xv-ywd)+(yv-xw)\sqrt{d}).\]
 \[\text{Put ~~ }\alpha = \frac{xv - ywd}{N(b)},~\beta = \frac{yv - xw}{N(b)}, \in \Q\]
 
 Set $m,n$ to be the integers closest to $\alpha,\beta$ respectively. So 
 \[|\alpha - m| \leq \frac{1}{2} \quad \text{and} \quad |\beta - n| \leq \frac{1}{2} \quad (*)\]
 \[\text{Put~~ } q = m + n\sqrt{d} \in \Z[\sqrt{d}] \text{, and } r = a - bq\]
 We show that $|N(r)| < |N(b)|$:\\
 
 Define $c  = N(b)(\frac{a}{b} - q) = N(b)\frac{r}{b} \in \C.$ So $bc = N(b)r$. We have: 
 \[\begin{aligned}
	c &= N(b)(\alpha + \beta \sqrt{d} - m - n\sqrt{d}) \\
	& = N(b)(\alpha - m) + N(b)(\beta - n)\sqrt{d}\\
	& \in \Z[\sqrt{d}], \text{ since } N(b)\alpha \text{ and } N(b)\beta \in \Z
\end{aligned}
 \]
 We have $N(bc) = N(b)N(c) = N(N(b)r) =  N(b)^2N(r)$.\\ 
 
 So $N(c) = N(b)N(r)$. But $N(c) = N(b)^2(\alpha-m)^2 - N(b)^2(\beta-n)^2d$. $N(r) = N(b)(\alpha-m)^2 -d(\beta-n)^2).$\\
 
 Now from $(*)$, $|\alpha-m|$ and $|\beta -n| \leq \frac{1}{2}$. So if $-2 \leq d \leq 3$, it is easy to see that $|(\alpha-m)^2 -d(\beta-n)^2| < 1$. So $|N(r)| < |N(b)|$ as required.
\end{proof}\vspace*{10pt}


  
\begin{example}Find   \lecturemarker{31}{25/03}
a hcf of $a = 4 + \sqrt{2}$ and $b = 2 - 2\sqrt{2}$ in $\Z[\sqrt{2}]$.\\
  
 In $\C$ we calculate: \[\frac{a}{b} = \dfrac{4 + \sqrt{2}}{2 - 2\sqrt{2}} = \dfrac{(4 + \sqrt{2})(2 + 2\sqrt{2})}{(2 - 2\sqrt{2})(2 + 2\sqrt{2})} = -3 - \frac{5}{2}\sqrt{2}\]
 
 So set $q = -3 - 2\sqrt{2}$. ($q = -3-3\sqrt{2}$ would also work.)\\
 
  Now $r = a - bq = 2-\sqrt{2}$. (Notice $|N(r)| = 4 - 2 = 2$, $|N(b)| = |4 - 8| = 4$). Continue, replacing $a$ with $b$ and $b$ with $r$. In $\C$:
 \[\frac{b}{r} = \frac{2-2\sqrt{2}}{2 -\sqrt{2}} = \sqrt{2}\in \Z[\sqrt{2}]\].
 So $q' = -\sqrt{2},~r' = 0$. So $r = 2-\sqrt{2}$ divices $b = 2-2\sqrt{2}$, and so $r$ is a hcf for $b$ and $r$, hence $r$ is a hcf for $a$ and $b$.
 \end{example}
 
 \begin{lemma} [B\'{e}zout's Lemma] \index{B\'{e}zout's Lemma|idxbf} Let $R$ be a Euclidean domain. Let $a$ and $b$ be elements of $R$ and let $d$ be a hcf for $a$ and $b$. Then there exists $s,t\in R$ such that $as + bt = d$.	
 \end{lemma}

 
 \begin{proof}
 	Define $X = X_{a,b} = \{ax + by : x,y \in R\}$. So $X \subseteq R$. The ring $R$ has a Euclidean function $f$. So if $a$ and $b$ are not both $0$, then $X$ has non-zero elements. So there exists some least $n \in \mathbb{N}\cup\backslash\{0\}$ such that $f(x) = n$ for some $x \in X\backslash\{0\}.$ Let $c$ be some elements of $X \backslash\{0\}$ such that $f(c) = n$. We show that $c$ is a hcf for $a$ and $b$:\\
 	
 	Since $c = ax + by$ for some $x,y$, it is clear that any common factor of $x$ and $y$ must divide $c$.\\
 	
 	We know that there exists $q$ and $r\in R$ such that $a = qc + r$, and either $ r = 0$ or $f(r) < f(c)$. Now $c = ax + by$, so $r = a -qc = a - q(ax + by) = a(1-qx) -b(qy) \in X$. We can't have $f(r) < f(c)$, since $f(c)$ is the smallest possible $f$ for elements of $X$. Hence $r = 0$, and so $c~|~ a$. A similar argument shows that $c ~|~ b$, and so $c$ is a common factor - and hence a hcf - for $a$ and $b$.\\
 	
 	%Proved for some hcf, not all of them 
 	Now let $d$ be any hcf of $a$ and $b$. Then $d =  cu$ for some unit $u \in R$. Now $d = (ax + by)u = axu + byu$. So put $r = xu$ and $s = yu$, and we're done.
 \end{proof}\vspace*{10pt}
 
 
 
 
 \subsektion{Unique Factorisation}
 
\begin{definition}Let $R$ be an integral domain, and let $a,b \in R$. We say $a$ and $b$ are \emph{coprime}\index{Coprime}  if $1$ is a hcf for $a$ and $b$. (Equivalently, any unit is a hcf.)	
\end{definition}\vspace*{5pt}
 
 \begin{proposition} Let $R$ be a Euclidean Domain. Suppose $a$ and $b$ are coprime in $R$, and suppose $a$ divides $bc$. Then $a$ divides $c$.
\end{proposition}

 \begin{proof}
 Since $a ~|~ bc$, we can write $ad = bc$, for some $d \in R$. Since $R$ is Euclidean, and $a$ and $b$ are coprime, we use B\'{e}zout's Lemma: there exist $s,t \in R$ such that $as + bt = 1$. Now $c = 1c = (as + bt)c = asc + btc = asc + (bc)t = asc + (ad)t = a(sc + dt)$, which is divisible by $a$. 
 \end{proof}\vspace*{5pt}
 
 Proposition 172 can fail if $R$ is not Euclidean.
 
 \textbf{Example:} $R = \Z[\sqrt{-3}]$. $a = 2, ~b = 1 + \sqrt{-3},~ c = 1-\sqrt{-3}$. Then $a$ and $b$ are coprime, and $2$ divides $bc = 4$. But $a$ does not divide $c$. (This shows $\Z[\sqrt{-3}]$ is not Euclidean.)\\
 
 \begin{definition} A \emph{unique factorisation domain}\index{Unique Factorisation Domain} is an integral domain $R$ with the following properties:\begin{enumerate}
\item For every $a \in R$, not zero and not a unit, there exists irreducible elements $p_1,\dots,p_s$ in $R$ such that $a = p_1\dots p_s$.
\item Let $p_1,\dots,p_s$ and $q_1,\dots,q_t$ be irreducible in $R$, such that $p_1\dots p_s = q_1\dots q_t$. Then $t = s$, and reordering the $q_i$ if necessary, we have $p_i = q_iu_i$ for some unit $u_i$, for all $i \in \{1,\dots,s\}$. (Factorisation is unique ``up to units''.)
\end{enumerate} \end{definition}

\textit{Example:} $R = \Z$. Then every element except $0,1,-1$ can be written as a product of irreducibles (primes), unique up to sign. $30 = 2\times 3 \times 5 = (-2) \times 3 \times (-5)$.\\

  
 Not   \lecturemarker{32}{27/03}
 every integral domain is a UFD. We've seen that $\Z[\sqrt{-z}]$ is not a UDF.\\
  
\begin{example} In $\Z[\sqrt{-5}]$ we can factorise $6$ as $2\times 3$, and also as $(1 + \sqrt{-5})(1-\sqrt{-5})$. Are these factors irreducible?\vspace*{5pt}\\ We have $N(2) = 4$, $N(3) = 9$. $N(1 \pm \sqrt{-5}) = 6$. If $ab = 2$ or $3$ or $(1 \pm \sqrt{-5})$, and if neither $a$ nor $b$ is a unit, then $N(a)$ must be $\pm 2$ or $\pm 3$. But it is easy to check that the equations $x^2 + 5y^2 = \pm 2$ or $\pm 3$ has no solutions for $x,y \in \Z$. So all of $2,3,1\pm \sqrt{-5}$ are irreducible in $\Z[\sqrt{-5}]$. \vspace*{5pt}\\ Are the factorisations of $6$ the same ``up to units''? No, since $N(2) \neq N(1 \pm \sqrt{-5})$. So $6$ is not uniquely factorisable in $\Z[\sqrt{-5}]$
	
\end{example}

  
  
  \begin{theorem} Every Euclidean Domain is a Unique Factorisation Domain. %converse isn't true\\	
  \end{theorem}

   \textit{Proof.} Omitted. (In Algebra II (?)) \includegraphics[width=0.08\textwidth]{al2.jpg} 

It follows that $\Z[\sqrt{-5}]$ is not a Euclidean Domain, by Example 174.


\begin{proposition} Let $R$ be a UFD. Let $p$ be an irreducible element of $R$, and let $a,b\in R$ be such that $p ~|~ ab$. Then either $p~|~ a$ or $p~|~ b$.	
\end{proposition}

\begin{proof}
If $a$ or $b$ is $0$ or a unit, then the result is clear. So assume otherwise. So by the first UFD property, there exist irreducible elements $q_1,\dots,q_s$, $r_1,\dots,r_t$, such that $a = q_1\dots q_s$, $b= r_1\dots r_t.$ So $ab = q_1\dots q_s r_1 \dots r_t$.\vspace*{5pt}\\ Now $p~|~ ab$, so $ab = pc$ for some $c \in R$. Suppose that $c = p_1\dots p_u$ as a product of irreducibles. Then $ab = pp_1\dots p_u$. Now by the second (uniqueness) property of UFDs, we have $p = q_iw$ or $r_i w$ for some $i$, and some unit, $w$. If $p = q_iw$ then $p~|~ a$, and if $p = r_1w$ then $p ~|~b$, as required.
\end{proof}

\begin{definition} A non-unit element, $r$, of a ring $R$ is \emph{prime}\index{Prime} if it has the property that whenever $r ~|~ab$ we have either $r~|~ a$ or $r~|~b$.	
\end{definition}\vspace*{5pt}

 \textit{In any integral domain, any prime element is irreducible. Why?}

If $r$ is prime, and $r = ab$, then $r ~|~ ab,$ and then $r~|~ a$ or $r~|~b$ by the prime property. Suppose $r~|~a$. Then since $a ~|~ r$, we have $r = au$ for a unit $u$. So $ab = au$, and so $a(b-u) = 0$. But $R$ is an integral domain, so $b - u =0$. Hence $b = u$, a unit.

In general the converse is not true - we've seen that $2$ is irreducible but not prime in $\Z[\sqrt{-5}]$. (Example 174).\\

\textbf{Felina.} At the start of the Ring Theory section, we mentioned two equations:\begin{enumerate}

\item $x^2 -2 = -1$ we could have solved (in $\Z$) then.
 
\item  The other was $y^3 = x^2 + 2$. \textit{ Can we find all solutions $x,y \in \Z$?}
  \end{enumerate}\vspace*{5pt}
  
  First notice that if $x^2 +2 = y^3$ then both $x$ and $y$ are odd. (It is clear that if one of them is even, then so is the other. Suppose both are even. Then mod $4$, we have $x^2 + 2 \equiv y^3$, and $x^2,y^3 \equiv 0$, so $2 \equiv 0 \mod 4$, a contradiction.)
  
 Move into $\Z[\sqrt{-2}]$. We have $x^2 +2 = (x + \sqrt{-2})(x-\sqrt{-2})$. Put $a = (x+\sqrt{-2})$ and $b = (x-\sqrt{-2}).$ So $y^3 = ab$. Hence $N(y^3) = N(a)N(b).$
 
  Let $d$ the hcf($a,b)$ in $\Z[\sqrt{-2}]$. Then $d$ divides $a-b = 2\sqrt{-2}$. So $N(d)$ divides $N(2\sqrt{-2}),$ so $N(d) $ divides $8$. But $N(d)$ divides $N(a)$, so divides $N(y)^3 = y^6$, which is odd. So $N(d) = 1$.
 
  So hcf$(a,b)$ is a unit, and so $a$ and $b$ are coprime. So we have $y^3 = ab$, where $a,b$ are co-prime. Since $\Z[\sqrt{-2}]$ is a UFD, we can factorise $y^3 = r_1^3 \dots r_t^3$, where $r_1,\dots,r_t$ are irreducible. Now for all $i$, we must have $r_i^3 ~|~ a$ or $r_1^3 ~|~ b$. So it's easy to see that $a=c^3$ and $b = d^3$ for $b,d \in \Z[\sqrt{-2}]$. 
  
  Let $c = m+n\sqrt{-2}$. So $(m + n\sqrt{-2})^3 = x + \sqrt{-2}$. So
  \[(m + n\sqrt{-2})^3  = (m^3 -6mn^2) + (3m^2n-2n^3)\sqrt{-2} = x+\sqrt{-2}\]
  So $m^3 - 6mn^2 = x$ and $3m^2n - 2n^3 = 1$. We have $(3m^2 - 2n^3)n = 1$, so $n~|~1$, hence $n = \pm 1$, and $3m^2 - 2n^2 = n$, so $3m^2 - 2 =\pm 1$. So $n = \pm 1, m = \pm 1$. Now we have $x = \pm 5$. So $x^2 + 2 = 27, y = 3$. So the only solutions are $x = \pm 5, y = 3$.
  
  
  \begin{center}
  
  \textsf{\textbf{- End of Algebra I -}}	
  \end{center}


 
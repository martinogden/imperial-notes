%!TEX root = linear-algebra.tex
\stepcounter{lecture}
\setcounter{lecture}{7}


\sektion{Systems of Particles}
\lecturemarker{26}{22 Oct}

\begin{definition}
\begin{itemize}
\item N: Total number of particles
\item $\vec{r}_i$: Position of particle $i$
\item $\vec{v}_i$: Velocity of particle $i$
\item $\vec{F}_i$: Force on particle $i$
 \item $m_i$: Mass of particle $i$
\end{itemize}	
\end{definition}

\vspace*{60pt}

Consider the average motion of the system:

\begin{definition}
\emph{Centre of Mass}, $\vec{r}_{cm}$:
\[\vec{r}_{cm} = \frac{\sum_{i=1}^Nm_i\vec{r}_i}{\sum_{i=1}^Nm_i} =\frac{\sum_{i=1}^Nm_i\vec{r}_i}{M}  \]
Where $M = \sum_{i=1}^Nm_i$ is the \emph{total mass}. 	
\end{definition}

\subsubsektion{Momentum}

The total momentum $\vec{p}$ is
\[\begin{aligned}\vec{p} = \sum_i\vec{p}_i = \sum_im_i\vec{v}_i &= \sum_i m_i\frac{d\vec{r}_i}{dt} \\
 &= \frac{d}{dt}(\sum_i m_i\vec{r}_i) \\
 &= \frac{d}{dt}(M\vec{r}_{cm}) \\
 &= M\frac{d\vec{r}_{cm}}{dt} = M\vec{v}_{cm}
\end{aligned}
\]

Where $\vec{v}_{cm}$ is the velocity of the centre of mass.


\[\vec{F}_i = \vec{F}_i^{EXT} + \sum_{j=1}^N \vec{F}_{ij}\]
where $\vec{F}_i^{EXT}$ is the external forces on particle $i$, $\vec{F}_{ij}$ is the force on $i$ due to $j$
\begin{example}
\vspace*{45pt}

Here $\vec{F}_{gi}$(Force due to gravity on $i$) is the only external force on $i \implies \vec{F}_i^{EXT} = \vec{F}_{gi}$
	
\end{example}

Note that
\begin{enumerate}
\item $\vec{F}_{ii} = \vec{0}$
\item $\vec{F}_{ij} = -\vec{F}_{ji}$ By Newton's Third Law
\end{enumerate}

\begin{theorem}[Newton's Second Law for a System]
The external force is equal to the rate of change of momentum of the centre of mass
	\[M\frac{d\vec{v}_{cm}}{dt} = \vec{F}^{EXT}\]
	Where the total external force on the system $\vec{F}^{EXT} = \sum_i\vec{F}_i^{EXT}$.
\end{theorem}
\begin{proof}

For particle $i$, \[\dfrac{d\vec{p}_i}{dt} = \vec{F}_i = \vec{F}_i^{EXT} + \sum_{j=1}^N\vec{F}_{ij}\] 
\[\implies \sum_i \dfrac{d\vec{p}_i}{dt} = \sum_i\vec{F}_i = \sum_i\vec{F}_i^{EXT} + \sum_i\sum_j\vec{F}_{ij} \]
Due to Newton's Third Law $\sum_i\sum_j\vec{F}_{ij} = \vec{0}$. We are then left with 
\[\begin{aligned}\sum_i\dfrac{d\vec{p}_i}{dt} &= \sum_i\vec{F}_i^{EXT}\\
\implies \dfrac{d}{dt}(\sum_i \vec{p}_i) &= \vec{F}^{EXT}\end{aligned}
\]
\[\implies M\frac{d\vec{v}_{cm}}{dt} = \vec{F}^{EXT}\qedhere\]
\end{proof}

\begin{enumerate}
\item If there is no external forces then 
\[M \frac{d\vec{v}_{cm}}{dt} = 0 = \frac{d\vec{p}}{dt}\]
(The conservation of momentum)
\item	If there are external forces then the centre of mass moves as though it were a point particle of mass $m$ subject to force $\vec{F}^{EXT}$
\end{enumerate}


\subsektion{Two Body Problems}
\vspace*{50pt}

$\vec{F}_1 = m_1g \hat{i} + \vec{F}_{12}$

$\vec{F}_2 = m_2g \hat{i} + \vec{F}_{21}$

The total external force:
\[\vec{F}^{EXT} = m_1g\hat{i} + m_2g\hat{i} = Mg\hat{i} ~(M = m_1 + m+2)\]
Thus
\[M\frac{d\vec{v}_{cm}}{dt} = Mg\hat{i} \implies \frac{d\vec{v}_{cm}}{dt} = g\hat{i}\]

For two body problems this is half of the information.

\begin{equation}
m_1\frac{d^2\vec{r}_1}{dt^2} = \vec{F}_1^{EXT} + \vec{F}_{12}	
\end{equation}

\begin{equation}
m_2\frac{d^2\vec{r}_2}{dt^2} = \vec{F}_2^{EXT} + \vec{F}_{21}	
\end{equation}
Calling $\dfrac{m_1\vec{r}_1 + m_2\vec{r}_2}{m_1 + m_2}$, and adding the equations
\setlength{\jot}{10pt}
\[\begin{aligned}m_1\frac{d^2\vec{r}_1}{dt^2} + m_2\frac{d^2\vec{r}_2}{dt^2} &= \vec{F}_1^{EXT} + \vec{F}_{2}^{EXT}\\
M \frac{d}{dt}\left(\frac{m_1\vec{v}_1 + m_2\vec{v}_2}{M}\right) &=  \vec{F}_1^{EXT} + \vec{F}_{2}^{EXT}\\
M\frac{d\vec{v}_{cm}}{dt} &=  \vec{F}_1^{EXT} + \vec{F}_{2}^{EXT}
\end{aligned}\]

\lecturemarker{27}{22 Oct}
Consider: $m_2 \times (7.4) - m_1\times(7.3)$
\[m_1m_2\frac{d^2}{dt^2}(\vec{r}_1-\vec{r}_2) = m_2\vec{F}_1^{EXT} + m_1\vec{F}_2^{EXT} + m_2\vec{F}_{12} - m_1\vec{F}_{21}\]
Call $\vec{r}_{12} = (\vec{r}_1 - \vec{r}_2)$. Since $\vec{F}_{12} = -\vec{F}_{21}$
\[m_1m_2\frac{d^2\vec{r}_{12}}{dt^2} = m_2\vec{F}_1^{EXT} + m_1\vec{F}_2^{EXT} + (m_1 + m_2)\vec{F}_{12}\]
Divide through by $M$
\[\frac{m_1m_2}{M}\frac{d^2\vec{r}_{12}}{dt^2} = \frac{m_2\vec{F}_1^{EXT} + m_1\vec{F}_2^{EXT}}{M} + \vec{F}_{12}\]
\begin{definition} Introduce $\mu = \dfrac{m_1m_2}{M}$, the \emph{reduced mass}.
\end{definition}
Then for our two body system we have:
\begin{equation}
M\frac{d\vec{v}_{cm}}{dt} =  \vec{F}_1^{EXT} + \vec{F}_{2}^{EXT}
\end{equation}
\begin{equation}
\mu\frac{d^2\vec{r}_{12}}{dt^2} = \frac{m_2\vec{F}_1^{EXT} + m_1\vec{F}_2^{EXT}}{M} + \vec{F}_{12}
\end{equation}~

If $\vec{F}_1^{EXT} = \vec{F}^{EXT}_2 = 0$, then $M\dfrac{d\vec{v}_{cm}}{dt} = 0$, and $\mu\dfrac{d^2\vec{r}_{12}}{dt^2}  = \vec{F}_{12}$.

If $\vec{F}_1^{EXT} = -m_1g\hat{j}$ and $\vec{F}^{EXT}_2 = -m_2g\hat{j}0$, then $M\dfrac{d\vec{v}_{cm}}{dt} = -Mg\hat{j}$, and $\mu\dfrac{d^2\vec{r}_{12}}{dt^2}  = \vec{F}_{12}$.



\begin{example}[Spring]
\vspace*{50pt}

Speing has a spring constant $k$ and equilibrium lnegth $l$. 
\[\vec{F}_{12} = -k(x_1 - x_2 - l)\hat{i}\]

Initially $x_1(0) = k,~ \dot{x}_1 = v_0$. $x_2(0) = \dot{x}_2(0) = 0$.

$\vec{F}_{12}$ is the only force in the $\hat{i}$ direction. No external forces in the $\hat{i}$ direction. 
\[\implies M\ddot{x}_{cm} = 0 \implies \dot{x}_{cm} = C\]
We can find $C$ using the conservation of momentum
\[\vec{p} = m\dot{x}_1 + m\dot{x}_2 = M\dot{x}_{cm}\]
At $t = 0, \dot{x}_1 = v_0$ and $\dot{x}_2 = 0$. Then $p = mv_0$. Since $M = 2m$:
\[\dot{x}_{cm} = v_0/2\]

For $x_{12} = x_1 - x_2$
\[\mu = \frac{m_1m_2}{m_1 + m_2} = \frac{m^2}{2m} = \frac{m}{2}\]
\[\vec{F}_{12} = -k(x_1 - x_2 - l) = -k(x_{12} - l)\]

Using the equation for $\vec{r}_{12}$
\[\begin{aligned}\mu\ddot{x}_{12} &= \vec{F}_{12}\\
\frac{m}{2}\ddot{x}_{12} &= -k(x_{12} - l)\\
\ddot{x}_{12} + \frac{2k}{m}x_{12} &= \frac{2kl}{m}	
\end{aligned}
\]

The general solution is 
\[x_{12} = A\cos\omega t + B\sin\omega t + l\]
where $\omega^2 = \frac{2k}{m}$. 

From our initial conditions $x_{12}(0) = x_1(0) - x_2(0) = l$ and $\dot{x}_{12} = v_0$. 
\[\implies A = 0,~B = v_0/\omega\]Thus
\[x_{12} = \frac{v_0}{\omega}\sin\omega t + l\]
\[\dot{x}_{12} = v_0\cos\omega t\]

We can show that (in general)
\[\vec{r}_1 = \vec{r}_{cm} + \vec{m_2}{M}\vec{r}_{12}\]
\[\vec{r}_2 = \vec{r}_{cm} + \vec{m_1}{M}\vec{r}_{12}\]

Thus
\[x_1 = x_{cm} + \frac{1}{2}x_{12}\]
\[\dot{x}_1 = \dot{x}_{cm} + \frac{1}{2}\dot{x}_{12} = \frac{v_0}{2} + \frac{1}{2}v_0\cos\omega t = \frac{v_0}{2}(1+\cos\omega t)\]
Similarly 
\[\dot{x}_2 =  \frac{v_0}{2}(1 - \cos\omega t)\]
This is a push-me-pull-you system.
\end{example}

\emph{What about more than two particles?}\\

\begin{definition}[Centre of Mass Coordinates]
\[\vec{R}_i = \vec{r}_i - \vec{r}_{cm}\]	
This is the position of particle $i$ relative to the position of the centre of mass
\end{definition}

\[\sum_im_i\vec{R}_i = \underbrace{\sum_im_i\vec{r}_i}_{M\vec{r}_{cm}}- \vec{r}_{cm}\underbrace{\sum_im_i}_{M} = 0\]








\subsubsektion{Kinetic Energy}
\lecturemarker{28}{22 Oct}
\[T = \sum_i \frac{1}{2}m_iv_i^2\]
We can write $\vec{v}_i = \vec{v}_{cm} + \dfrac{d\vec{R}_i}{dt},~\vec{u}_i = \dfrac{d\vec{R}_i}{dt}$, so $\vec{v}_i = \vec{v}_{cm} + \vec{u}_i$

\[\begin{aligned}T &= \sum_i \frac{1}{2}m_i(\vec{v}_{cm}+\vec{u}_i) \cdot(\vec{v}_{cm}+\vec{u}_i)\\
&= \sum_i \frac{1}{2}[v_{cm}^2 + 2\vec{u}_i\cdot\vec{v}_{cm} + u_i^2\\
&= 	\frac{1}{2}v_{cm}^2\sum_i m_i + \vec{v}_{cm}\cdot\sum_im_i\vec{u}_i + \frac{1}{2}\sum_im_iu_i^2\\
&= \frac{1}{2}Mv_{cm}^2 + \frac{1}{2}\sum_im_iu_i^2 + \vec{v}_{cm}\cdot\sum_im_i\vec{u}_i
\end{aligned}
\]
Consider $\sum_i m_i\vec{u}_i = \sum_im_i\dfrac{d\vec{R}_i}{dt} = \dfrac{d}{dt}(\sum_i m_i\vec{R}_i) = 0$. Then 
\begin{equation}\boxed{T = \frac{1}{2}Mv_{cm}^2 + \sum_i\frac{1}{2}m_iu_i^2}\end{equation}

\subsektion{Angular Momentum}

\[\vec{J} = \vec{r} \times \vec{p} = \vec{r} \times m\vec{v}\]
For central forces where the motion was restricted to a plane $\vec{J} = mh\hat{k} =$ constant vector. 

\emph{What causes $\vec{J}$ to change?}

\[\begin{aligned}\frac{d\vec{J}}{dt} &= \frac{d\vec{r}}{dt} \times m \vec{v} + \vec{r}\times \frac{d\vec{v}}{dt}\\ &= m\cancelto{0}{[\vec{v} \times \vec{v}]} + \vec{r} \times \vec{F} = \vec{\tau}\end{aligned}
\]

\begin{definition}
$\vec{\tau} = \vec{r} \times \vec{F}$ is the \emph{Torque} or the \emph{Moment}. 
\vspace*{50pt}
\begin{itemize}
\item $\vec{\tau}$ is in the direction out of the screen
\item $|\vec{\tau}| = |\vec{F}||\vec{r}|\sin\phi$ 	
\end{itemize}
\end{definition}

For central forces 
\vspace*{50pt}

Since $\phi = 0 \implies \vec{\tau} = 0$. 

For a system, the total angular momentum
\[\vec{J} = \sum_{i}\vec{J}_i = \sum_i\vec{r}_i \times m_i \vec{v}_i\]
\[\implies \vec{\tau} = \frac{d\vec{J}}{dt} = \sum_i\dfrac{d\vec{J}_i}{dt} = \sum_i\vec{r}_i\times\vec{F}_i\]
Write $\vec{F}_i = \vec{F}_i^{EXT} + \sum_j\vec{F}_{ij}$. Then we have
\begin{equation} \vec{\tau} \frac{d\vec{J}}{dt} = \sum_i\vec{r}_i\times\vec{F}_i^{EXT} + \sum_i\sum_j\vec{r}_i\times\vec{F}_{ij}\end{equation}


\begin{theorem}[Conservation of Angular Momentum for a System]
If there is no net torque, the angular momentum is conserved.
\end{theorem}

\begin{proof}[Proof (for two body system)]

Suppose we have two particles. Then the double sum is 
\[\vec{r}_1 \times\vec{F}_{12} +\vec{r}_2 \times\vec{F}_{21} \]

By Newton's Third Law $\vec{F}_{12} = -\vec{F}_{21}$. Thus
\[\vec{r}_1 \times\vec{F}_{12} +\vec{r}_2 \times\vec{F}_{21} = (\vec{r}_1 - \vec{r}_2) \times \vec{F}_{12}\]

If $\vec{F}_{12}$ is parallel to $\vec{r}_1 - \vec{r}_2$, then $(\vec{r}_1 - \vec{r}_2) \times \vec{F}_{12} = 0$. 

This is the case if $\vec{F}_{12}$ is a central force, i.e. no torque. 

Thus if $\vec{F}_{ij}$ is a central force for all $i$ and $j$. Then 
\[\sum_i\sum_j \vec{r}_i \times \vec{F}_{ij} = \vec{0}\]

Then 
\[\frac{d\vec{J}}{dt} = \sum_i\vec{r}_i \times \vec{F}_i^{EXT} = \vec{\tau}^{EXT}\]
where $\vec{\tau}^{EXT}$ is the total external torque on the system.

So if $\vec{\tau}^{EXT} = \vec{0}$ then $\dfrac{d\vec{J}}{dt} = \vec{0}$, hence the angular momentum is conserved.
\end{proof}

\begin{example}
\vspace*{50pt}

Each particle has mass $m$. Each mass has velocity $\vec{v}_i = \vec{\omega} \times \vec{r}_i$, with $\vec{\omega} = \omega\hat{k}$	

The angular momentum of particle $i$ is:
\[\vec{J}_i = \vec{r}_i \times m_i\vec{v}_i = m[\vec{r}_i \times (\vec{\omega} \times \vec{r}_i)]\]

Recall that $\vec{A} \times (\vec{B} \times \vec{C}) = (\vec{A}\cdot\vec{C})\vec{B} - (\vec{A}\cdot\vec{B})\vec{C}$
\[\vec{r}_i \times (\vec{\omega} \times \vec{r}_i) = (\vec{r}_i \cdot \vec{r}_i)\vec{\omega} - \cancelto{0}{(\vec{r}_i \cdot \vec{\omega})}\vec{r}_i = r^2\omega\hat{k}\]

Thus 
\[\vec{J}_i = mr^2\omega\hat{k}\]
\[\implies \vec{J} = \sum_i \vec{J}_i = 4mr^2\omega\hat{k} = 2ml^2\omega\hat{k}\]

Suppose that 
\vspace*{50pt}

$\vec{v}_i = \vec{\omega} \times \vec{r}_i \longrightarrow \vec{v}_i = \vec{\Omega} \times\vec{r}_i$. What's $\vec{\Omega}$?

Single the configuration changed to to internal, central forces, $\dfrac{d\vec{J}}{dt} = 0$

For our new configuration
\[\vec{J}_i = 2m[\vec{r}_i \times (\vec{\Omega} \times \vec{r}_i)] = 2mr_i^2\Omega\hat{k} = \frac{ml^2\Omega}{2}\hat{k}\]
The total angular momentum
\[\vec{J} = 2\vec{J}_i = ml^2\Omega\hat{k}\]

Since $\dfrac{d\vec{J}}{dt} = 0 \implies \vec{J}_{before} = \vec{J}_{after}$

\[\implies 2ml^2\omega\hat{k}= ml^2\Omega\hat{k}\]
\[\implies \Omega = 2\omega\]

The angular speed doubles as a result of the change. 
\end{example}~\\

\subsubsektion{Centre of Mass Coordinates}
\lecturemarker{29}{22 Oct}

\[\vec{r}_i = \vec{r}_{cm} + \vec{R}_i\]
\[\vec{v}_i = \vec{v}_{cm} + \vec{u}_i,~\left(\vec{u}_i = \frac{d\vec{R}_i}{dt}\right)\]
Thus
\[\begin{aligned}
\vec{J} &= \sum_i (\vec{r}_{cm} + \vec{R}_i) \times m_i(\vec{v}_{cm} + \vec{u}_i)\\
&= \sum_i \vec{r}_{cm} \times m_i\vec{v}_{cm} + \sum_i \vec{r}_{cm} \times m_i\vec{u}_i + \sum_i \vec{R}_i \times m_i\vec{v}_{cm} + \sum_i \vec{R}_i \times m_i\vec{u}_i\\
&= \vec{r}_{cm}\times \vec{v}_{cm}(\sum_i m_i) + \vec{r}_{cm} \times (\sum_i m_i\vec{u}_i) + (\sum_im_iR_i)\times\vec{v}_{cm} + \sum_iR_i\times m_i\vec{u}_i
\end{aligned}
\]

We know that $\sum_i m_i = M$, $\sum_i m_i\vec{R}_i = \sum_i m_i \vec{u}_i = 0$. Thus
\[\vec{J} = \vec{r}_{cm} \times M\vec{v}_{cm} + \sum_i\vec{R}_i\times m_i\vec{u}_i\]

%Contribution to centre of mass, and motion relative to centre of mass.
Call $\vec{J}_{cm} = \sum_i \vec{R}_i \times m_i\vec{u}_i$

Recall that 
\[\frac{d\vec{J}}{dt} = \sum_i \vec{r}_i \times \vec{F}_i^{EXT} (= \vec{\tau}^{EXT})\]
Since $\vec{r}_i = \vec{r}_{cm} + \vec{R}_i$
\[\begin{aligned} \frac{d\vec{J}}{dt} &= = \sum_i \vec{r}_{cm} \times \vec{F}_i^{EXT}\\
&= \vec{r}_{cm} \times \vec{F}^{EXT} + \sum_i\vec{R}_i\times\vec{F}_i^{EXT}
\end{aligned}
\]

We can show (P.S. 4 Problem 9)
\[\frac{d\vec{J}_{cm}}{dt} = \sum_i\vec{R}_i\times\vec{F}_i^{EXT}\]
Call
\[\vec{\tau}_{cm}^{EXT} = \sum_i\vec{R}_i \times \vec{F}_i^{EXT}\]

\subsubsektion{Complete Picture}
\begin{enumerate}
\item Momentum: \[\vec{p} = M\vec{v}_{cm}\]
\[\frac{d\vec{p}}{dt} =  M\frac{d\vec{v}_{cm}}{dt} = \vec{F}^{EXT}\]
\item Angular Momentum: \[\vec{J} = \vec{r}_{cm} \times M\vec{v}_{cm} + \vec{J}_{cm}\]
\[\vec{J}_{cm} = \sum_i\vec{R}_i \times m_i\vec{u}_i\]
\[\frac{d\vec{J}}{dt} = \vec{r}_{cm} \times \vec{F}^{EXT} + \vec{\tau}_{cm}^{EXT}\]
\end{enumerate}




















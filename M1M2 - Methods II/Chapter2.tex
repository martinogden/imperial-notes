%!TEX root = linear-algebra.tex
\stepcounter{lecture}
\setcounter{lecture}{2}

\pagebreak

\section{Multivariable Calculus}

\subsection{Stationary Points} % (fold)
\label{sub:stationary points}

\lecturemarker{25}{5 March}

\begin{definition}
For $u(\vec{x})$ with $\vec{x} \in \mathbb{R}^2$, the \emph{stationary points}, $\vec{x}^*$, occur when
\[\frac{\del u}{\del x}(\vec{x}^*) = \frac{\del u}{\del y}(\vec{x}^*) = 0\]
\end{definition}

\textit{What about the character? Maximum / Minimum / Saddle Point.}

\begin{example}Looking at $f(x)$, $x \in \mathbb{R}$:
\[f(x) = f(x^*) + \cancel{\frac{df}{dx}(x^*)\delta x} + \frac{1}{2} \frac{d^2f}{dx^2}(x^*)(\delta x)^2 + \cal{O}(\delta x^3)\]
\[\implies f(x) - f(x^*) = \frac{1}{2} \frac{d^2f}{dx^2}(x^*)(\delta x)^2 + \cal{O}(\delta x^3)\]

So  $\frac{d^2f}{dx^2}(x^*) > 0 \iff f(x) - f(x^*) > 0$. Hence local minimum when $f(x) -f(x^*) > 0$.
\end{example}

For $u = u(x,y)$, consider the taylor expansion about a stationary point $(x^*,y^*) = \vec{x}^*$  (so $\frac{\del u}{\del x}(\vec{x}^*) = \frac{\del u}{\del y}(\vec{x}^*) = 0$):
\[u(\vec{x}^* + \delta \vec{x}) = u(\vec{x}^*) + \cancel{\left[\frac{\del u}{\del x}(\vec{x}^*)\delta x + \frac{\del u}{\del y}(\vec{x}^*)\delta y \right]} 
+ \frac{1}{2} \delta\vec{x}^* \equivto{\begin{pmatrix}
 \frac{\del^2u}{\del x^2} & \frac{\del^2u}{\del x\del y}\\
 \frac{\del^2u}{\del y \del x} &\frac{\del^2u}{\del y^2}	
 \end{pmatrix}_{\vec{x}^*}}{H(\vec{x}^*)}\hspace*{-10pt}\delta \vec{x} + \mathcal{O}(||\delta \vec{x}||^3)
\]
Thus:
\[u(\vec{x}^* + \delta\vec{x}) - u(\vec{x}^*) = \frac{1}{2}\delta \vec{x}^* H(\vec{x}^*) \delta \vec{x} + \mathcal{O}(||\delta \vec{x}||^3)\]
So we get:
\[
\text{ Local }
\begin{cases}
\text{Minimum: } &\forall \delta\vec{x}^*,~\delta \vec{x}^* H(\vec{x}^*) \delta \vec{x} > 0 \\
& ||\delta\vec{x}|| \text{ is small }\\

\text{Maximum: } &\forall \delta\vec{x}^*,~\delta \vec{x}^* H(\vec{x}^*) \delta \vec{x} < 0 \\
& ||\delta\vec{x}|| \text{ is small }

\end{cases}\]

Note that $\forall \vec{x},~ \vec{x}^*A\vec{x} > 0 \iff A \text{ is positive definite} \iff \text{All eigenvalues } \lambda_i \text{ are positive.}$

So for $H(\vec{x}^*) = \begin{pmatrix}
 A & B \\ B & C	
 \end{pmatrix}
$ (continuity $\implies \frac{\del^2u}{\del x \del y} = \frac{\del^2u}{\del y \del x} \implies H(\vec{x}^*) = H^T(\vec{x}^*)$), we have that:
\begin{enumerate}
\item[(1)] $H(\vec{x}^*)$ is always diagonalisable: 
\[v^{-1}Hv = \Lambda = \begin{pmatrix}
 \lambda_1 & 0 \\ 0 & \lambda_2	
 \end{pmatrix}
\]

\item[(2)] $v$ is orthogonal (i.e. $v^{-1} = v^T$), so: 
\[v^THv = \Lambda \implies H = v \Lambda v^T\]

\item[(3)] All eigenvalues are real:
\[\begin{pmatrix}
A & B \\ B & C	
\end{pmatrix},~
\lambda = \frac{\tau \pm \sqrt{\tau^2 - 4\Delta}}{2}
,~\tau^2-4\Delta > 0\]

\[\begin{aligned}
(A+C)^2 - 4(AC - B^2) &= A^2 + C^2 + 2AC - 4AC + 4B^2\\
&= A^2 + C^2 - 2AC + 4B^2\\
&= (A-C)^2 + 4B^2 > 0	
\end{aligned}
\]

\end{enumerate}


\begin{theorem}
$H = H^T$ is positive definite $\iff \lambda_1,\lambda_2$ are positive.	
\end{theorem}

\begin{proof}
$\lambda_1,\lambda_2 > 0 \implies$ positive definite:
\[\begin{aligned}\vec{x}^TH\vec{x} = \vec{x}^TV\Lambda V^T\vec{x} 
&= \vec{x}^T(\vec{v_1}~\vec{v_2})\begin{pmatrix}
\lambda_1 & 0 \\ 0 & \lambda_2	
\end{pmatrix}
\begin{pmatrix}
\vec{v_1}^T\\ \vec{v_2}^T	
\end{pmatrix}\vec{x}\\
&= (\vec{x}^T\cdot\vec{v_1} ~\vec{x}^T\cdot\vec{v_2})
\begin{pmatrix}
\lambda_1 & 0 \\ 0 & \lambda_2	
\end{pmatrix}
\begin{pmatrix}
\vec{v_1}^T\cdot\vec{x}\\ \vec{v_2}^T\cdot\vec{x}	
\end{pmatrix}\\
&= \lambda_1(\vec{x}^T\vec{v_1})^2 + \lambda_2(\vec{x}^T\vec{v_2})^2 > 0
\end{aligned}
\]

Positive definite $\implies \lambda_1,\lambda_2 > 0$:
\[\vec{x}^TH\vec{x} = \lambda_1(\vec{x}^T\vec{v_1})^2 +\lambda_2(\vec{x}^T\vec{v_3})^2  \]
Assume $\vec{x} = \vec{v_1}$, then 
$\begin{cases}
 	\vec{v_1}^T\cdot\vec{v_2} = 0\\
    \vec{v_1}^T\cdot\vec{v_1} = 1
 \end{cases}
 $
from $VV^T = V^TV = I$

\[\vec{v_1}^TH\vec{v_1} = \lambda_1\cdot 1 + \lambda_2 \cdot 0 \text{ and } \vec{v_2}^TH\vec{v_2} = \lambda_2\]
So $\lambda_1,\lambda_2 >0$ is necessary so that $\vec{x}^TH\vec{x} > 0$
\end{proof}

Going back to the character of stationary points: \begin{enumerate}
 \item Minimum: 
 \item Maximum:
 \item Saddle-Point: 
 \item If $\lambda_1$ or $\lambda_2$ or both are zero, we need to go to higher derivatives	
 \end{enumerate}



\begin{example}
$u(x,y) = (x-y)(x^2 + y^2 -1)$
\begin{enumerate}
\item Contour lines for $u= 0$, $(x-y)(x^2 + y^2 - 1) = 0 \implies$
\item Stationary points:	
\end{enumerate}
	
\end{example}





%!TEX root = linear-algebra.tex
\stepcounter{lecture}
\setcounter{lecture}{1}

\pagebreak

\section{Basics}
\setcounter{page}{4}

\subsection{Sets} % (fold)
\label{sub:sets}

\lecturemarker{1}{5 Oct}
A set $S$ is a collection of objects (called the \emph{elements} of the set) 
\begin{example}
A way to specify a set is to list the objects (between curly brackets):
\[S = \{1,3,7\}\]	
The order of elements is unimportant, as is repetition:
\[\{1,2\} = \{2,1\} = \{1,1,2\}\]
\end{example}

I say $S_1 \subset S_2$ ($S_1$ \emph{contained} in $S_2$) if every element of $S_1$ is also an element of $S_2$. I can write this as a \emph{statement}: $x \in S_1 \implies x \in S_2$\\

I say $S_1 = S_2$ if $S_1 \subset S_2$ \& $S_2 \subset S_1$. Elements can be sets: $S = \{1,2,\{1,2\}\}$.\\
 But there is one thing you are never allowed to do:

\begin{axiom}[Foundation Axiom]
$S \not \in S$	
\end{axiom}

More things about sets:
\begin{itemize}
\item $a \not \in S$ ``$a$ is not an element of $S$''
\item $S_1 \cup S_2$ ``$S_1$ union $S_2$'' = $\{x ~|~ x \in S_1 \text{ or } x \in S_2 \text{ (or both)}\} = \{x : x \in S_1 \text{ or } x \in S_2\}$
\item $S_1 \cap S_2$ ``$S_1$ intersection $S_2$'' = $\{x ~|~ x \in S_1 \text{ and } x \in S_2\}$
\item $S_1 \backslash S_2$ ``$S_1$ take away $S_2$'' = $\{x ~|~ x \in S_1 \& x \not \in S_2\}$
\item $S_1 \triangle S_2$ ``symmetric difference'' = $\{x ~|~ x \in S_1 \text{ or } x \in S_2 \text{ but not both }\}$\\ = $(S_1 \cup S_2) \backslash (S_1 \cap S_2)$
\end{itemize}

When you reason about sets \& other mathematical objects, it is useful to draw pictures:

% Definition of circles
\def\firstcircle{(0,0) circle (1.5cm)}
\def\secondcircle{(0:2cm) circle (1.5cm)}

\colorlet{circle edge}{blue!50}
\colorlet{circle area}{blue!20}

\tikzset{filled/.style={fill=circle area, draw=circle edge, thick},
    outline/.style={draw=circle edge, thick}}

\setlength{\parskip}{5mm}
% Set A and B
\begin{minipage}{0.5\textwidth}
\begin{tikzpicture}
    \begin{scope}
        \clip \firstcircle;
        \fill[filled] \secondcircle;
    \end{scope}
    \draw[outline] \firstcircle node {$S_1$};
    \draw[outline] \secondcircle node {$S_2$};
    \node[anchor=south] at (current bounding box.north) {$S_1 \cap S_2$}; 
   
\end{tikzpicture}
\end{minipage}
\begin{minipage}{0.5\textwidth}
%Set A or B but not (A and B) also known a A xor B
\begin{tikzpicture}
    \draw[filled, even odd rule] \firstcircle node {$S_1$}
                                 \secondcircle node{$S_2$};
    \node[anchor=south] at (current bounding box.north) {$S_1\triangle S_2$};
\end{tikzpicture}
\end{minipage}

\begin{minipage}{0.5\textwidth}	
\begin{tikzpicture}
    \draw[filled] \firstcircle node {$S_1$}
                  \secondcircle node {$S_2$};
    \node[anchor=south] at (current bounding box.north) {$S_1 \cup S_2$};
\end{tikzpicture}
\end{minipage}
\begin{minipage}{0.5\textwidth}	
% Set A but not B
\begin{tikzpicture}
    \begin{scope}
        \clip \firstcircle;
        \draw[filled, even odd rule] \firstcircle node {$S_1$}
                                     \secondcircle;
    \end{scope}
    \draw[outline] \firstcircle
                   \secondcircle node {$S_2$};
    \node[anchor=south] at (current bounding box.north) {$S_1\backslash S_2$};
\end{tikzpicture}
\end{minipage}




Sometimes (often) it is not practical to list all the elements of a set: 
\begin{examples}	\shortskip
\item $\Z$ = set of integers = $\{0, +1,-1,2,-2,3,-3,\dots\}$	
\item $\N$ = set of natural numbers = $\{0, 1, 2, 3,\dots\} = \{n \in \Z ~|~ n \geq 0\}$	
\item $\Q$ = set of rational numbers = $\{x ~|~ x = \frac{p}{q},~ p \in \Z, q \in \N\backslash\{0\} \}$	
\item $\R$ = set of real numbers 	
\item $\C$ = set of complex numbers
 \[\N \subset \Z \subset \Q \subset \R \subset \C\]

\end{examples}


\lecturemarker{2}{5 Oct}


